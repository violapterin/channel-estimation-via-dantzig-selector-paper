\Frame {Conclusion}
{
\I We propose a feasible method to estimate MIMO mm-Wave channel \m {\M {H}} by exploiting its sparsity, and employing Dantzig Selector (DS).

\I We modified DS for present setting, for the almost-sparse condition in the angular domain, with respect to the effective beamformer \m {\M {P}} in our setting, and concerning complex vectors.

\I We bound the error norm with respect to parameters \m {N_H}, \m {N_Y}, and \m{L}, and investigated the big \m {\MC{O}} of the successful probability
}
% XXX % % XXX % % XXX % % XXX % % XXX % % XXX %
\Frame {Future Work}
{
\I Investigate the reason of high complexity and reduce it.

\I Run for bigger values of \m {N_H}, \m {N_Y}, and \m{L} if possible.

\I Maybe try the real case and use existent code for DS.

\I More literature review on other methods.
}
% XXX % % XXX % % XXX % % XXX % % XXX % % XXX %
\Frame {References}
{
{\tfx
\I E. Candès and J. Romberg, \It {\m {\ell_1}-magic: Recovery of Sparse Signals via Convex Programming}, retrieved from \Tt {www.acm.caltech.edu/l1magic/downloads/l1magic.pdf}, 2005.

\I E. Candès and T. Tao, ``The Dantzig Selector: Statistical estimation when \m {p} is much larger than \m {n}'', The annals of Statistics 35(6), 2313–2351, 2007.

\I M.P. Friedlander and M.A. Saunders, ``Discussion: the Dantzig Selector'', The Annals of Statistics 35(6), 2385–2391, 2007.

\I B. Klartag and E. Milman, \It {Geometric Aspects of Functional Analysis}, Springer, 2017.

\I B. Laurent and P. Massart, ``Adaptive estimation of a quadratic functional by model selection'', Annals of Statistics, 1302–1338, 2000.
}
}

