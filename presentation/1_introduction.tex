
\blank [big, force]

\Title {\TitleText}
\blank [big]

\Subtitle {\AuthorText}
\blank [big]

\Subsubtitle {\InstitutionText}

\page [yes]
% XXX % % XXX % % XXX % % XXX % % XXX % % XXX %
\Frame {Organization}
{
\I Problem Setting

\I Proposed Method

\I Error Analysis

\I Numerical Experiments

\I Conclusion
}
% XXX % % XXX % % XXX % % XXX % % XXX % % XXX %
\Frame {Background}
{
\I Estimation of Multiple-input multiple-output (MIMO) systems results in high complexity, due to a large number of antennae.

\I Millimeter wave (mm-Wave) channels are often sparse due its poor scattering nature

\I According to compressive sensing theory, we might exploit the sparsity to reduce complexity of estimation procedure.
}
% XXX % % XXX % % XXX % % XXX % % XXX % % XXX %
\Frame {Terminology}
{
\I \m {\V {g}} is called \Emph {\m {s}-sparse}, if at most \m {s} of its components are nonzero.

\I We say that \m {\M {P}} satisfies the \Emph {restricted isometry property} (RIP) of sparsity \m {s} with respect to \m {0 < \d_s < 1}, if, for all \m {s}-sparse \m {\V {g}},
\Disp {
\NC \RB {1-\d_s} \VNm {\V {g}} _2^2
\leq \NC \VNm {\m {P} \V {g}} _2^2
\leq \RB {1+\d_s} \VNm {\V {g}} _2^2
}

\I We may think \m {\M {P}} to be almost unitary up to relative error \m {\d_s}.
}


