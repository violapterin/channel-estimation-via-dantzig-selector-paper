
\blank [big, force]

\Title {\TitleText}
\Subtitle {II. Completing the Proof}
\blank [big]

\Subtitle {\AuthorText}
\blank [big]

\Subsubtitle {\DateText}

\page [yes]
% % % % % % % % % % % % % % % % % % % % % % % % % %
\Frame {Organization}
{
\I Introduction

\I Problem setting and proposed method

\I Proving the performance bound: sparsity, concentration inequality, and expected error

\I Simulation (preliminary)

\I Conclusion and future work

\I References
}
% % % % % % % % % % % % % % % % % % % % % % % % % %
\Frame {Background}
{
\I \Emph {Multiple-input-multiple-output} (MIMO) wireless communication is expected to be the next-generation r\'egime.

\I But channel response has to be known at the receiver to facilitate such algorithms as beamforming and channel calibration.

\I It calls for high complexity to estimate MIMO channels.

\I Meanwhile, \Emph {millimeter wave} (mmWave) channels often exhibit sparse properties.

\I Under mmWave setting, if the \Emph {sparsity} is exploited, it turns out few observations may suffice to estimate the channel.
}
% % % % % % % % % % % % % % % % % % % % % % % % % %
\Frame {Compressive Sensing}
{
\I The situation that the number of model parameters is much larger than the number of measurements has been well studied, and and \Emph {compressed sensing} (CS) address exactly that.

\I The problem can be considered to be inverting a wide, rectangular matrix.
With insufficient (even noisy) measurements, this does not make sense.

\I But what if we impose other conditions?
Indeed, it turns out that, when the signal is sparse, few observations may be sufficient for the reconstruction.
And Dantzig Selector (DS), proposed by Cand\`es and Tao (2006) in the advent, is a possible solution.
}

