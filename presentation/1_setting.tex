\blank [big, force]

\Title {\TitleText}
\blank [big]

\Subtitle {\AuthorText}
\blank [big]

\Subsubtitle {\InstitutionText}
\Subsubtitle {November 2020}

\page [yes]
% XXX % % XXX % % XXX % % XXX % % XXX % % XXX %
\Frame {System Configuration}
{
\blank [big]
\externalfigure [system.png] [factor=fit]

\I Estimation of Multiple-input multiple-output (MIMO) systems results in high complexity, due to a large number of antennae.

\I Problem (a): How to reconstruct the channel with fewer measurements?

\I Problem (b): How to design precoders and combiners?
}
% XXX % % XXX % % XXX % % XXX % % XXX % % XXX %
\Frame {Channel Model and Beamformers}
{
\I Given MIMO channel representation \m {\M {H} \in \NC \MB {C} ^{N_H \D N_H}} with uniform linear array, having \m {L} paths, with hybrid beamforming.

\I Complex standard normal noise \m {\M {Z} \in \NC \MB {C} ^{N_H \D N_H}}

\I Tx: Digital precoder \m {\M {F}_B \in \NC \MB {C} ^{N_R \D N_B}}
\I Tx: Analog precoder \m {\NC \M {F}_R \in \NC \MB {C} ^{N_H \D N_R}}

\I Rx: Digital combiner \m {\M {W}_B \in \NC \MB {C} ^{N_R \D N_H}}
\I Rx: Analog combiner \m {\M {W}_R \in \NC \MB {C} ^{N_B \D N_R}}

\I Magnitude constraint \m {\Nm {\M {F}_{R,\SB {n_H, n_R}}} = 1}, \m {\Nm {\M {W}_{R,\SB {n_H, n_R}}} = 1}
}
% XXX % % XXX % % XXX % % XXX % % XXX % % XXX %
\Frame {Problem Description}
{
\I We have \m {\M {Y} =\M {W}_B \M {W}_R \RB {\M {H} \M {F}_R \M {F}_B +\M {Z}}}
%
\I Problem (a): How to recover \m {\M {H}} with knowledge of \m {\M {Y}}?

\I Can we measure less than \m {N_H^2} times, but recover successfully for most cases?

\I Problem (b): How to design \m {\M {W}_R}, \m {\M {W}_B}, \m {\M {F}_R}, and \m {\M {F}_B}?

\I What kind of matrices facilitates our goal?
}
% XXX % % XXX % % XXX % % XXX % % XXX % % XXX %
\Frame {Cast to Linear System}
{
\I Introduce the DFT matrix \m {\M {K} \in \NC \MB {C} ^{N_H \D N_H}}, and write
\Disp {
\NC \V {y}
:= \NC \Rm {vec} \SB {\M {Y}},
\Q \M {P}
:= \RB {\M {F}_B^\Tr \M {F}_R^\Tr \m {K}^\ast} \otimes \RB {\M {W}_B \M {W}_R \M {K}}, \NR
\NC \V {g}
:= \NC \Rm {vec} \SB {\M {K}^\Adj \M {H} \M {K}},
\Q \V {z}
:= \Rm {vec} \SB {\M {W}_B \M {W}_R \M {Z}} \NR
}

\I We have an underdetermined system \m {\V {y} =\M {P} \V {g} +\V {z}}
}
% XXX % % XXX % % XXX % % XXX % % XXX % % XXX %
\Frame {Compressive Sensing}
{
\blank [big]
\externalfigure [compressive-sensing.png] [factor=fit]

\I If \m {\V {g}} is sparse, fewer measurements suffice for reconstruction.

\I Can we find a basis on which \m {\V {g}} is likely to be sparse?
}
% XXX % % XXX % % XXX % % XXX % % XXX % % XXX %
\Frame {Literature on Compressive Sensing}
{
\I Bajwa, Haupt, Sayeed, and Nowak (2010): MIMO, linear invariant system, using Dantzig Selector (DS)

\I Alkhateeb, Ayach, Leus, Heath (2014): mm-wave, adaptive training algorithm for given sensing matrix

\I Alkhateeb, Leus, and Heath (2015): single path mm-wave, all-phase-shifter beamformers, using Orthogonal Matching Pursuit (OMP)

\I Lee, Lee, and Yong (2016): MIMO mm-wave, hybrid beamforming, using Orthogonal Matching Pursuit (OMP)
}
% XXX % % XXX % % XXX % % XXX % % XXX % % XXX %
\Frame {Questions}
{
\I Does greedy algorithms (like OMP) trade precision for complexity, and is the opposite true of convex programs (like DS)?

\I We consider the hybrid system as in Lee, Lee, and Yong (2016), where they proposed an optimal joint design by a nonconvex program, but made considerable approximation.
Is the simplified solution still optimal?

\I Candès and Tao (2007) derived a performance bound for DS.
Does it prove DS to be better?
Does it apply to our case?
}
% XXX % % XXX % % XXX % % XXX % % XXX % % XXX %
\Frame {Contribution}
{
\I We demonstrate that the beamformer may serve as the sensing matrix, so DS can be applied.

\I We give a bound (holding for high probability) on expected error norm.

\I Numerical results illustrate that in our configuration DS is superior to OMP among other methods, at the cost of higher complexity.
}
% XXX % % XXX % % XXX % % XXX % % XXX % % XXX %
\Frame {Design of Beamformers}
{
\I We set each entry of \m {\m {F}_B} (resp.\ \m {\m {W}_B}) to be i.i.d.\ complex standard normal, up to a design constant

\I We set each entry of \m {\m {F}_R} (resp.\ \m {\m {W}_R}) be uniformly distributed in argument and unity magnitude, up to a design constant

\I We argue that \m {\M {F} _R ^\Tr}, \m {\M {F} _B ^\Tr}, \m {\M {W} _R}, and \m {\M {W} _B} all have restricted isometry, thus \m {\M {P}} has too.
}
% XXX % % XXX % % XXX % % XXX % % XXX % % XXX %
\Frame {Applying Dantzig Selector}
{
\I Input \m {\M {P} \in \MB {C} ^{N_B^2 \D N_H^2}}, \m {\V {y} \in \MB {C} ^{N_H^2}}, \m {\g_{\Ss {DS}} \geq 0}.

\I Calculate
\Disp {
\NC \Hat {\V {g}}
\LA \NC \startcases
\NC \Min {\V {g}'}  \MC \VNm {\V {g}'} _1 \NR
\NC \Rm {subject} \; \Rm {to} \Q \MC \VNm {\M {P}^\Adj \RB {\V {y} -\M {P} \V {g}'}} _\infty \leq \g_{\Ss {DS}} \NR
\stopcases \NR
}

\I Output \m {\Hat {\M {H}} \LA \M {K} \Rm {vec}^{-1} \SB {\Hat {g}} \M {K}^\Adj}
}


