\startsection [title={Bound for error norm}]

\startsubsection [title={Technical propositions in complex case}]

The third and last part of our task is to plug in the almost-sparsity found in Section 3.1, and the failure probability found in Section 3.2.
Set for short
%
\DispNum {d:d:gg:gg} {
\NC \V {d} 
= \NC \Hat {\V {g}} - \V {g} \NR
}
%
And we seek to bound \m {\VNm {\V {d}} _2}.
To do so, we argue that \m {\V {g}} can be seen as sparse, and for that purpose we generously bound \m {\d_{S}} by \m {1}.
As shall be seen, it suffices to set
%
\DispNum {g':g':2N:Nh} {
\NC \g
= \NC 2 \R {\log N_H} \NR
}
%
The propositions below are taken from \cite [CaT07] and modified slightly in accordance to our settings, besides the generalization to complex vectors.
%
\Result
{Proposition}
{
%
\DispNum {d:1:dA:C1} {
\NC \VNm {\V {d} _{\SB{\MS {C}}}} _1
\leq \NC \VNm {\V {d} _{\SB{\MS {A}}}} _1
+2\VNm {\V {g} _{\SB{\MS {C}}}} _1 \NR
}
}

To show this, apply triangle inequality respectively on \m {\MS {A}} and \m {\MS {C}}, and recall that \m {\V {g} =\V {g} _{\SB{\MS {A}}}},
%
\DispNum {g:1:gd:g1} {
\NC \VNm {\V {g}} _1
- \VNm {\V {d} _{\SB{\MS {A}}}} _1
+ \VNm {\V {d} _{\SB{\MS {C}}}} _1
- \VNm {\V {g} _{\SB{\MS {C}}}} _1
\leq \NC \VNm {\V {g} + \V {d}} _1 \NR
%
\NC =\NC \VNm {\hat {\V {g}}} _1 \NR
}
%
Also, by construction that \m {\hat {g}} minimizes the \m {\ell_1}-norm,
%
\DispNum {g:1:g1:g1} {
\NC \VNm {\hat {\V {g}}} _1
\leq \NC \VNm {\V {g}} _1 \NR
}
as desired.

%
\Result
{Proposition}
{
If \m {\M {P}} is \m {\d_S}-RIP, then it holds that
%
\DispNum {P:d:4N:Nh} {
\NC \VNm {\M {P}^\Adj \M {P} \V {d}} _\infty
\leq \NC  4 \R {2 \log N_H} \NR
}
%
for probability \m {p}, with
%
\DispNum {1:p:NH:H2} {
\NC 1 -p
\leq \NC N_H^{-2}. \NR
}
}

To show this, we would like to bound \m {\Nm {\IP {\V {z} , \M {P} _{\SB {:, n_h}}}}}, which observes standard normal.
Recall the tail bound for \m {Q} function
%
\DispNum {Q:x:12:22} {
\NC Q\SB {x'}
\leq \NC \F {1} {2} \Ss {e}^{-x'^2 /2} \NR
}
%
Particularly, fix threshold \m {x' =2 \R {\log N_H}}, yielding
%
\DispNum {Q:h:1N:Nh} {
\NC Q\SB {2 \R {\log N_H}}
=\NC \F {1} {2 N_H^2}. \NR
}
%
Now, by definition
%
\DispNum {z:y:Pg:ny} {
\NC \IP {\V {z} - \RB {\V {y} - \M {P} \hat {\V {g}}}, \M {P} _{\SB {:,n_h}}}
= \NC \IP {\M {P} \hat {\V {g}} - \M {P} \V {g}, \M {P} _{\SB {:,n_h}}} \NR
%
\NC = \NC \IP {\M {P} \V {d}, \M {P} _{\SB {:,n_h}}} \NR
}
%
By construction
%
\DispNum {P:g:Py:Nh} {
\NC \VNm {\M {P} _{\SB {:,n_h}}^\Adj \RB {\V {y} - \M {P} \hat {\V {g}}}} _\infty
\leq \NC \VNm {\M {P}^\Adj \RB {\V {y} - \M {P} \hat {\V {g}}}} _\infty \NR
%
\NC \leq \NC 2\R {2 \log N_H} \NR
}
%
By triangle inequality, together with \Rf {Q:h:1N:Nh} and \Rf {P:g:Py:Nh}, we obtain the desired result.
%
\DispNum {P:d:zP:Nh} {
\NC \VNm {\M {P} _{\SB {:,n_h}}^\Adj \M {P} \V {d}} _\infty
\leq \NC \VNm {\IP {\V {z}, \M {P} _{\SB {:,n_h}}}} _\infty
+ \IP {\V {y} - \M {P} \hat {\V {g}}, \M {P} _{\SB {:,n_h}}} \NR
%
\NC \simeq \NC 2 \R {2 \log N_H} +2 \R {2 \log N_H} \NR
}

Lastly, the following two propositions are taken from \cite [CaT07], Lemma 1.
The generalization from real vector spaces to complex spaces is trivial.
One of the quoted propositions reads
%
\DispNum {d:2:dA:12} {
\NC \VNm {\V {d}} _2^2
\leq \NC \VNm {\V {d} _{\SB {\MS {AB}}}} _2^2 + \F {1} {S} \VNm {\V {d} _{\SB{\MS {C}}}} _1^2 \NR
}
%
By Jensen inequality,
%
\Result
{Proposition}
{
%
\DispNum {d:2:dA:C1} {
\NC \VNm {\V {d}} _2
\leq \NC \VNm {\V {d} _{\SB {\MS {AB}}}} _2 + \F {1} {\R {S}} \VNm {\V {d} _{\SB{\MS {C}}}} _1 \NR
}
}

The other quoted proposition is
%
\DispNum {d:2:11:C1} {
\NC \VNm {\V {d} _{\SB{\MS {AB}}}} _2
\leq \NC \F {1} {1- \d_{2S}} \VNm {P _{\SB {\MS {AB}}}^\Tr P d} _2
+ \F {\T {\d}_{S, 2S}} {\RB {1- \d_{2S}} \R {S}} \VNm {d_{\SB{\MS {C}}}} _1 \NR
}
%
Here \m {\T {\d} _{s_1, s_2}} denotes the \m {s_1,s_2}-restricted isometry constant, and is defined in \cite [Can05] (which uses \m {\th}), as follows.
Let \m {\M {Q}'} be fixed.
For all \m {s_1}-sparse \m {\V {x}_1}, for all \m {s_2}-sparse \m {\V {x}_2}, whose nonzero-components are disjoint,
%
\DispNum {d:2:in:x2} {
\NC \T {\d} _{s_1, s_2}
= \NC \inf \CB {
\d':\; \Nm {\IP {\M {Q}' \V {x}_1, \M {Q}' \V {x}_2}}
   \leq \d' \VNm {\V {x}_1} \VNm {\V {x}_2}
} \NR
}
%
\m {\T {\d} _{s_1, s_2}} can be related to \m {\d_s} by \cite [Can05]
\DispNum {d:2:ds:s2} {
\NC \T {\d} _{s_1, s_2}
\leq \NC \d_{s_1+s_2} \NR
}
And it remains to relate \m {\d_{2S}} to \m {\d_{S}}.
By triangle inequality, by Jensen inequality,
\DispNum {Q:x:Qx:x2} {
\NC \VNm {Q' \RB {\V {x}_1 + \V {x}_2}}
%
\leq \NC \VNm {Q' \V {x}_1}
+ \VNm {Q' \V {x}_2} \NR
%
\NC \leq \NC \RB {1 + \d _{s'}} \RB {\VNm {\V {x}_1} + \VNm {\V {x}_2}} \NR
%
\NC \leq \NC \RB {1 + \d _{s'}} \D \R{2} \VNm {\V {x}_1 + \V {x}_2} \NR
}
Therefore
\DispNum {d:s:2s:2s} {
\NC \d_{2s'}
\leq \NC \R {2} \d_{s'} \NR
}
By the same token,
%
\Result
{Proposition}
{
If \m {\M {Q}'} has \m {\d_{s'}}-RIP, then, for \m {M=1,2,3,\dots} so that \m {\d _{Ms'}} is meaningful,
%
\DispNum {d:s:Md':d's} {
\NC \d_{Ms'}
\leq \NC \R {M} \d_{s'} \NR
}
}
%
In particular,
%
\DispNum {d':S:Ll:d's} {
\NC \d_{S}
\leq \NC \R {L} \RB {\log N_H} ^{1/4} \d_{s} \NR
}
%
So \Rf {d:2:11:C1} becomes
%
\Result
{Proposition}
{
If \m {\M {P}} is \m {\d_S}-RIP,
%
\DispNum {d:2:11:C1} {
\NC \VNm {\V {d} _{\SB{\MS {AB}}}} _2
\leq \NC \F {1} {1- \R{2} \d_{S}} \VNm {P _{\SB {\MS {AB}}}^\Tr P d} _2
+ \F {\R{3} \d_{S}} {\RB {1- \R{2} \d_{S}} \R {S}} \VNm {d_{\SB{\MS {C}}}} _1 \NR
}
}
\stopsubsection

\startsubsection [title={Proposed bound}]

Finally, we shall show the proposed bound by combining previous propositions.
For simplicity, set \m {S = Ls^2}, thus
%
\DispNum {S:S:4N:NH} {
\NC S
=\NC L \log N_H \NR
}
%
And, for concreteness,
%
\DispNum {d':S:18:18} {
\NC \d_S
\leq \NC \F {1} {8} \NR
}
%
\Result
{Theorem}
{
Let \m {\V {y}}, \m {\M {P}}, \m {\V {g}}, \m {\hat {\V {g}}}, \m {\V {d}} be defined as above.
Then it holds that
%
\DispNum {d:2:OL:H3} {
\NC \VNm {\V {d}} _2
\leq \NC \MS {O} \RB {\R {L} \R {\log N_H}^3} \NR
}
%
for probability \m {p}, with
%
\DispNum {1:p:ON:13} {
\NC 1 -p
=\NC \MS {O} \SB {N_H ^{-1/3}} \NR
}
}
%
To show this, simplify the expression of \m {\VNm {\V {d} _{\SB{\MS {AB}}}} _2} in \Rf {d:2:11:C1} and \m {\VNm {\V {d} _{\SB{\MS {C}}}} _1} in \Rf {d:1:dA:C1}, and substitute rough numbers.
By the definition of truncation, by \m {\ell_p} norm inequality, by \Rf {P:d:4N:Nh},
%
\DispNum {P:2:PP:NH} {
\NC \VNm {\M {P} _{\SB {\MS {AB}}}^\Tr \M {P} \V {d}} _2
\leq \NC \VNm {\M {P}^\Tr \M {P} \V {d}} _2 \NR
%
\NC \leq \NC \R {S} \VNm {\M {P}^\Tr \M {P} \V {d}} _\infty \NR
%
\NC \leq \NC 5.65 \R {\log N_H} \NR
}
%
By \m {\ell_p}-norm inequality, by the definition of truncation, by \Rf {d:2:11:C1}, and by \Rf {P:2:PP:NH} just above,
%
\DispNum {d:1:Sd:C1} {
\NC \VNm {\V {d} _{\SB{\MS {A}}}} _1
\leq \NC \R {S} \VNm {\V {d} _{\SB{\MS {A}}}} _2 \NR
%
\NC \leq \NC \R {S} \VNm {\V {d} _{\SB{\MS {AB}}}} _2 \NR
%
\NC \leq \NC 1.21 L \R {\log N_H}^3
+0.262 \VNm {\V {d} _{\SB{\MS {C}}}} _1 \NR
}
%
And \Rf {a:2:4N:NH} gives, after arrangement,
%
\DispNum {d:A1:16:H2} {
\NC \VNm {\V {d} _{\SB{\MS {A}}}} _1
%
\leq \NC 1.64 L \R {\log N_H}^3 +0.951 L \RB {\log N_H}^2 \NR
}
%
Then, substituting \Rf {g:C:4p':NH} and \Rf {d:A1:16:H2} above into \Rf {d:1:dA:C1}, we get
\DispNum {d:C1:16:H2} {
\NC \VNm {\V {d} _{\SB{\MS {C}}}} _1
%
\leq \NC 1.64 L \R {\log N_H}^3 +3.61 L \RB {\log N_H}^2 \NR
}
%
And \Rf {d:2:11:C1} becomes
%
\DispNum {d:2:16:H3} {
\NC \VNm {\V {d} _{\SB{\MS {AB}}}} _2
%
\leq \NC 1.64 \R {L} \log N_H +0.951 \R {L} \R {\log N_H}^3 \NR
}
We are now in a position to bound \m {\VNm {d} _2}, again by \Rf {d:2:dA:C1}, and plugging in \Rf {d:2:16:H3} and \Rf {d:C1:16:H2}, giving
\DispNum {d:2:42:NH} {
\NC \VNm {\V {d}} _2
\leq \NC 3.29 \R {L} \log N_H +4.56 \R {L} \R {\log N_H}^3 \NR
}
Therefore, we set
\DispNum {c':c':8L:H3} {
\NC \chi
= \NC 8 \R {L} \R {\log N_H}^3 \NR
}
so that
\DispNum {d:d:c'-:c'-} {
\NC \VNm {\V {d}} _2
\leq \NC \chi \NR
}

We further investigate the four events of failure, as follows.
Consider failure probabilities \m {q _{\Rm {iso}, B}}, \m {q _{\Rm {iso}, R}}, \m {q _{\Rm {no}}}, \m {q _{\Rm {sp}}}, respectively for restricted isometry, noise, sparsity.
If all of them are small, a naive application of union bound suggests
\DispNum {q:q:2q:ar} {
\NC q
\eqsim \NC 2 \RB {q _{\Rm {iso}, B} + q _{\Rm {iso}, R}} +q _{\Rm {no}} +q _{\Rm {sp}} \NR
}
Quoting previous results \Rf {1:p:2e:s4}, \Rf {1:p:ex:d's}, \Rf {1:p:ex:d's}, and \Rf {1:p:NH:H2}, we make some approximation and set
\DispNum {q:B:2e:Lp'} {
\NC q _{\Rm {iso}, B}
\lesssim \NC 2 \Ss {e} ^{-N_B /12} \NR
%
\NC q _{\Rm {iso}, R}
\lesssim \NC 2 N_H ^{-1/3} \RB {\log N_H} ^{-1/6} \NR
%
\NC q _{\Rm {no}}
\lesssim \NC N_H ^{-2} \NR
%
\NC q _{\Rm {sp}}
\lesssim \NC 2 \Ss {e} ^{-9L/\pi} \NR
}
%
If we provide the ideal values
\DispNum {N:B:4l:H2} {
\NC N_B
\gtrsim \NC 4 \log N_H \NR
%
\NC N_R
\gtrsim \NC 16 L \RB {\log N_H}^2 \NR
}
%
It can be seen that
\DispNum {q:p:qn:oR} {
\NC q _{\Rm {sp}}
\lesssim \NC q _{\Rm {no}} \NR
%
\NC \lesssim \NC q _{\Rm {iso}, B} \NR
%
\NC \eqsim \NC q _{\Rm {iso}, R} \NR
}
%
then we get the estimation
\DispNum {q:q:ON:13} {
\NC q
\eqsim \NC 2 N_H ^{-1/3} \NR
}

\stopsubsection
\stopsection


