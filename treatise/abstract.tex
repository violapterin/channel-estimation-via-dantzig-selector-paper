
\starttitle [title={Abstract}]

Multiple-input-multiple-output (MIMO) is the next-generation standard of wireless communication.
However, for MIMO system it is not easy to estimate the channel, while many communication algorithms often requires that the channel state information be known at the receiver.
Meanwhile, the millimeter wave (mm-Wave) band, often used in MIMO context, is known to exhibit sparsity, and results on compressive sensing suggests that sparsity implies that fewer channel measurement shall suffice.
For that reason, Orthogonal Matching Pursuit (OMP) is usually used, as it has a very lower complexity.
However, Dantzig Selector (DS), proposed by Tao and Candès, has long been introduced and ignored.
DS has several promising properties, and it is curious how it fares on the present problem.

In this treatise, we consider a mm-Wave MIMO channel, and a hybrid structure at both the transmitter and receiver end.
Using pilot signal and designed random beamformer, we apply DS for channel estimation, arguing in favor of its greater generality, its tolerance of noise corruption, and its lower averaged error.
The treatise also gives a bound for the expected square error of DS, as well as bounded probability of success, and in so doing, showing dependency on the number of path and channel sparsity.

DS can be cast as a second order cone problem (SOCP), so that well known methods for convex programs may be used.
Simulation is done on DS and other methods, and we discuss their respective merits.
It turns out DS is often slightly better than other alternatives, while it has siginificantly greater complexity.


\stoptitle
