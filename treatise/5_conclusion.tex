
\startchapter [title={Conclusion}]

The treatise addresses the problem of effective estimation of the millimeter wave channel by exploiting its sparsity.
Towards that end, we try the Dantzig Selector (DS) rather than Orthogonal Matching Pursuit (OMP), which is a more common choice, as the estimation algorithm.
We proceed by exploiting the sparsity of spatial frequency domain, rather than 
the spatial domain itself, and suggested the generalization of DS for complex vectors work in present case.
Moreover, assuming the virtual channel model according to uniform linear array response, we managed to prove explicitly, following the spirit of Candès and Tao \cite [CaT07], that for overwhelming probability, the expected square error is bounded.

We also point out that DS may be cast into an LP, and simulation is done.
DS is steadily more accurate than all other methods, while exhibiting rather high complexity.
Unfortunately, and it remains to investigate why there is no obvious trend between the signal level and the relative error norm, and why CVXPY sometimes gives overflowing answers.

One must be careful in giving meaningful comparison of existent analytic bounds for compressive sensing algorithms, as they might work in different settings.
Indeed, DS requires a RIP matrix and performance is guaranteed in noisy observation \cite [CaT07], while RIP is not easily to justify rigorously, and its easy construction is even an open problem.
OMP, on the other hand, may require only an entrywise i.i.d.\ sensing matrix \cite [TrG07a], but much of its property remains open.
On the whole, DS trades off low time and space complexity high precision, and for OMP the opposite is true.

\stopchapter


