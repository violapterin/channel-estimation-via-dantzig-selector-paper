
\startchapter [title={Conclusion}]

The treatise addresses the problem of effective estimation of the millimeter wave channel by exploiting its sparsity.
Towards that end, we try the Dantzig Selector (DS) rather than Orthogonal Matching Pursuit (OMP), which is a more common choice, as the estimation algorithm.
We proceed by exploiting the sparsity of spatial frequency domain, rather than 
the spatial domain itself, and suggested the generalization of DS for complex vectors work in present case.
Moreover, assuming the virtual channel model according to uniform linear array response, we managed to prove explicitly, following the spirit of Candès and Tao \cite [CaT07], that for overwhelming probability, the expected square error is bounded.

We also point out that DS may be cast into an LP, and simulation is done.
DS is steadily more accurate than all other methods, while exhibiting rather high complexity.
Unfortunately, , and it remains to investigate why there is no obvious trend between the signal level and the relative error norm, and a wider range of parameter has to be tried for further conclusions.
We also point out the Python package CVXPY sometimes gives overflown values.

One must be careful in giving meaningful comparison of existent analytic bounds for compressive sensing algorithms, as they might work in different settings.
Indeed, DS requires a RIP matrix and performance is guaranteed in noisy observation \cite [CaT07], while RIP is not easily to justify rigorously, and its easy construction is even an open problem.
OMP, on the other hand, requires only an entrywise i.i.d.\ Normal sensing matrix \cite [TrG07a], but much analysis remains to be done.
OMP trades off precision for a lower time and space complexity, and for DS the other way around, and this eloquently illustrates that different technique suits different problems and constraints, as is true in every discipline of engineering.

\stopchapter


