\startchapter [title={Simulation}]

We are ready to verify bound of eror norm of DS by numerical experiments.
DS is a convex optimization program, and there are many library devoted on that.
For present purpose, we choose the ready-made Python library CVXPY.
For the implementation details, see Diamond and Boyd \cite [DiB16]

However, it turns out direct implementation of Algorithm 8 exhibit extraordinarily large complexity, and we would like to recast the problem in a more tractable form.
Towards that end, we found out that Candès and Romberg \cite [CaR05] has shown that DS can be simplified as an linear program (LP).
However, the situation is more cubersome than that, since they deal with real vectors, but we consider the complex ones.
Fortunately, they have briefly pointed out, complex DS can be recast as an linear program (LP), one of the several standard convex programming problems for which there are already several efficient algorithms.

\startsection [title={Method}]

\startsubsection [title={Some Notation}]

For brevity we shall represent complex vector and matrices by real ones, comprising of its real an imaginary parts.
For \m {\V {x} \in \MB {C} ^{M}}, define the real representation \m {\MS {R} \SB {\V {x}}} of \m {\V {x}} to be
%
\DispNum {R:x:R2:M1} {
\NC \MS {R} \SB {\V {x}}
\in \NC \MB {R} ^{2M} \NR
\NC \MS {R} \SB {\V {x}} _{\SB {m}}
= \NC \startcases
\NC \MF {Re} \SB {\V {x} _{\SB {m'}}}, \MC m =2m' \NR
\NC \MF {Im} \SB {\V {x} _{\SB {m'}}}, \MC m =2m'+1 \NR
\stopcases \NR
\NC m' 
= \NC  0, 1, 2, \ldots, M-1 \NR
}
%
The injectivity is obvious, and we may define \m {\MS {R} ^{-1}} so that
%
\DispNum {R:1:x::x:} {
\NC \MS {R} ^{-1} \SB {\MS {R} \SB {\V {x}}}
=\NC \V {x} \NR
}
%
Accordingly, the following generalization to complex matrices is valid, once we call the ring representation of complex numbers.
For \m {\M {A} \in \MB {C} ^{M_1 \D M_2}}, define real representation \m {\MS {R} \SB {\M {A}}} of \m {\M {A}} to be
%
\DispNum {R:A:R2:21} {
\NC \MS {R} \SB {\M {A}}
\in \NC \MB {R} ^{2M_1 \D 2M_2} \NR
\NC \MS {R} \SB {\M {A}} _{\SB {m_1,m_2}} =
\NC \startcases
\NC \MF {Re} \SB {\M {A} _{\SB {m_1',m_2'}}}, \MC (m_1, m_2) = (2m_1', 2m_2') \NR
\NC \MF {Im} \SB {\M {A} _{\SB {m_1',m_2'}}}, \MC (m_1, m_2) = (2m_1'+1, 2m_2') \NR
\NC -\MF {Im} \SB {\M {A} _{\SB {m_1',m_2'}}}, \MC (m_1, m_2) = (2m_1', 2m_2'+1) \NR
\NC \MF {Re} \SB {\M {A} _{\SB {m_1',m_2'}}}, \MC (m_1, m_2) = (2m_1'+1, 2m_2'+1) \NR
\stopcases \NR
\NC m_1 
= \NC 0, 1, 2, \ldots, M_1 -1 \NR
\NC m_2 
= \NC 0, 1, 2, \ldots, M_2 -1 \NR
}

With these, we define
%
\DispNum {y:::Ry:Ny} {
\NC \T {\V {y}}
= \NC \MS {R} \SB {\V {y}}
\in \MB {R} ^{2 N_Y^2} \NR
%
\NC \T {\V {g}}
= \NC \MS {R} \SB {\V {g}}
\in \MB {R} ^{2 N_H^2} \NR
%
\NC \T {\M {P}}
= \NC \MS {R} \SB {\M {P}}
\in \MB {R} ^{2 N_Y^2 \D 2 N_H^2} \NR
%
\NC \T {\M {P}} ^\Adj
= \NC \MS {R} \SB {\M {P} ^\Adj}
\in \MB {R} ^{2 N_Y^2 \D 2 N_H^2} \NR
%
\NC \T {\V {z}}
= \NC \MS {R} \SB {\V {z}}
\in \MB {R} ^{2 N_Y^2} \NR
}
%
such that, by construction,
%
\DispNum {y:::Pg:gz} {
\NC \V {\T {y}}
= \NC \M {\T {P}} \V {\T {g}} +\V {\T {z}} \NR
}

Furthermore, introduce the following matrices to identify the components where we want to take \m {\ell_2}-norm, in the manner similar to an indicator function.
%
\DispNum {u:h:RN:h1} {
\NC {\V {u}}_{n_h}
\in \NC \MB {R} ^{N_H^2} \NR
\NC \RB {\V {u}_{n_h}} _{\SB {n_h'}}
= \NC
\startcases
1, \Q \MC n_h' =n_h \NR
0, \Q \NC \Rm {otherwise} \NR
\stopcases \NR
\NC \T {\M {U}}_{n_h} \in \NC \MB {R} ^{2 N_H^2 \D 2 N_H^2} \NR
\NC \RB {\T {\M {U}}_{n_h}} _{\SB {n_h', n_h''}}
= \NC
\startcases
1, \Q \MC n_h' =2n_h,\; n_h'' =2n_h \NR
1, \Q \MC n_h' =2n_h +1,\; n_h'' =2n_h +1 \NR
0, \Q \NC \Rm {otherwise} \NR
\stopcases \NR
\NC n_h, n_h', n_h'' 
= \NC 0, 1, 2, \ldots, N_H^2 -1 \NR
}
%
And denote \m {\V {1}} to be the all-\m{1} vector for short, and \m {\V {0}} the all-\m{0} vector, whose dimension will be specified from context.
Now, if we introduce vector \m{\V {m}}
%
\DispNum {m:::RN:h1} {
\NC \V {m} \in \NC \MB {R} ^{N_H^2} \NR
\NC \V {m} \DB{n_h}
= \NC \Nm{\V {g} \DB{n_h}} \NR
\NC n_h, n_h', n_h'' 
= \NC 0, 1, 2, \ldots, N_H^2 -1 \NR
}
%
to denote the entrywise complex modulus of \m{\V {g}}, we see that the convex optimization now takes the form
%
\DispNum {g:::mi:h1} {
\NC \Hat {\T {\V {g}}}
= \NC \startcases
\NC \Min {\T {\V {g}}', \V {m}} \Q
\MC \IP { \V {1}, \V {m} } \NR
\NC \Rm {subject} \; \Rm {to} \Q
\MC \VNm { \T {\M {U}}_{n_h} \T {\V {g}}' }_2
\leq \IP { \V {u}_{n_h}, \V {m} } \NR
\NC \MC \VNm { \T {\M {U}}_{n_h} \T {\M {P}}^\Adj \RB { \T {\M {P}} \T {\V {g}}' -\T {\V {y}} } }_2
\leq \g_{\Rm {DS}} \NR
\stopcases \NR
\NC n_h 
= \NC 0, 1, 2, \ldots, N_H^2-1 \NR
}

\stopsubsection

\startsubsection [title={A Linear Program}]

We shall see furthermore that the program can be cast into an SOCP.
To make the point clearer, we rewrite the program just above into an extended block matrix form, by defining auxiliary variables as follows,
%
\DispNum {t:t:01:Nh} {
\NC \V {t}
= \NC \startTheMatrix
\NC \V {0} \NR
\NC \V {1} \NR
\stopTheMatrix
\in \MB {R} ^{3 N_H^2} \NR
%
\NC \V {x}'
= \NC \startTheMatrix
\NC \T {\V {g}}' \NR
\NC \V {m} \NR
\stopTheMatrix
\in \MB {R} ^{3 N_H^2} \NR
}
%
with
%
\DispNum {A:i:Ui:21} {
\NC \M {A}_i
= \NC \startTheMatrix
\NC \T {\M {U}}_{i}, \NC \M {0} \NR
\stopTheMatrix
\in \MB {R} ^{2 N_H^2 \D 3 N_H^2} \NR
%
\NC \V {b}_i
= \NC \V {0}
\in \MB {R} ^{2 N_H^2} \NR
%
\NC \V {c}_i
= \NC \startTheMatrix
\NC \V {0} \NR
\NC \V {u}_{i} \NR
\stopTheMatrix
\in \MB {R} ^{3 N_H^2} \NR
%
\NC d_i
= \NC 0 \NR
%
\NC i 
= \NC 0, 1, 2, \ldots, N_H^2 -1 \NR
}
%
and
%
\DispNum {A:i:Ui:21:"} {
\NC \M {A}_i
= \NC \startTheMatrix
\NC -\T {\M {U}}_{i - N_H^2} \T {\M {P}}^\Adj \T {\M {P}}, \NC \M {0} \NR
\stopTheMatrix
\in \MB {R} ^{2 N_H^2 \D 3 N_H^2} \NR
%
\NC \V {b}_i
=\NC \T {\M {U}}_{i - N_H^2} \T {\M {P}}^\Adj \T {\V {y}}
\in \MB {R} ^{2 N_H^2} \NR
%
\NC \V {c}_i
= \NC \V {0}
\in \MB {R} ^{3 N_H^2} \NR
%
\NC d_i
= \NC \g_{\Rm {DS}} \NR
%
\NC i 
= \NC N_H^2, N_H^2 +1, N_H^2 +2, \ldots, 2 N_H^2 -1 \NR
}

\Result
{Algorithm}
{
\startitemize[n]
\item Input \m{\M {P} \in \MB {C} ^{ N_Y^2 \D N_H^2}}, \m{\V {y} \in \MB {C} ^{ N_Y^2}}, \m {\g_{\Rm {DS}} > 0}.
\item Define \m{\V {t}, \V {x}, \M {A}_i, \V {b}_i, \V {c}_i, d_i} according to \Rf {t:t:01:Nh}, \Rf {A:i:Ui:21}, \Rf {A:i:Ui:21:"}.

\item Calculate
%
\DispNum {x:::mi:h1} {
\NC \Hat {\V {x}}
\LA \NC \startcases
\NC \Min {\V {x}' \in \MB {C} ^{N_H^2}}
\MC \IP {\V {t}, \V {x}'} \NR
\NC \Rm {subject} \; \Rm {to}
\Q  \MC \VNm {\M {A} _{i} \V {x}' +\V {b} _{i}} _2
\leq \IP {\V {c}_{i}, \V {x}'} +\V {d}_i \NR
\NC \MC i 
=0, 1, 2, \ldots, 2 N_H^2 -1 \NR
\stopcases \NR
}
\item Extract
%
\DispNum {g:::x0:h1} {
\NC \Hat{\T{\V {g}}}
\LA \NC \Hat{\V {x}} _{\SB {0 : 2 N_H^2-1}} \NR
}
\item Convert 
%
\DispNum {g:::R1:1g} {
\NC \Hat{\V {g}}
\LA \NC \MS {R} ^{-1} \SB {\Hat{\T{\V {g}}}} \NR
}
\item Calculate
%
\DispNum {G:G:ve:1g} {
\NC \Hat {G}
\LA \NC \Rm {vec}^{-1} \SB {\Hat {g}} \NR
}
\item Calculate
%
\DispNum {H:H:KG:GK} {
\NC \Hat {\M {H}}
\LA \NC \M {K} \Hat {\M {G}} \M {K}^\Adj \NR
}
\item Output \m {\Hat {\M {H}}}.
\stopitemize
}

\stopsubsection

\startsubsection [title={A Two-Stage Version}]

We also tried another version of DS, in which DS is applied twice in estimated nonzero-position.
The two-stage method was briefly mentioned \cite [CaT07] but without further justification by numerical experiments.
To apply the method, we first apply DS as before.
Then, we extract largest positions of the estimated vector, and apply DS again.
Futhermore, we extract largest positions of the second estimated vector, and apply LS.
The final vector's indices has to be scrambled back according to the original index.

\Result
{Algorithm}
{
\startitemize[n]
\item Let \m {\g_{\Rm {DS}} \geq 0} be given, and \m {\M {P}}

\item Set
\Disp {
\NC N_0
=\NC 2 N_H^2, \NR
\NC N_2
=\NC \lfloor 4 \log N_H \rfloor, \NR
\NC N_1
=\NC \lfloor \R {N_0 N_2} \rfloor \NR
}

\item Apply DS to \m {\V {y}, \M {P}} to get \m {\Hat {\V {g}}_0}, call the \m {N_1} largest component of \m {\Hat {\V {g}}_0} and to be \m {\V {g}_1}, and corresponding columns of \m {\M {P}} to be \m {\M {P} _1}.

\item Apply DS to \m {\V {y}, \M {P}_1} to get \m {\Hat {\V {g}}_1}, call the \m {N_2} largest component of \m {\Hat {\V {g}}_1} and to be \m {\V {g}_2}, and corresponding columns of \m {\M {P}} to be \m {\M {P} _2}.

\item Apply least square to \m {\V {y}, \M {P} _2} to get \m {\Hat {\V {g}}_2}, which corresponds to \m {\Hat {\V {g}}}.
\stopitemize
}

Unfortunately, the two-stage method does not always appear to be better than the one-stage method, and we did not include the result in the figures.
It might have something to do with appropriately scaling \m {\g_{\Rm {DS}}} with respect to \m {N_H},
and it is not clear what is most reasonable choice of \m {N_1, N_1, N_2}.

\stopsubsection

\startsection [title={Result}]

\startsubsection [title={Settings}]

To get some idea of the order of magnitude of noise-to-signal level, plug in some actual numbers.
Consider only path loss in the simplest form according to Friis Law.
Suppose the power of mobile phone antenna is 0.25 W,
the carrier frequency is 5GHz,
the base station is \m {1.5} km away,
the noise is \m {–40} dB W.
The noise-to-signal ratio would be
\DispNum {1:8:02:24} {
\NC 10^{-4} \F {1} {0.25} \R {\F {5 \D 10^9 \D 4 \pi \D 1500} {3 \D 10^8}}
=\NC 0.224 \NR
}
In the following simulation, we use dimensionless noise-to-signal level \m {\s}.
Its value start with \m {2^{-4}}, and increase by multiplying 2.
For the plot of assorted methods, 8 values of \m {\s} are used; for the plots of only OMP, 9 values are used; for the plots of only OMP, 6 values are used.

We take \m {N_H} to be \m {8, 12, 16, 20, 24}, respectively.
For assorted plots, two series of plots are simulated.
One series for \m {N_Y = \lfloor N_H /2 \rfloor}, giving \m {4, 6, 8, 10, 12};
one series for \m {N_Y = \lfloor N_H /3 \rfloor}, giving \m {3, 4, 5, 7, 8}.
For OMP and DS only plots, we just choose \m {N_Y = \lfloor N_H /3 \rfloor}.
On the other hand, \m {N_R} always set to be \m {\lfloor \RB {\log N_H}^2 \rfloor}.
Unfortunately, this is very far from achieving the design values \Rf {N:B:4l:H2}, and this may be part of the reason the results is not as successful as expected.

Other parameters are fixed in these experiments; 
The number of grid of quantization of analog precoder is \m {16}.
The number of paths \m {L} is \m {4}.
The wavelength of carrier \m {\l _{\Rm {ant}}}, is set to be \m {0.1}.
and the antenna spacing \m {d _{\Rm {ant}} =0.2}.
Only their ratio matters.

For parameters related to the convex optimization algorithm, the tolerance of absolute error in the primal-dual algorithm used in CVXPY is set as \m {10^{-4}}, and the tolerance of relative error, as \m {10^{-3}}.
The maximal number of iteration of CVXPY is set to be \m {32}.
On the other hand, the maximal number of iteration of OMP is set to be \m {4 N_H}.

Denote the threshold used in DS program to be \m {\g_{\Rm {DS}}}, and similar threshold of Lasso to be \m {\g_{\Rm {Lasso}}}.
Various several values are tried for them by first taking the theory value, and multiply or divide it by powers of 2.

In the plots of assorted methods, we set \m {\g_{\Rm {DS}} = 2 \R {\log N_H}} for concreteness, as suggested in \cite [CaT07].
For sake of comparison, \m {\g_{\Rm {Lasso}} = 2 \R {\log N_H}} too.
For OMP, suggested values in Cai \& Wang (2011) is illuminating, where they (in their Theorem 7) considered \m {\ell _2}-norm, and Theorem 8, \m {\ell _\infty}-norm.
We make simplification and set respectively \m {\R {3} N_Y} and \m {2 \R {\log N_H}}.
Meanwhile, the threshold used for the remainder norm used in OMP program is suggested in \cite [CaW11].
We take \m {\h_{\Rm {OMP}} = 2 \R {\log N_H}} for \m {\infty}-norm constraint, \m {\h_{\Rm {OMP}} = \R {3 N_Y}} for 2-norm constraint.

In the plots of DS only or OMP only, we vary the values of \m {\g_{\Rm {DS}}} and \h_{\Rm {OMP}} to observe the effects of thresholds.

Each data point for DS and Lasso is repeated for 256 times, and taken average.
Other methods are repeated for more times: OMP for \m {4 \D 256} times, LS for \m {12 \D 256} times.

For performance metric, we follow Lee, Gil, and Lee \cite [LGL16],
\DispNum {h:h:lo:vg} {
\NC \T {\chi}
=\NC \RB {
   \F {\log_2 {\VNm {\V {h} -\Hat {\V {h}}} _2}}
   {\log_2 {\VNm {\V {h}}_2}}
} _{\Ss {avg}}, \NR
}
However, we note that when \m {\VNm {\V {h}}_2} is small, this can blow up.
It is not clear, either, whether \m {\T {\chi}} is a good indicator of channel capacity, since we do not fix a channel model in our investigation.

The parameters related to precision of the Newton step may be adjusted from CVXPY's class methods.
We set the maximum absolute tolerance to be \m {5 \D 10 ^ {-7}},
the maximum absolute tolerance to be \m {5 \D 10 ^ {-6}},
the maximum feasible tolerance to be \m {5 \D 10 ^ {-7}}.
It remains to be investigated what values are most suitable for our purposes.
If the tolerance parameters are too small, the program often gives overflown output, perhaps because it cannot find feasible solutions.
On the other hand, if they are too big, the steps become imprecise and the program may fail to find correct solutions.

\stopsubsection

\startsubsection [title={Assorted Plots}]

In the following, we plot the reciprocal of noise level vs relative error norm, for \m {N_H = 8, 12, 16, 20, 24}, respectively.
Plots 1 to 5 have \m {N_H = 3 N_Y}, and the plots 6 to 10 has \m {N_H = 2 N_Y}.
%
\blank [big]
%\externalfigure [assorted-wide-very-small-error.png] [wfactor=fit, hfactor=fit]
%
\blank [big]
%\externalfigure [assorted-wide-small-error.png] [wfactor=fit, hfactor=fit]
%
\blank [big]
%\externalfigure [assorted-wide-medium-error.png] [wfactor=fit, hfactor=fit]
%
\blank [big]
\externalfigure [assorted-wide-big-error.png] [wfactor=fit, hfactor=fit]
%
\blank [big]
\externalfigure [assorted-wide-very-big-error.png] [wfactor=fit, hfactor=fit]
%
\blank [big]
%\externalfigure [assorted-narrow-very-small-error.png] [wfactor=fit, hfactor=fit]
%
\blank [big]
\externalfigure [assorted-narrow-small-error.png] [wfactor=fit, hfactor=fit]
%
\blank [big]
%\externalfigure [assorted-narrow-medium-error.png] [wfactor=fit, hfactor=fit]
%
\blank [big]
\externalfigure [assorted-narrow-big-error.png] [wfactor=fit, hfactor=fit]
%
\blank [big]
\externalfigure [assorted-narrow-very-big-error.png] [wfactor=fit, hfactor=fit]

As of the plot of time taken in minutes, we give only one of \m {N_H = 12}, as an example.
%
\blank [big]
\externalfigure [assorted-small-time.png] [wfactor=fit, hfactor=fit]

\stopsubsection

\startsubsection [title={Plots of DS only}]

Again, we plot the reciprocal of noise level vs relative error norm, for \m {N_H = 8, 12, 16, 20, 24}, respectively, and \m {N_H = 3 N_Y} for each of them.
Here, the bound \m {\T {\chi}} is shown.

%
\blank [big]
%\externalfigure [ddss-very-small-error.png] [wfactor=fit, hfactor=fit]
%
\blank [big]
\externalfigure [ddss-small-error.png] [wfactor=fit, hfactor=fit]
%
\blank [big]
\externalfigure [ddss-medium-error.png] [wfactor=fit, hfactor=fit]
%
\blank [big]
\externalfigure [ddss-big-error.png] [wfactor=fit, hfactor=fit]
%
\blank [big]
\externalfigure [ddss-very-big-error.png] [wfactor=fit, hfactor=fit]

\stopsubsection

\startsubsection [title={Plots of OMP only}]

Again, we plot the reciprocal of noise level vs relative error norm, for \m {N_H = 8, 12, 16, 20, 24}, respectively, and \m {N_H = 3 N_Y} for each of them.

%
\blank [big]
%\externalfigure [oommpp-very-small-error.png] [wfactor=fit, hfactor=fit]
%
\blank [big]
%\externalfigure [oommpp-small-error.png] [wfactor=fit, hfactor=fit]
%
\blank [big]
\externalfigure [oommpp-medium-error.png] [wfactor=fit, hfactor=fit]
%
\blank [big]
\externalfigure [oommpp-big-error.png] [wfactor=fit, hfactor=fit]
%
\blank [big]
\externalfigure [oommpp-very-big-error.png] [wfactor=fit, hfactor=fit]

\stopsubsection

\startsubsection [title={Discussion}]

In the assorted plot, DS steadily outperforms other methods, and LS is worse than DS, and Lasso still worse, and OMP still worse.
However, it is not clear how higher \m {\s} corresponds to lower error.
It seems when \m {\s} is further increased near the leftmost of the figures, the error might also increase rapidly, but in that case, many methods give overflown values.

Varying thresholds of DS (namely \m {\g_{\Rm {DS}}}) and OMP (namely \m {\h_{\Rm {OMP}}}) does not have significant effect on the output, and from the cases simulated, there can be no conclusive answer.
Nevertheless, it appears \m {\g_{\Rm {DS}}} is usually best chosen in the order of magnitude of \m {\s}, which is intuitive.
Sometimes the iteration stablizes much earlier than is constrained by maximum iteration number.

Our \m {\T {\chi}}, unfortunately, does not agree the simulated error of DS, and the matter remains to be investigated in the future.
As pointed above, the simulated error is very constant around 1, so \m {\T {\chi}} is an overestimation when \m {\s} is high and underestimation when \m {\s} is low.
A possible explanation is that the non-sparsity of \m {\V {g}} undermines the analysis in chapter 3 (despite our effort to estimate), and thus the validity of performance guarantee of DS becomes more dubious.

\blank [big]
\externalfigure [scatter-ddss-failure.png] [wfactor=300]

To give some idea when DS works or not, the figure above illustrates a case where DS failed.
We see that the soft thresholding policy does rule out those entries which are too small and gives a reasonable guess of the positions of non-zero entries.
Subsequent application of DS on the wrong index set fails, as a result.

\blank [big]
\externalfigure [scatter-ddss-success.png] [wfactor=300]

The figure above illustrates a case where DS succeeded.
The true values of \m {\M {H}} entries are ordered with respect to absolute value, and corresponding estimated value in the first and second stage, are plotted agains the true values.
We see that the soft thresholding policy does rule out those entries which are too small and gives a reasonable guess of the positions of non-zero entries.
The subsequent application of DS on this smaller index set refines the estimated values.

\stopsection

\startsection [title={Complexity}]

\startsubsection [title={Asymtotic Analysis}]

We discuss the complexity of DS and OMP.

For DS, we note that, in general, complexity analysis of convex programs is difficult, in that it is not easily ensured how many iteration the program uses.

There is an analysis (Boyd and Vandenberghe \cite [BoV04]) of inequality constrained Newton method, in which they assumed self-concordance (p.496ff, also p.531).
It is also pointed out (Candès and Romberg \cite [CaR05]) that the complex DS can be cast into an Second Order Cone Programming (SOCP), and it is clear that self concordance applies to SOCP.

However, it is not clear whether these results hold for methods faster than Newton method, and of course they are not necessarily utilized the same way in CVXPY.
Specifically, the implementers of CVXPY, Diamond and Boyd \cite [DiB16], alleges that the program is usually converted in to primal-dual problem, for which and there are several different approaches.
For our purpose, we only discuss the complexity analysis for Newton method.

The function being minimized is \m {\VNm {\V {g}}_1}.
Let \m {\V {g} _0} denote the starting value of \m {\V {g}}, and \m {\V {g} ^{\star}} the the point of convergence.
Then (\cite [BoV04] p.505, Eqn.\ 9.56) gives a bound of the number of iteration of Newton method.
If we just take that as the complexity bound for DS, it becomes
%
\DispNum {C:S:C0:1e'} {
\NC C_{\Rm {DS}}
= \NC C_0 \VNm {\V {g}_0 -\V {g} ^{\star}}_1
+ \log_2 \log_2 \F {1} {\e} \NR
}
%
where \m {C_0} is constant related to implementation of Newton method, and \m {\e} the tolerance of error.
According to the analysis of \cite [BoV04] (eq.9.57), for most cases \m {-\log_2 \log_2 \e} can be bounded by \m {6}.
Also, \m {C_0} usually assumes the value of about several hundred;
They gives an example \m {C_0 =375} (with their notation, \m {C_0 =\dfrac {20 -8\a} {\a \b \RB {1 -2\a}^2}}, where \m {\a = 0.1, \b = 0.8, \e =0.01}, and \m {\a, \b} are parameters used in the Backtrack Line Tracing algorithm).

Every step involves a inversion of \m {\MS {O} \SB {\M {P} ^\Adj \M {P}}}, which is of the same order of magnitude to the time needed for matrix multiplication.
Since the dimension of \m {\M {P} ^\Adj \M {P}} is \m {N_H^2}, the cost of inversion is \m {\MS {O} \SB {N_H^4}}.


We now take some license to approximate.
Assume that we start the iteration by specifying the Least Square estimator as the initial value of \m {\V {g}_0}, namely
%
\DispNum {g:0:gL:Py} {
\NC \V {g}_0
=\NC \V {g}_{\Rm {LS}} \NR
\NC = \NC \RB {\M {P} ^\Adj \M {P}} ^{-1} \M {P} ^\Adj \V {y} \NR
}
%
By above, we may write the complexity of DS to be
%
\DispNum {C:S:Og:g1} {
\NC C_{\Rm {DS}}
=\NC \MS {O} \SB {\VNm {\V {g}_{\Rm {LS}} -\V {g} ^\star} _1} N_H^4 \NR
}
%
Also assume
%
\DispNum {g:g:g;:;g} {
\NC \V {g} ^\star
\approx \NC \V {g} \NR
}
%
Indeed, this is the main objective of our investigation!
And by restricted isometry, \m {\M {P}} has unity-normed, almost orthogonal columns, so
%
\DispNum {P:P:IN:H2} {
\NC \M {P} ^\Adj \M {P}
\approx \NC I _{N_{H}^2} \NR
}
%
Thus, by \Rf {g:0:gL:Py}, \Rf {C:S:Og:g1}, and \Rf {g:g:g;:;g},
%
\DispNum {C:S:Og:z1} {
\NC C_{\Rm {DS}}
=\NC \MS {O} \SB {
   \VNm {\V {g} +\RB {\M {P} ^\Adj \M {P}} ^{-1} \M {P} ^\Adj \V {z}
   -\V {g}} _1} N_H^4 \NR
\NC =\NC \MS {O} \SB {\VNm {\M {P} ^\Adj \V {z}} _1} N_H^4 \NR
}

If we agree that, for high probability,
%
\DispNum {O:j:sO:O1} {
\NC \MS {O} \SB {\VNm {\M {P} ^\Adj \V {z}} _{\SB {j}}}
= \NC \s \MS {O} \RB {1} \NR
}

It is easy to see that \m {\RB {\M {P} ^\Adj \V {z}} _{\SB {j}}} observes Complex Standard Normal distribution.
If the quantity can be estimated by \m {\s}, we have
%
\DispNum {C:S:ON:H6} {
\NC C_{\Rm {DS}}
=\NC \MS {O} \SB {N_{H}^6} \NR
}

On the other hand, Tropp and Gilbert \cite [TrG07a] discussed the complexity of OMP, and the proof of technical lemmata in it can be found in the companion paper \cite [TrG07b].
They give
%
\DispNum {C:P:ON:NY} {
\NC C_{\Rm {OMP}}
=\NC \MS {O} \SB {N_H^2 \log N_Y} \NR
}
%
And it would appear that this is greater than \m {C_{\Rm {DS}}}.
However, \m {\log N_Y} is really small in our experiments.
and we dropped some constants, too, in analyzing \m {C_{\Rm {DS}}}.
If we drop \m {\log N_Y}, then we get \m {\MS {O} \SB {N_H^2}} again.

Lastly, since LS involves the same target function as DS, the same argument is valid, giving \m {C_{\Rm {Lasso}} =\MS {O} \SB {N_H^6}}.

\stopsubsection

\startsubsection [title={Runtime Statistics}]

Friedlander and Saunders \cite [FrS07] criticizes that DS often gives a higher computational cost, compared to Lasso.
Agreeing their observation, simulation shows that DS indeed exhibits high complexity when (in our case) \m {N_H} grows.
It is not suprising from the fact that the convex program requires a sizable memory when finding the Newton step, where our program has to calculate (in our case) \m {\M {P}}, a matrix of dimension \m {\M {N}_H^2}.
However we note that different tolerance of error and value of maximal iteration number may result vastly different computation time.

Meanwhile, we see the complexity of OMP is several order of magnitude lower than that of DS.
This is reasonable, given that it is a greedy alrogithm.
In fact, the time OMP takes is negligibly small.

Lasso usually takes time \m {1/2} to \m {1/4} of that of DS.

\stopsubsection

\stopsection

\stopchapter

