\section {Generating beamforming matrices}

\subsection {The plan}

We want to set the four beamforming matrices to be i.i.d.\ random matrix, hoping that the resulting \m {\M {P}} has RIP.
Let us simply set each entry of \m {\M {F} _B} to be i.i.d.\ normal random variables with mean \m {0}, standard deviation \m {1/2}, multiplied by a normalizing constant \m {\l_B >0}.
We know that the magnitude \m {\RB {\M {F} _B} _{\SB {n_R, n_B}}} follows Rayleigh distribution.
\m {\RB {\M {W} _B ^\Tr} _{\SB {n_B, n_R}}} is set similarly.
In addition, let \m {\RB {\M {F} _R} _{\SB {n_H, n_R}}} be uniformly distributed on the unit circle on the complex plane, multiplied by a normalizing constant \m {\l_R >0}.
\m {\RB {\M {W} _R ^\Tr} _{\SB {n_R, n_H}}} is set similarly.

It turns out we have results that guarantee that both \m {\M {F} _B ^\Tr} and \m {\M {F} _R ^\Tr} have RIP, and the same can be said of \m {\M {W} _B} and \m {\M {W} _R}.
Consider a sparse \m {\V {u}} with unity norm, then we are sure that \m {\M {F} _R ^\intercal \V {u}} has almost unity norm.
If, furthermore, \m {\M {F} _R ^\intercal \V {u}} is sparse, we see that \m {\M {F} _B ^\intercal \M {F} _R ^\intercal \V {u}} has almost unity norm, and the RIP of \m {\M{P}} also follows.

However, we cannot be sure that \m {\M {F} _B ^\intercal \M {F} _R ^\Tr}, the product of two RIP matrices, has RIP, nor do we know that \m {\M {P}}, the result of Kronecker product, has RIP.
In fact, the statement seems to be false in general.
What now?
Can we argue instead that \m {\M {F} _B ^\intercal \M {F} _R ^\intercal \V {u}} has similar norm with \m {\M {F} _R ^\intercal \V {u}}?

\subsection {Operator norm of digital beamforming matrices}

Use \m {\chi _{i,j}} to denote independent complex standard normal random variables.
It follows that (respectively for \m {\M {W} _B})
%
\DispNum {C':C':WB:j2} {
\VNm {\M {F} _B \V {u}} _2
=\R {
   \sum _{i=0} ^{N_R-1} \VNm {\sum _{j=0} ^{N_B-1} \frac {1} {\R{N_B}} \chi _{i,j} \V {u} _{j}} _2 ^2} 
}
%
We need a large deviation result for i.i.d.\ sum of chi-square random variables.
To do so, we apply Lemma \m {1} in Laurent and Massart \cite {LaM00} for equal weights, with \m {x > 0} below.
By some calculation, we have the upper tail inequality (respectively for \m {\M {W} _B})
%
\DispNum {P:x:ex:ex} {
\mathbb {P} \SB {\VNm {\M {F} _B \V {u}} _2 ^2 \geq 1 + 2 \R {\frac {x} {N_R}} + \frac {2x} {N_R}}
\leq \mathsf {e} ^{-x} 
}
%
and the lower tail inequality
%
\DispNum {P:x:ex:ex!} {
\mathbb {P} \SB {\VNm {\M {F} _B \V {u}} _2 ^2 \leq 1 - 2 \R {\frac {x} {N_R}}}
\leq \mathsf {e} ^{-x} 
}
%
With substitution \m {\d_s =16x^2 /N_R^2}, the upper tail can be simplified with \m {x <N_R}, and we get the following bound for overall failure probability.

\Result
{Proposition}
{
For \m {0 <\d_s <1}, it holds that
%
\DispNum {W:1:ds:ds} {
\Nm {\VNm {\M {F} _B \V {u}} _2 ^2 - 1}
\geq \d_s 
}
for probability \m {p}, with
\DispNum {1:p:2e:s4} {
1 -p
\leq 2 \mathsf {e} ^{-N_R \R {\d_s} /4} 
}
}

\subsection {Operator norm of analog beamforming matrices}

In Haviv and Regev, ``The restricted isometry property of subsampled Fourier matrices'' \cite {KlM17}, Theorem \m {4.5} states a big O bound for the failure probability of a DFT submatrix with \m {\d_s} RIP.
We argue that their argument can be refined to give precise constants, by Chernoff-Hoeffding bounds and their theorem \m {2.2} with (say) \m {C = 1/3}.
Their Theorem \m {3.7} and Theorem \m {4.5} may be changed accordingly, so that the relation below gives a bound for failure probability for \m {\d_s /N_{H,t} s} RIP.
In conclusion, (respectively for \m {\M {W} _R})
%
\DispNum {N:R:sd':NH} {
N_R
\geq \frac {s} {\d_s^2} \RB {\log \frac{1}{\d_s}}^2 \RB {\log \frac{s}{\d_s}}^2 \log N_{H,t}
}
%
Drop \m {\log \SB {1 /\d_s}^2} and the denominator \m {\d_s} in \m {\log \SB {s /\d_s}^2} in \eqref {N:R:sd':NH}, to get the statement below.

\Result
{Proposition}
{
Suppose
%
\DispNum {N:R:sd'NH} {
N_R
\geq \frac {s} {\d_s^2} \RB {\log s}^2 \log N_{H,t}
}
%
Then \m {\M {F}_R} (respectively \m {\M {W} _R}) has \m {\d_s} RIP for probability \m {p}, with
\DispNum {1:p:ex:d's} {
1 -p
\leq \RB {\frac {\d_s} {N_{H,t} s}} ^{1/3} 
}
}



\subsection {Confirming the restricted isometry}

So, how do we relate RIP of \m {\M {F} _R ^\Tr}, \m {\M {F} _B ^\Tr}, \m {\M {W} _R}, and \m {\M {W} _B} to RIP of \m {\M {P}}?
Notice
\DispNum {P:2:FB:WR} {
\VNm {P} _2
=\VNm {\M {F}_B^\Tr \M {F}_R^\Tr} _2 \VNm {\M {W}_B \M {W}_R} _2 
}
Since both \m {\M {F}_B^\Tr \M {F}_B^\Tr} and \m {\M {W}_B \M {W}_R} are \m {\d_s} RIP, it does appear, if we ignore for a moment the sparsity constraint, that \m {\M {P}} has approximately \m {2\d_s} RIP.
However, it is doubtful to assume that their Kronecker product still has RIP.

We argue as follows.
If angles \m {\th_l} (respectively \m {\f_l}) are exactly the multiples of \m {2 \pi / N_{H,t}} (respectively \m {2 \pi / N_{H,r}}), then
\DispNum {G:G:Ud:dU} {
\M {G}
=\M {U} \mathrm {diag} \SB {\V {d}'} \M {U} ^\dagger 
}
for some \m {S}-sparse \m {\V {d}'} and unitary \m {\M {U}}.
Clearly,
\DispNum {G:2:g2:g2} {
\VNm {\M {G}} _2
=\VNm {\mathrm {abs} \SB {\V {g}}} _2 
}
where \m {\mathrm {abs}} denotes entrywise absolute value.
Then if we notice that \m {\M {W} _B \M {W} _R \M {K} _r \M {U}} and \m {\M {U} ^\dagger \M {K}^\dagger _t \M {F} _R^\dagger \M {F} _B^\Adj} has \m {\d_s} RIP too, we see
\DispNum {W:2:WB:d's} {
&\VNm {\M {W} _B \M {W} _R \M {H} \M {F} _R \M {F} _B} _2 \notag \\
%
= &\VNm {
   \M {W} _B \M {W} _R \M {K} _r \M {U} \D
   \mathrm {diag} \SB {\mathrm {abs} \SB {\V {d}'}} \D
   \M {U} ^\dagger \M {K} _t ^\dagger \M {F} _R \M {F} _B} _2 \notag \\
%
= &\VNm {
   \M {W} _B \M {W} _R \M {K} _r \M {U}
   \R {\mathrm {abs} \SB {\V {d}'}} \D
   \R {\mathrm {abs} \SB {\V {d}'}}
   \M {U} ^\dagger \M {K} _t ^\dagger \M {F} _R \M {F} _B} _2 \notag \\
%
= &\VNm {
   \M {W} _B \M {W} _R \M {K} _r \M {U}
   \R {\mathrm {abs} \SB {\V {d}'}}} _2 \D
   \VNm {\M {F} _B ^\dagger \M {F} _R ^\dagger \M {K} _t \M {U}
   \R {\mathrm {abs} \SB {\V {d}'}}} _2 \notag \\
%
\eqsim &\VNm {\M {G}} _2 \RB {1 +2\d_s} 
}
Thus, even we cannot make sure that \m {\M {P}} is \m {\d_s} RIP with respect to all possible \m {\V {g}}'s, at least \m {\M {P}} has RIP with respect to relevant values of \m {\V {g}} according to the form \eqref {H:H:l0:ql}.

