
\chapter {Conclusion}

The treatise aims to answer the problem of effective estimating a MIMO mm-wave channel by exploiting its sparsity.
To do so, we apply the Dantzig Selector (DS), by exploiting the sparsity in the spatial frequency domain, and justified the restricted isometry of the effective beamforming matrix.
We then prove quantitatively that the expected error norm is bounded for overwhelming probability.
We also suggest several ways of reducing the complexity without losing much performance.
Simulation is done, and we see that DS indeed outperforms other methods in most datasets.

Since we have RIP, a whole series of compressive sensing techniques become possible.
But moreover, we corroborate the prediction that DS gives better regularization, and it may extract more information than other methods do when the sampling rate is low.
In particular, when the sample is less abundant and the noise level is higher, sometimes only DS can recover successfully.
Therefore, one may use DS for sake of higher precision, or with limited RF chains and fewer stages of estimation, for example, in the ultra reliable or low latency scenario.

