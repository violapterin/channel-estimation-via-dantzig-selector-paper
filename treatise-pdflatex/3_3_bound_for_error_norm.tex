\section {Bound for expected error norm}

\subsection {Technical propositions in complex case}

The third and last part of our task is to plug in the almost-sparsity found in Section \m {3.1}, and the failure probability found in Section \m {3.2}.
Set for short
%
\DispNum {d:d:gg:gg} {
\V {d} 
= \hat {\V {g}} - \V {g} \NR
}
%
To bound \m {\VNm {\V {d}} _2}, we illustrate that \m {\V {g}} can be seen as sparse, and for that purpose we generously bound \m {\d_{S}} by \m {1}.
We shall see below that it suffices to set
%
\DispNum {g':g':2N:Nh} {
\g
= 2 \R {\log N_H} \NR
}
%
The propositions below are taken from \cite {CaT07} and modified slightly in accordance to our settings, in addition to the generalization to complex vectors.
%
\Result
{Proposition}
{
%
\DispNum {d:1:dA:C1} {
\VNm {\V {d} _{\SB{\mathscr {C}}}} _1
\leq \VNm {\V {d} _{\SB{\mathscr {A}}}} _1
+\VNm {\V {g} _{\SB{\mathscr {C}}}} _1 \NR
}
}

To show this, apply triangle inequality respectively on \m {\mathscr {A}} and \m {\mathscr {C}},
%
\DispNum {g:1:gd:g1} {
\VNm {\V {g}} _1
- \VNm {\V {d} _{\SB{\mathscr {A}}}} _1
+ \VNm {\V {d} _{\SB{\mathscr {C}}}} _1
- \VNm {\V {g} _{\SB{\mathscr {C}}}} _1
\leq \VNm {\V {g} + \V {d}} _1 \NR
%
=\VNm {\hat {\V {g}}} _1 \NR
}
%
Also, by construction that \m {\hat {g}} minimizes the \m {\ell_1}-norm,
%
\DispNum {g:1:g1:g1} {
\VNm {\hat {\V {g}}} _1
\leq \VNm {\V {g}} _1 \NR
}
as desired.

%
\Result
{Proposition}
{
If \m {\M {P}} has \m {\d_S} RIP, then it holds that
%
\DispNum {P:d:4N:Nh} {
\VNm {\M {P}^\dagger \M {P} \V {d}} _\infty
\leq  4 \R {2 \log N_H} \NR
}
%
for probability \m {p}, with
%
\DispNum {1:p:NH:H2} {
1 -p
\leq N_H^{-2}. \NR
}
}

To show this, by definition,
%
\DispNum {z:y:Pg:ny} {
\IP {\V {z} - \RB {\V {y} - \M {P} \hat {\V {g}}}, \M {P} _{\SB {:,n_h}}}
= \IP {\M {P} \hat {\V {g}} - \M {P} \V {g}, \M {P} _{\SB {:,n_h}}} \NR
%
= \IP {\M {P} \V {d}, \M {P} _{\SB {:,n_h}}} \NR
}
%
On one hand, \m {\Nm {\IP {\V {z} , \M {P} _{\SB {:, n_h}}}}} observes standard normal.
Recall the tail bound for \m {Q} function
%
\DispNum {Q:x:12:22} {
Q\SB {x'}
\leq \frac {1} {2} \mathsf {e}^{-x'^2 /2} \NR
}
%
Particularly, fix threshold \m {x' =2 \R {\log N_H}}, yielding
%
\DispNum {Q:h:1N:Nh} {
Q\SB {2 \R {\log N_H}}
=\frac {1} {2 N_H^2}. \NR
}
%
On the other hand, by construction
%
\DispNum {P:g:Py:Nh} {
\VNm {\M {P} _{\SB {:,n_h}}^\dagger \RB {\V {y} - \M {P} \hat {\V {g}}}} _\infty
\leq \VNm {\M {P}^\dagger \RB {\V {y} - \M {P} \hat {\V {g}}}} _\infty \NR
%
\leq 2\R {2 \log N_H} \NR
}

Now, by triangle inequality, together with \Rf {Q:h:1N:Nh} and \Rf {P:g:Py:Nh}, we obtain the desired result.
%
\DispNum {P:d:zP:Nh} {
\VNm {\M {P} _{\SB {:,n_h}}^\dagger \M {P} \V {d}} _\infty
\leq \Nm {\IP {\V {z}, \M {P} _{\SB {:,n_h}}}}
+ \IP {\V {y} - \M {P} \hat {\V {g}}, \M {P} _{\SB {:,n_h}}} \NR
%
\simeq 2 \R {2 \log N_H} +2 \R {2 \log N_H} \NR
}

Lastly, the following two propositions are taken from \cite {CaT07}, Lemma \m {1}.
The generalization from real vector spaces to complex spaces is obvious.
The first quoted propositions reads
%
\DispNum {d:2:dA:12} {
\VNm {\V {d}} _2^2
\leq \VNm {\V {d} _{\SB {\mathscr {AB}}}} _2^2 + \frac {1} {S} \VNm {\V {d} _{\SB{\mathscr {C}}}} _1^2 \NR
}
%
By Jensen inequality,
%
\Result
{Proposition}
{
%
\DispNum {d:2:dA:C1} {
\VNm {\V {d}} _2
\leq \VNm {\V {d} _{\SB {\mathscr {AB}}}} _2 + \frac {1} {\R {S}} \VNm {\V {d} _{\SB{\mathscr {C}}}} _1 \NR
}
}

The second quoted proposition reads
%
\DispNum {d:2:11:C1} {
\VNm {\V {d} _{\SB{\mathscr {AB}}}} _2
\leq \frac {1} {1- \d_{2S}} \VNm {P _{\SB {\mathscr {AB}}}^\intercal P d} _2
+ \frac {\tilde {\d}_{S, 2S}} {\RB {1- \d_{2S}} \R {S}} \VNm {d_{\SB{\mathscr {C}}}} _1 \NR
}
%
Here \m {\tilde {\d} _{s_1, s_2}} denotes the \m {s_1,s_2}-restricted isometry constant, and is defined in \cite {Can05} (which uses \m {\th}), as follows.
Let \m {\M {Q}'} be fixed.
For all \m {s_1}-sparse \m {\V {x}_1}, for all \m {s_2}-sparse \m {\V {x}_2}, whose nonzero-components are disjoint,
%
\DispNum {d:2:in:x2} {
\tilde {\d} _{s_1, s_2}
= \inf \CB {
\d':\; \Nm {\IP {\M {Q}' \V {x}_1, \M {Q}' \V {x}_2}}
   \leq \d' \VNm {\V {x}_1} \VNm {\V {x}_2}
} \NR
}
%
\m {\tilde {\d} _{s_1, s_2}} can be related to \m {\d_s} by \cite {Can05}
\DispNum {d:2:ds:s2} {
\tilde {\d} _{s_1, s_2}
\leq \d_{s_1+s_2} \NR
}

We also make the approximation that
%
\DispNum {d':S:Ll:d's} {
\d_{S}
\leq \R {L} \RB {\log N_H} ^{1/4} \d_{s} \NR
}
%
So \Rf {d:2:11:C1} becomes
%
\Result
{Proposition}
{
If \m {\M {P}} has \m {\d_S} RIP,
%
\DispNum {d:2:11:C1} {
\VNm {\V {d} _{\SB{\mathscr {AB}}}} _2
\leq \frac {1} {1- \R{2} \d_{S}} \VNm {P _{\SB {\mathscr {AB}}}^\intercal P d} _2
+ \frac {\R{3} \d_{S}} {\RB {1- \R{2} \d_{S}} \R {S}} \VNm {d_{\SB{\mathscr {C}}}} _1 \NR
}
}


\subsection {Proposed bound}

Finally, we shall show the proposed bound by combining previous propositions.
For simplicity, set \m {S = Ls^2}, thus
%
\DispNum {S:S:4N:NH} {
S
=L \log N_H \NR
}
%
And, for concreteness,
%
\DispNum {d':S:18:18} {
\d_S
\leq \frac {1} {8} \NR
}
%
\Result
{Theorem}
{
Let \m {\V {y}}, \m {\M {P}}, \m {\V {g}}, \m {\hat {\V {g}}}, \m {\V {d}} be defined as above.
Then it holds that
%
\DispNum {d:2:OL:H3} {
\VNm {\V {d}} _2
\eqsim \mathscr {O} \SB {\R {L} \R {\log N_H}^3} \NR
}
%
for probability \m {p}, with
%
\DispNum {1:p:ON:13} {
1 -p
\eqsim \mathscr {O} \SB {\mathsf {e} ^{-N_B /12}} \NR
}
}

To show this, by the definition of truncation, by \m {\ell_p} norm inequality, by \Rf {P:d:4N:Nh},
%
\DispNum {P:2:PP:NH} {
\VNm {\M {P} _{\SB {\mathscr {AB}}}^\intercal \M {P} \V {d}} _2
\leq \VNm {\M {P}^\intercal \M {P} \V {d}} _2 \NR
%
\leq \R {S} \VNm {\M {P}^\intercal \M {P} \V {d}} _\infty \NR
%
\leq 5.65 \R {\log N_H} \NR
}
%
By \m {\ell_p}-norm inequality, by the definition of truncation, by \Rf {d:2:11:C1}, and by \Rf {P:2:PP:NH} just above,
%
\DispNum {d:1:Sd:C1} {
\VNm {\V {d} _{\SB{\mathscr {A}}}} _1
\leq \R {S} \VNm {\V {d} _{\SB{\mathscr {A}}}} _2 \NR
%
\leq \R {S} \VNm {\V {d} _{\SB{\mathscr {AB}}}} _2 \NR
%
\leq 1.21 L \R {\log N_H}^3
+0.262 \VNm {\V {d} _{\SB{\mathscr {C}}}} _1 \NR
}
%
Then, substituting \Rf {g:C:4p':NH} and \Rf {d:1:Sd:C1} above into \Rf {d:1:dA:C1}, we get
\DispNum {d:C1:16:H2} {
\VNm {\V {d} _{\SB{\mathscr {C}}}} _1
%
\leq 1.64 L \R {\log N_H}^3 +3.61 L \RB {\log N_H}^2 \NR
}

Now \Rf {d:2:11:C1} becomes
%
\DispNum {d:2:16:H3} {
\VNm {\V {d} _{\SB{\mathscr {AB}}}} _2
%
\leq 1.64 \R {L} \log N_H +0.951 \R {L} \R {\log N_H}^3 \NR
}
and we are ready to estimate \m {\VNm {d} _2}, again by \Rf {d:2:dA:C1}, and plugging in \Rf {d:2:16:H3} and \Rf {d:C1:16:H2}, giving
\DispNum {d:2:42:NH} {
\VNm {\V {d}} _2
\leq 3.29 \R {L} \log N_H +4.56 \R {L} \R {\log N_H}^3 \NR
}
Therefore, we set
\DispNum {c':c':8L:H3} {
\chi
= 8 \R {L} \R {\log N_H}^3 \NR
}
so that
\DispNum {d:d:c'-:c'-} {
\VNm {\V {d}} _2
\leq \chi \NR
}

We end this chapter by remarking on the four events of failure, as follows.
Consider failure probabilities \m {q _{\mathrm {iso}, B}}, \m {q _{\mathrm {iso}, R}}, \m {q _{\mathrm {no}}}, \m {q _{\mathrm {sp}}}, respectively for restricted isometry, noise, sparsity.
If all of them are small, a naive application of union bound suggests
\DispNum {q:q:2q:ar} {
q
\eqsim 2 \RB {q _{\mathrm {iso}, B} + q _{\mathrm {iso}, R}} +q _{\mathrm {no}} +q _{\mathrm {sp}} \NR
}
Quoting previous results \Rf {1:p:2e:s4}, \Rf {1:p:ex:d's}, \Rf {1:p:NH:H2}, and \Rf {1:p:1N:H3}, we make some approximation and set
\DispNum {q:B:2e:Lp'} {
q _{\mathrm {iso}, B}
\lesssim 2 \mathsf {e} ^{-N_B /12} \NR
%
q _{\mathrm {iso}, R}
\lesssim 2 N_H ^{-1/3} \RB {\log N_H} ^{-1/6} \NR
%
q _{\mathrm {no}}
\lesssim N_H ^{-2} \NR
%
q _{\mathrm {sp}}
\lesssim 2 \mathsf {e} ^{-9L/\pi} \NR
}
%
If we provide the ideal values
\DispNum {N:B:4l:H2} {
N_B
\gtrsim 4 \log N_H \NR
%
N_R
\gtrsim 16 L \RB {\log N_H}^2 \NR
}
%
It can be seen that
\DispNum {q:p:qn:oR} {
q _{\mathrm {sp}}
\lesssim q _{\mathrm {no}} \NR
%
\lesssim q _{\mathrm {iso}, R} \NR
%
\eqsim q _{\mathrm {iso}, B} \NR
}
%
then we get the estimation
\DispNum {q:q:ON:13} {
q
\eqsim 2 \mathsf {e} ^{-N_B /12} \NR
}





