\chapter {Simulation}

In this chapter we shall compare \Rf {d:2:OL:H3} with numerical experiments.
For the convex optimization, we choose the Python library CVXPY \cite {DiB16}.
While a direct implementation of Algorithm [7] leads to extraordinarily large complexity, Candès and Romberg \cite {CaR05} showed that DS can be cast as a linear program, which we elaborate below.

\section {Method}

\subsection {Some notation}

Let us represent complex vector and matrices by real ones, consisting of its real and imaginary parts.
For \m {\V {x} \in \mathbb {C} ^{M}}, define the real representation \m {\mathcal {R} \SB {\V {x}}} of \m {\V {x}} to be
%
\DispNum {R:x:R2:M1} {
\mathcal {R} \SB {\V {x}}
\in \mathbb {R} ^{2M} \\
\mathcal {R} \SB {\V {x}} _{\SB {m}}
= \startcases
\mathfrak {Re} \SB {\V {x} _{\SB {m'}}}, \MC m =2m' \\
\mathfrak {Im} \SB {\V {x} _{\SB {m'}}}, \MC m =2m'+1 \\
\stopcases \\
m' 
=  0, 1, 2, \ldots, M-1 \\
}
%
The injection is obvious, and we may define \m {\mathcal {R} ^{-1}} so that
%
\DispNum {R:1:x::x:} {
\mathcal {R} ^{-1} \SB {\mathcal {R} \SB {\V {x}}}
=\V {x} \\
}
%
Accordingly, the following generalization to complex matrices is valid, once we call the ring representation of complex numbers.
For \m {\M {A} \in \mathbb {C} ^{M_1 \D M_2}}, define real representation \m {\mathcal {R} \SB {\M {A}}} of \m {\M {A}} to be
%
\DispNum {R:A:R2:21} {
\mathcal {R} \SB {\M {A}}
\in \mathbb {R} ^{2M_1 \D 2M_2} \\
\mathcal {R} \SB {\M {A}} _{\SB {m_1,m_2}} =
\startcases
\mathfrak {Re} \SB {\M {A} _{\SB {m_1',m_2'}}}, \MC \RB {m_1, m_2} = \RB {2m_1', 2m_2'} \\
\mathfrak {Im} \SB {\M {A} _{\SB {m_1',m_2'}}}, \MC \RB {m_1, m_2} = \RB {2m_1'+1, 2m_2'} \\
-\mathfrak {Im} \SB {\M {A} _{\SB {m_1',m_2'}}}, \MC \RB {m_1, m_2} = \RB {2m_1', 2m_2'+1} \\
\mathfrak {Re} \SB {\M {A} _{\SB {m_1',m_2'}}}, \MC \RB {m_1, m_2} = \RB {2m_1'+1, 2m_2'+1} \\
\stopcases \\
m_1 
= 0, 1, 2, \ldots, M_1 -1 \\
m_2 
= 0, 1, 2, \ldots, M_2 -1 \\
}

With these, we define
%

\DispNum {y:y:Ry:Ny} {
\tilde {\V {y}}
= \mathcal {R} \SB {\V {y}}
\in \mathbb {R} ^{2 N_B^2} \\
%
\tilde {\V {g}}
= \mathcal {R} \SB {\V {g}}
\in \mathbb {R} ^{2 N_H^2} \\
%
\tilde {\M {P}}
= \mathcal {R} \SB {\M {P}}
\in \mathbb {R} ^{2 N_B^2 \D 2 N_H^2} \\
%
\tilde {\M {P}} ^\Adj
= \mathcal {R} \SB {\M {P} ^\Adj}
\in \mathbb {R} ^{2 N_B^2 \D 2 N_H^2} \\
%
\tilde {\V {z}}
= \mathcal {R} \SB {\V {z}}
\in \mathbb {R} ^{2 N_B^2} \\
}
%
so, by construction,
%
\DispNum {y:y:Pg:gz} {
\V {\tilde {y}}
= \M {\tilde {P}} \V {\tilde {g}} +\V {\tilde {z}} \\
}
%
And denote \m {\V {1}} to be the all-\m{1} vector, and \m {\V {0}} the all-\m{0} vector, whose dimension will be inferred from context.



\subsection {A Linear Program}

It is now straightforward to see that DS is equivalent to a linear program.

\Result
{Algorithm}
{
\startitemize[n]
\item Input \m{\M {P} \in \mathbb {C} ^{ N_B^2 \D N_H^2}}, \m{\V {y} \in \mathbb {C} ^{ N_B^2}}, \m {\g_{\mathrm {DS}} > 0}.
%
\item Define \m {\tilde {\V {y}}, \tilde {\V {g}}, \tilde {\M {P}}, \tilde {\M {P}} ^\Adj, \tilde {\V {z}}}, according to \Rf {y:y:Ry:Ny}.
%
\item Compute the convex program
\DispNum {g:f:gf:S1} {
\hat {\tilde {\V {g}}}, \hat {\tilde {\V {f}}}
\leftarrow \startcases
   \Min {\tilde {\V {g}}', \tilde {\V {f}}'}  \MC \IP {\V {1}, \tilde {\V {f}}'} \\
   \mathrm {subject} \; \mathrm {to} \MC \tilde {\V {g}}' \preceq \tilde {\V {f}}' \\
   \MC - \tilde {\V {g}}' \preceq \tilde {\V {f}}' \\
   \MC \tilde {\M {P}}^\dagger \tilde {\M {P}} \tilde {\V {g}}' \preceq \tilde {\M {P}}^\dagger \tilde {\V {y}} + \g_{\mathsf {DS}} \V {1} \\
   \MC - \tilde {\M {P}}^\dagger \tilde {\M {P}} \tilde {\V {g}}' \preceq - \tilde {\M {P}}^\dagger \tilde {\V {y}} + \g_{\mathsf {DS}} \V {1} \\
\stopcases \\
}
\item Convert
%
\DispNum {g:::R1:1g} {
\Hat{\V {g}}
\leftarrow \mathcal {R} ^{-1} \SB {\Hat{\T{\V {g}}}} \\
}
\item Calculate
%
\DispNum {G:G:ve:1g} {
\hat {G}
\leftarrow \mathrm {vec}^{-1} \SB {\hat {g}} \\
}
\item Calculate
%
\DispNum {H:H:KG:GK} {
\hat {\M {H}}
\leftarrow \M {K} \hat {\M {G}} \M {K}^\dagger \\
}
\item Output \m {\hat {\M {H}}}.
\stopitemize
}



\subsection {A Two-Stage Version}

We also tried another version of DS, in which DS is applied twice in estimated nonzero components.
The two-stage method was briefly mentioned \cite {CaT07} without further numerical verification.
It goes as follows:
We first apply DS as before;
we extract largest components of the estimated vector, and apply DS again;
we extract largest components of the second estimated vector, and apply Moore–Penrose inverse;
finally, indices set has to be scrambled back corresponding to the original ones.

\Result
{Algorithm}
{
\startitemize[n]
\item Let \m {\g_{\mathrm {DS}} \geq 0} be given, and \m {\M {P}}

\item Set
\Disp {
N_0
=2 N_H^2, \\
N_2
=\lfloor 4 \log N_H \rfloor, \\
N_1
=\lfloor \R {N_0 N_2} \rfloor \\
}

\item Apply DS to \m {\V {y}, \M {P}} to get \m {\hat {\V {g}}_0}, and call the \m {N_1} largest component of \m {\hat {\V {g}}_0} to be \m {\V {g}_1}, and corresponding columns of \m {\M {P}} to be \m {\M {P} _1}.

\item Apply DS to \m {\V {y}, \M {P}_1} to get \m {\hat {\V {g}}_1}, and call the \m {N_2} largest component of \m {\hat {\V {g}}_1} to be \m {\V {g}_2}, and corresponding columns of \m {\M {P}} to be \m {\M {P} _2}.

\item Apply Moore–Penrose inverse to \m {\V {y}, \M {P} _2} to get \m {\hat {\V {g}}_2}, which corresponds to \m {\hat {\V {g}}}.
\stopitemize
}

Unfortunately, the two-stage method is not always more accurate than the one-stage method, so we did not include it in the plots.
It might have something to do with appropriately scaling \m {\g_{\mathrm {DS}}} with respect to \m {N_H}, and we do not know what the best choice is.



\section {Result}

\subsection {Settings}

In addition to DS, we shall simulate OMP for three different stop conditions, Lasso, and Moore Penrose pseudoinverse (marked as LS which stands for least square).

To get some idea on the order of magnitude of noise-to-signal level, plug in some actual numbers.
Consider only path loss in the simplest form according to Friis Law.
Suppose the power of mobile phone antenna is \m {0.25} W,
the carrier frequency is \m {5} GHz,
the base station is \m {1.5} km away,
the noise is \m {–40} dB W.
If so, the noise-to-signal ratio is
\DispNum {1:8:02:24} {
10^{-4} \frac {1} {0.25} \R {\frac {5 \D 10^9 \D 4 \pi \D 1500} {3 \D 10^8}}
=0.224 \\
}

We use dimensionless noise level \m {\s} which takes value starting with \m {2^{-4}}, and being multiplied by powers of \m {2}.
For the plots of assorted methods, \m {7} values of \m {\s} are considered; for the plots of DS or OMP only, \m {9} values are considered.

We take \m {N_H} to be \m {12, 18, 24}, respectively.
For assorted plots, two series of plots are simulated.
The first series for \m {N_B = N_H}, giving \m {12, 18, 24}.
The second series for \m {N_B = \lfloor 5 N_H /6 \rfloor}, giving \m {10, 15, 20}.
The third series for \m {N_B = \lfloor 2 N_H /3 \rfloor}, giving \m {8, 12, 16}.
For plots of OMP only, we consider \m {N_B = \lfloor N_H \rfloor}.
For plots of DS only, we consider \m {N_B = \lfloor 5 N_H /6 \rfloor}.
On the other hand, \m {N_R} always set to be \m {\lfloor \RB {\log N_H}^2 \rfloor}.
Unfortunately, this is very far from achieving the ideal values \Rf {N:B:4l:H2}, and this may be part of the reason the result is not as successful as expected.

Other parameters are fixed in these experiments.
The number of grid of quantization of analog beamformers is \m {16}.
The number of paths \m {L} is \m {4}.
The wavelength of carrier \m {\l _{\mathrm {ant}}} is set to be \m {0.1}, and the antenna spacing \m {d _{\mathrm {ant}}} is \m {0.2}; only their ratio matters.


Denote the threshold for DS to be \m {\g_{\mathrm {DS}}}, and similar threshold of Lasso to be \m {\g_{\mathrm {Lasso}}}.
In the plots of assorted methods, we set \m {\g_{\mathrm {DS}} = 2 \R {\log N_H}} as suggested in \cite {CaT07}.
For sake of comparison, \m {\g_{\mathrm {Lasso}} = 2 \R {\log N_H}} too.
For OMP, suggested values in Cai and Wang \cite {CaW11} is illuminating, where they considered \m {\ell _2}-norm in their Theorem 7, and \m {\ell _\infty}-norm in their Theorem 8.
We simply take \m {\h_{\mathrm {OMP}} = 2 \R {\log N_H}} for \m {\infty}-norm constraint, \m {\h_{\mathrm {OMP}} = \R {3 N_B}} for 2-norm constraint.
Particularly, in the plots of DS only or OMP only, we vary the values of \m {\g_{\mathrm {DS}}} and \m {\h_{\mathrm {OMP}}} by powers of \m {2}, to observe the effects of thresholds.
The maximal number of iteration of CVXPY is set to be \m {32}, and that of OMP is set to be \m {4 N_B}.

Each data point for DS and Lasso is repeated for \m {256} times, and taken arithmetic average.
Other methods are repeated for more times: OMP for \m {4 \D 256} times, LS for \m {12 \D 256} times.

For performance metric, we follow Lee, Gil, and Lee \cite {LGL16},
\DispNum {h:h:lo:vg} {
\tilde {\chi}
=\RB {
   \frac {\log_2 {\VNm {\V {h} -\hat {\V {h}}} _2}}
   {\log_2 {\VNm {\V {h}}_2}}
} _{\mathsf {avg}}, \\
}
However, we note that when \m {\VNm {\V {h}}_2} is small, this can blow up.
We do not know either whether \m {\tilde {\chi}} is a good indicator of channel capacity; it seems to be, but actually we did not consider a definite a channel model in this treatise.

The parameters related to precision of the Newton step may be adjusted from CVXPY's class methods.
We set the maximum absolute tolerance to be \m {5 \D 10 ^ {-7}},
the maximum relative tolerance to be \m {5 \D 10 ^ {-6}},
the maximum feasible tolerance to be \m {5 \D 10 ^ {-7}}.
If the tolerance parameters are set too small, the program often gives overflowing values, perhaps because it was not able to find feasible solutions.
And if they are set too big, the Newton steps get imprecise and the performance is poorer.
More tests are necessary to determine the best choice of parameters.



\subsection {Plots of assorted methods}

In the following, we plot the reciprocal of noise level vs relative error norm in log scale, for different values and ratios of \m {N_H} and \m {N_B}.

\blank [big]
\externalfigure [assorted-square-small-error.png] [wfactor=fit, hfactor=fit]
\caption {Assorted methods, \m {N_B = 12, N_H = 12}, error norm.}
\blank [big]
%
\blank [big]
\externalfigure [assorted-square-medium-error.png] [wfactor=fit, hfactor=fit]
\caption {Assorted methods, \m {N_B = 18, N_H = 18}, error norm.}
\blank [big]
%
\blank [big]
\externalfigure [assorted-square-big-error.png] [wfactor=fit, hfactor=fit]
\caption {Assorted methods, \m {N_B = 24, N_H = 24}, error norm.}
\blank [big]
%
\blank [big]
\externalfigure [assorted-narrow-small-error.png] [wfactor=fit, hfactor=fit]
\caption {Assorted methods, \m {N_B = 10, N_H = 12}, error norm.}
\blank [big]
%
\blank [big]
\externalfigure [assorted-narrow-medium-error.png] [wfactor=fit, hfactor=fit]
\caption {Assorted methods, \m {N_B = 15, N_H = 18}, error norm.}
\blank [big]
%
\blank [big]
\externalfigure [assorted-narrow-big-error.png] [wfactor=fit, hfactor=fit]
\caption {Assorted methods, \m {N_B = 20, N_H = 24}, error norm.}
%
\blank [big]
\externalfigure [assorted-wide-small-error.png] [wfactor=fit, hfactor=fit]
\caption {Assorted methods, \m {N_B = 8, N_H = 12}, error norm.}
%
\blank [big]
\externalfigure [assorted-wide-medium-error.png] [wfactor=fit, hfactor=fit]
\caption {Assorted methods, \m {N_B = 12, N_H = 18}, error norm.}
%
\blank [big]
\externalfigure [assorted-wide-big-error.png] [wfactor=fit, hfactor=fit]
\caption {Assorted methods, \m {N_B = 16, N_H = 24}, error norm.}

As for the plot of time taken in minutes, we give only one example of \m {N_H = 16}.
%
\blank [big]
\externalfigure [assorted-narrow-medium-time.png] [wfactor=fit, hfactor=fit]
\caption {Assorted methods, \m {N_B = 15, N_H = 18}, time.}
\blank [big]




\subsection {Plots of DS only}

Again, we plot the reciprocal of noise level vs relative error norm in log scale, for \m {N_H = 8, 12, 16, 20, 24}, respectively, and \m {N_H = 3 N_B} for each of them.
Here, \m {\tilde {\chi}} is shown too.
%
\blank [big]
\externalfigure [ddss-narrow-small-error.png] [wfactor=fit, hfactor=fit]
\caption {DS, \m {N_B = 10, N_H = 12}, error.}
\blank [big]
%
\blank [big]
\externalfigure [ddss-narrow-medium-error.png] [wfactor=fit, hfactor=fit]
\caption {DS, \m {N_B = 15, N_H = 18}, error.}
\blank [big]
%
\blank [big]
\externalfigure [ddss-narrow-big-error.png] [wfactor=fit, hfactor=fit]
\caption {DS, \m {N_B = 20, N_H = 24}, error.}
\blank [big]
%
As for the plot of time taken in minutes, we give only one example of \m {N_H = 18}.
%
\blank [big]
\externalfigure [ddss-narrow-medium-time.png] [wfactor=fit, hfactor=fit]
\caption {DS, \m {N_B = 15, N_H = 18}, time.}
\blank [big]



\subsection {Plots of OMP only}

Again, we plot the reciprocal of noise level vs relative error norm in log scale, for \m {N_H = 8, 12, 16, 20, 24}, respectively, and \m {N_H = 3 N_B} for each of them.

%
\blank [big]
\externalfigure [oommpp-square-small-error.png] [wfactor=fit, hfactor=fit]
\caption {OMP, \m {N_H = 12, N_B = 12}, error.}
\blank [big]
%
\blank [big]
\externalfigure [oommpp-square-medium-error.png] [wfactor=fit, hfactor=fit]
\caption {OMP, \m {N_H = 18, N_B = 18}, error.}
\blank [big]
%
\blank [big]
\externalfigure [oommpp-square-big-error.png] [wfactor=fit, hfactor=fit]
\caption {OMP, \m {N_H = 24, N_B = 24}, error.}
\blank [big]
%
As for the plot of time taken in minutes, we give only one example of \m {N_H = 18}.
%
\blank [big]
\externalfigure [oommpp-square-medium-time.png] [wfactor=fit, hfactor=fit]
\caption {OMP, \m {N_H = 18, N_B = 18}, time.}
\blank [big]



\subsection {Discussion}

In the assorted plots, DS steadily outperforms other methods.
From lowest to highest error, it usually goes like this: DS, LS, Lasso, OMP.

The performance of OMP appears to be poor.
In most cases, the estimation fails, and we have \m {\tilde {\chi}} close to the unity.
Varying thresholds of OMP, namely \m {\h_{\mathrm {OMP}}}, does not give different result.

A crucial reason might be that Lee, Gil, and Lee \cite {LGL16} proposed a high-complexity nonconvex program for the beamformers, which they gave approximation, and we have here a low-complexity, randomly-generated beamformers.
It may also have something to do with the distribution of angles of departure and arrival;
their distribution lies within a cluster of standard deviation 15 degrees, and in ours, they are uniform on the unit circle.
It remains to see how OMP and DS compare, if done according to their procedure.

The performance of Lasso also fails mostly, and \m {\tilde {\chi}} even increase as \m {\s} decreases, which is strange.

For Moore Penrose pseudoinverse, \m {\tilde {\chi}} decreases steadily as \m {\s} decreases.

For DS itself, \m {\tilde {\chi}} decreases as \m {\s} increases, which is expected, but \m {\tilde {\chi}} does not go down indefinitely, even for very low \m {\s}.
The values of threshold of DS \m {\g_{\mathrm {DS}}} are chosen within the order of magnitude of \m {\s}.
For high signal level, \m {\g_{\mathrm {DS}}} does not affect much, but for low signal level a greater value may be more effective.
It may have been that the iteration stabilizes much earlier than is constrained by maximum iteration number.

Unfortunately, we report that CVXPY sometimes gives overflowing values, some of them as large as \m {10^{11}}.
This probably indicates some typical-looking output may in fact be unreliable.
Therefore we have discarded outputs larger than a given threshold, for instance \m {10^4}, and simply return the answer to be a Moore–Penrose inverse.

We are not sure either why \m {\tilde {\chi}} is an overestimation of \m {\chi} by several orders of magnitude.
A possible explanation is that the non-sparsity of \m {\V {g}} undermines the analysis in chapter 3, despite our attempts to account for that effect.

In figures 21 and 22, we reproduce two cases of success and failure of DS.
In both figures the true values of \m {\M {H}} entries are ordered by magnitude, and corresponding estimated values in the first and second stage are plotted against them.
When DS fails, the soft thresholding did not rule out smaller entries, thus a second application of DS on the wrong index set failed too.
When DS succeeded, the soft thresholding gave a reasonable guess of the nonzero components, thus a second application of DS on this index set refined the estimated values.

\blank [big]
\externalfigure [scatter-ddss-failure.png] [wfactor=300]
\caption {A scatter plot of position versus magnitude for a case in which DS failed.}
\blank [big]

\blank [big]
\externalfigure [scatter-ddss-success.png] [wfactor=300]
\caption {A scatter plot of position versus magnitude for a case in which DS succeeded.}
\blank [big]



\section {Complexity}

\subsection {Asymtotic analysis}

We discuss the complexity of DS and OMP.
In general, complexity of convex programs is difficult to analyze, in that it is not easily to determine how many iterations were done.
Boyd and Vandenberghe \cite {BoV04} analyzed complexity of inequality constrained Newton method, in which self-concordance is assumed (p.496, also p.531).
In our case, Candès and Romberg \cite {CaR05} pointed out that the complex DS can be cast into an second order cone programming (SOCP), and where self concordance holds.
Diamond and Boyd (writers of CVXPY) \cite {DiB16} explained that convex programs are usually converted into primal-dual problems.
The library we called in the simulation is ECOS, where Newton method is used.
Of course, we are not certain there is no method faster than Newton method, and implementation details of CVXPY may change the conclusion too.

We attemp to give a heuristic argument regarding the complexity of Newton steps.
We say the function being minimized is \m {\VNm {\V {g}}_1}.
Let \m {\V {g} _0} denote the starting value of \m {\V {g}}, and \m {\V {g} ^{\star}} the the point of convergence.
Then Boyd and Vandenberghe \cite {BoV04} (p.505, Eqn.9.56) gave a bound of the number of steps.
If we just take that as the complexity bound for DS, it becomes
%
\DispNum {C:S:C0:1e'} {
C_{\mathrm {DS}}
= \RB {C_0 \VNm {\V {g}_0 -\V {g} ^{\star}}_1
+ \log_2 \log_2 \frac {1} {\e}} C_{\mathrm {step}} \\
}
%
Here \m {C_0} is constant related to implementation of Newton method, and \m {\e} the tolerance of error, and \m {C_{\mathrm {step}}} is the complexity of Newton step.
According to their analysis (Eqn.9.57), for most cases \m {-\log_2 \log_2 \e} can be bounded by \m {6}.
Also, \m {C_0} usually assumes the value of about several hundred.
There was an example they gave in which \m {C_0 =375}, where, with their notation,
\m {C_0 =(20 -8\a) / (\a \b \RB {1 -2\a}^2}), where \m {\a = 0.1, \b = 0.8, \e =0.01}, and \m {\a, \b} are parameters used in the Backtrack Line Tracing algorithm).

Every Newton step involves a inversion of \m {\mathcal {O} \SB {\M {P} ^\dagger \M {P}}}, having the same order of magnitude to the time needed for matrix multiplication.
Since the dimension of \m {\M {P} ^\dagger \M {P}} is \m {N_H^2}, the cost of inversion may be roughly considered to be \m {\mathcal {O} \SB {N_H^4}}, as we know from newer results on the complexity of matrix inversion.
That is, we suppose \m {C_{\mathrm {step}} = N_H^4}.

We take some liberty to assume that, at the start the iteration, the Moore–Penrose inverse was taken to be the initial value of \m {\V {g}_0}, namely
%
\DispNum {g:0:gL:Py} {
\V {g}_0
=\V {g}_{\mathrm {LS}} \\
= \RB {\M {P} ^\dagger \M {P}} ^{-1} \M {P} ^\dagger \V {y} \\
}
%
By above, we may write the complexity of DS to be
%
\DispNum {C:S:Og:g1} {
C_{\mathrm {DS}}
=\mathcal {O} \SB {\VNm {\V {g}_{\mathrm {LS}} -\V {g} ^\star} _1} N_H^4 \\
}
%
Also assume
%
\DispNum {g:g:g;:;g} {
\V {g} ^\star
\approx \V {g} \\
}
%
Indeed, this is the main thesis of our investigation.
And by restricted isometry, \m {\M {P}} has unity-normed, almost orthogonal columns, so
%
\DispNum {P:P:IN:H2} {
\M {P} ^\dagger \M {P}
\approx I _{N_{H}^2} \\
}
%
Thus, by \Rf {g:0:gL:Py}, \Rf {C:S:Og:g1}, and \Rf {g:g:g;:;g},
%
\DispNum {C:S:Og:z1} {
C_{\mathrm {DS}}
=\mathcal {O} \SB {
   \VNm {\V {g} +\RB {\M {P} ^\dagger \M {P}} ^{-1} \M {P} ^\dagger \V {z}
   -\V {g}} _1} N_H^4 \\
=\mathcal {O} \SB {\VNm {\M {P} ^\dagger \V {z}} _1} N_H^4 \\
}

We see that \m {\RB {\M {P} ^\dagger \V {z}} _{\SB {j}}} observes complex standard normal distribution.
If we agree that, for high probability,
%
\DispNum {O:j:sO:O1} {
\mathcal {O} \SB {\VNm {\M {P} ^\dagger \V {z}} _{\SB {j}}}
= \s \mathcal {O} \RB {1} \\
}
Then we simply have
%
\DispNum {C:S:ON:H6} {
C_{\mathrm {DS}}
=\mathcal {O} \SB {N_{H}^6} \\
}

On the other hand, Tropp and Gilbert \cite {TrG07a} discussed the complexity of OMP in the companion paper \cite {TrG07b}.
They gave
%
\DispNum {C:P:ON:NY} {
C_{\mathrm {OMP}}
=\mathcal {O} \SB {N_H^2 \log N_B} \\
}
%
Here \m {\log N_B} is small in our experiments, and if we dropped some constants and \m {\log N_B}, then we may assume \m {\mathcal {O} \SB {N_H^2}}.

Lastly, since Lasso involves the same target function as DS, the same argument is valid, giving \m {C_{\mathrm {Lasso}} =\mathcal {O} \SB {N_H^6}}.



\subsection {Runtime statistics}

All plots included in this treatise, in total, took probably one month to run.
Figures 6 and 10 each took about \m {3} days.
The computer has \m {32} GB RAM, and Intel Core i9 processor.
Simulation ran on Linux subsystem 2 of Windows 10.

Use \m {t_{\mathrm {OMP}} \SB {N_H}}, \m {t_{\mathrm {Lasso}} \SB {N_H}}, \m {t_{\mathrm {DS}} \SB {N_H}} to denote their respective time taken.
We reproduce the arithmetic average (including different \m {\s}) of simulated values below, in seconds.
\DispNum {t:2:95:55} {
t_{\mathrm {OMP}} \SB {12} = 0.0119,
t_{\mathrm {OMP}} \SB {16} = 0.0297, \\
t_{\mathrm {OMP}} \SB {20} = 0.0600,
t_{\mathrm {OMP}} \SB {24} = 0.155, \\
%
t_{\mathrm {Lasso}} \SB {12} = 0.0436,
t_{\mathrm {Lasso}} \SB {16} = 0.176, \\
t_{\mathrm {Lasso}} \SB {20} = 0.471,
t_{\mathrm {Lasso}} \SB {24} = 1.26, \\
%
t_{\mathrm {DS}} \SB {12} = 0.357,
t_{\mathrm {DS}} \SB {16} = 2.07, \\
t_{\mathrm {DS}} \SB {20} = 29.8,
t_{\mathrm {DS}} \SB {24} = 91.5, \\
}
Simulation indeed shows that \m {t_{\mathrm {DS}}} grows tremendously with (in our case) \m {N_H}, agreeing the criticism by Friedlander and Saunders \cite {FrS07} on the high complexity of DS.
It is not surprising, regarding the fact that the convex program has to find the Newton step by calculating (in our case) \m {\M {P}}, a matrix of dimension \m {\M {N}_H^2}.
The matrix inversion is probably the bottleneck of complexity.
However, as we noted above, that different tolerance level of and maximal iteration number may result in vastly different computation time.

Meanwhile, \m {t_{\mathrm {Lasso}}} and \m {t_{\mathrm {OMP}}} are often an order of magnitude lower than \m {t_{\mathrm {DS}}}.
In general, \m {t_{\mathrm {DS}} \gg t_{\mathrm {Lasso}} \gg t_{\mathrm {OMP}}}.
It is now wonder that, given that OMP is a greedy algorithm, the time OMP takes is negligibly small.
But it is also notable that, while Lasso has form similar to DS, it takes much lower time.

Assuming the complexity of each of three methods is polynomial time, we calculate a least square fit for the power of complexity, giving the following estimated values, on which we add a tilde.

\DispNum {t:P:NH:83} {
\tilde {t}_{\mathrm {OMP}} \eqsim \mathcal {O} \SB {N_H^{3.6}} \\
\tilde {t}_{\mathrm {Lasso}} \eqsim \mathcal {O} \SB {N_H^{4.8}} \\
\tilde {t}_{\mathrm {DS}} \eqsim \mathcal {O} \SB {N_H^{8.3}} \\
}

However, note that we must be careful in giving meaningful comparison between compressive sensing algorithms, as they might work in different settings.
DS requires a RIP linear transformation, and it works for noisy linear systems \cite {CaT07}.
Here, it is not easy to justify RIP rigorously, and a more feasible construction is even a research problem.
OMP, on the other hand, requires only an entrywise i.i.d.\ sensing matrix \cite {TrG07a}, of which much is yet to be studied.
All in all, we might say that on the whole, DS trades off low time- and space-complexity for high precision, and for OMP the opposite is true.








