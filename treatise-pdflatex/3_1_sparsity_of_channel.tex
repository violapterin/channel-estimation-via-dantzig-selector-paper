\startsection [title={Sparsity of channel}]

\startsubsection [title={Norm of array response}]

Let \m {\hat {\V {g}}} be the Dantzig Selector, and let the sparsity level \m {S} be fixed.
Split \m {\V {g}} into two parts: \m {\V {g} _{\SB{\mathscr {A}}}}, the largest-magnitude \m {s} components of \m {\V {g}}, \m {\V {g} _{\SB{\mathscr {B}}}} the next \m {s} largest-magnitude components of \m {\V {g}}, and \m {\V {g} _{\SB{\mathscr {C}}}} are the components complement to \m {\V {g} _{\SB{\mathscr {A}}}}.
For example, if \m {\V {g} =\IP {-1,3,-4,2,8}}, and \m {S=2}, then \m {\V {g} _{\SB{\mathscr {A}}} =\IP {0,0,-4,0,8}}, \m {\V {g} _{\SB{\mathscr {B}}} =\IP {0,3,0,2,0}}, and \m {\V {g} _{\SB{\mathscr {C}}} =\IP {-1,3,0,2,0}}.
In this section, the subscripts \m {\mathscr {A}, \mathscr {B}, \mathscr {C}} will bear analogous meaning.

We hope that \m {\VNm {\V {g} _{\SB{\mathscr {C}}}} _1} is small.
If so, we shall substitute the quantity into the expected square error of DS, thus generalizing to the almost-sparse case.
Recall that \m {\M {G}} is just the vectorization of \m {\V {g}}, and is the function of \m {\V {a}}, so we seek to establish that \m {\V {a}} is almost sparse.
We define, similarly, \m {\V {a} _{\SB{\mathscr {A}}}} and \m {\V {a} _{\SB{\mathscr {C}}}}, with sparsity level \m {s} different from \m {S}.
If \m {N_H} is large, we may forget for a moment that \m {s} is an integer.

First, we investigate
%
\DispNum {a:2:Ka:22} {
\NC \VNm {\V {a} \SB {\f} _{\SB{\mathscr {C}}}} _2 ^2
=\NC \VNm {\RB {\M {K}^\dagger \V {a} \SB {\f}} _{\SB{\mathscr {C}}}} _2 ^2 \NR
}
%
Let \m {\f} be fixed.
Towards that end, introduce
%
\DispNum {y':H:f'M:p'p'} {
\NC \psi \SB {\f, n_H}
:=\NC \RB {
   \f \; \mathrm {Mod}\; \frac {2\pi} {N_H}
   + \RB {\frac {2 n_H} {N_H} + 1} \pi
} \;
\mathrm {Mod}\; \RB {2\pi}
- \pi \NR
}
%
Note that by construction
%
\DispNum {y':H:p'::p':} {
\NC \Nm {\psi \SB {\f, n_H}}
\leq \NC \pi \NR
}
%
And set for short
%
\DispNum {D:y':nH:Hy'} {
\NC D \SB {\psi'}
:= \NC \sum_{n_H=0}^{N_H-1} \mathsf {e}^{i n_H \psi'} \NR
}
%
And
%
\DispNum {b:f':Ka:af'} {
\NC \V {b} \SB {\f}
=\NC \M {K}^\dagger \V {a} \SB {\f} \NR
}
%
Then observe
%
\DispNum {K:H:1N:nH} {
\NC \V {b} \SB {\f} _{\SB {n_H}}
=\NC \frac {1} {N_H} D \SB {\psi \SB {\f, n_H}} \NR
}

By the nature of alternating series, for \m {- \pi \leq x < \pi}, it holds that
%
\DispNum {x:6:si:nx} {
\NC \Nm {x - \frac {x^3} {6}}
\leq \NC \Nm {\sin x}. \NR
}
Therefore,
%
\DispNum {D:y':si:y'p'} {
\NC \Nm {D \SB {\psi'}}
= \NC \frac {\Nm {\sin \SB {N_H \psi'/2}}} {\Nm {\sin \SB {\psi' /2}}} \NR
%
\NC \leq \NC B \SB {\psi'} \NR
%
\NC := \NC \frac {48} {\Nm {\psi'^2 -24} \Nm {\psi'}}, \NR
%
\NC - \pi \leq \NC \psi' < \pi. \NR
}
%
Combining \Rf {y':H:f'M:p'p'}, \Rf {K:H:1N:nH}, and \Rf {D:y':si:y'p'} we have
%
\DispNum {b:1:1N:C1} {
\NC \VNm {\V {b} \SB {\f} _{\SB{\mathscr {C}}}} _1
\leq \NC \frac {1} {N_H}
\VNm {
\RB {
   \sum_{n_H' =0}^{N_H -1}
      B \SB {\psi \SB {\f, n_H'}}
      \V {u} _{n_H'}
} _{\SB{\mathscr {C}}}
} _1
\NR
}

Note that \m {\Nm {B \SB {\psi'}}} is strictly decreasing in \m {\SB {0,\pi}}.
We are going to bound the rectangles by an integral, where we have to split the cases that \m {N_H} is odd and even.
%
\DispNum {b:1:1N:NH} {
\NC \VNm {\V {b} \SB {\f} _{\SB{\mathscr {C}}}} _1
\leq \NC \frac {1} {N_H} \D \frac {N_H} {2\pi} \D 2 \int_{\pi s/N_H}^{\pi} B \SB {\psi'} ^2 d \psi' \NR
%
\NC = \NC \frac {48} {N_H \pi^3}
\int _{s /N_H} ^1 \frac {1} {\RB {24/\pi^2 -x'^2} ^2 x'^2} dx' \NR
%
\NC =\NC \frac {1} {\pi} \log \frac {N_H^2 /s^2 - \pi^2 /24} {1 - \pi^2 /24} \NR
%
\NC \eqsim \NC \frac {2} {\pi} \RB {\log N_H - \log s} \NR
%
\NC \leq \NC \frac {2} {\pi} \log N_H \NR
}
%
Since the value of \m {s} is at our disposal, we choose
%
\DispNum {s:s:lo:NH} {
\NC s
=\NC \R {\log N_H} \NR
}
%
so that, by \m {\ell_p}-norm inequality, we have the following estimation.

\Result
{Proposition}
{
Let \m {\f } be given, and linear array response in spacial frequency domain \m {\V {b} \SB {\f}} defined in \Rf {b:f':Ka:af'}.
Then, for any random instance of \m {\f},
%
\DispNum {a:2:4N:NH} {
\NC \VNm {\V {b} \SB {\f} _{\SB{\mathscr {C}}}} _1
\leq \NC \frac {2} {\pi} \log N_H \NR
}
}

\stopsubsection

\startsubsection [title={Norm of angular channel response}]

If we further choose that
\DispNum {S:S:Ls:s2} {
\NC S
=\NC L s^2 \NR
}
by triangle inequality, by the property of Frobenius norm,
%
\DispNum {g:1:l0:a'1} {
\NC \VNm {\V {g} _{\SB{\mathscr {C}}}} _1
\leq \NC
\sum _{l=0} ^{L-1} \Nm {\a_l}
\VNm {\V {b} \SB {\f_l} _{\SB{\mathscr {C}}}} _1
\VNm {\V {b} \SB {\th_l} _{\SB{\mathscr {C}}}} _1 \NR
%
\NC \leq \NC
\VNm {\V {b} \SB {\f_l} _{\SB{\mathscr {C}}}} _1
\VNm {\V {b} \SB {\th_l} _{\SB{\mathscr {C}}}} _1
\sum _{l=0} ^{L-1} \Nm {\a_l} \NR
%
\NC \leq \NC
\frac {4} {\pi^2} L \RB {\log N_H}^2 \D
\frac {1} {L} \sum _{l=0} ^{L-1} \Nm {\a_l} \NR
}
%
Then observe that
%
\DispNum {E:1:2p':2p'} {
\NC \mathbb {E} \SB {\frac {1} {L} \sum _{l=0} ^{L-1} \Nm {\a_l}}
=\NC \F{\R {2}} {\R {\pi}} \NR
}
By Hoeffding inequality,
%
\DispNum {P:p':2e:3p'} {
\NC \mathbb {P} \SB {
   \frac {1} {L} \sum _{l=0} ^{L-1} \Nm {\a_l}
   \geq \frac {4 \R{2}} {\R {\pi}}
}
\leq \NC 2 \exp \SB {- \frac {9L} {\pi}} \NR
}
%
Now we have
\DispNum {g:2:12:43} {
\NC \frac {\VNm {\M {g} _{\SB{\mathscr {C}}}} _1} {\RB {\log N_H}^2}
\leq \NC \frac {12 \R{2}} {\pi ^{5/2}}
\leq \frac {4} {3}. \NR
}
%
\Result
{Proposition}
{
Let \m {\V {g} \in \mathbb {C} ^{N_H^2}} be defined as in \Rf {g:g:ve:Nh}.
Then the bound
%
\DispNum {g:C:4p':NH} {
\NC \VNm {\M {g} _{\SB{\mathscr {C}}}} _1
\leq \NC \frac {4} {3} \RB {\log N_H}^2 \NR
}
%
holds for probability \m {p}, with
%
\DispNum {1:p:1N:H3} {
\NC 1 -p
\leq \NC 2 \exp \SB {- \frac {9L} {\pi}} \NR
}
}

\stopsubsection
\stopsection

