
\Frame {Definition: Sparsity}
{
\I \m {\V {x}} is called \m {s}-sparse, if there is \m {\MC {A} \subseteq \MC {T} \SB {N_p}} such that
\Disp { 
\NC \V{x} \DB {\MC {A}}
=\NC \V{x} \NR
}
with
\Disp { 
\NC \# \MC {A} \leq s. \NR
}
}

\page[yes] % % % % % % % % % % % % % % % % % % % % % % % % % %

\Frame {Definition: Restricted Isometry Property}
{
\I For fixed \m {s =0, \dots N_p -1}, we say that \m {\M{\Phi}} satisfies the restricted isometry property (RIP) of sparsity \m {s} with respect to \m {0 \leq \d_s \leq 1}, if, for all \m {\V {x}}, and for all \m {\MC {A} \subset \MC {T} \SB {N_m}} with \m {\# \MC {A} \leq s},
\Disp {
\NC \NC \RB {1-\d_s} \VNm {\V {x} \DB {\MC {A}}}^2 \NR
\NC \leq \NC \VNm {\M{\Phi} \V {x} \DB {\MC {A}}} _2^2 \NR
\NC \leq \NC \RB {1+\d_s} \VNm {\V {x} \DB {\MC {A}}} _2^2 \NR
}

\I That is, \m {\Phi} is ``almost unitary'' up to ``relative error'' \m {\d_s}.
}

\page[yes] % % % % % % % % % % % % % % % % % % % % % % % % % %

\Frame {Proof Strategy}
{
\I Show that \m {\V{g}} is almost sparse

\I Show that \m {\M{G}} is almost sparse

\I Find \m {\MB {E} \SB {\VNm {\M{P} \V{u}} _2 ^2}} and \m {\MB {E} \SB {\VNm {\M{P} \V{u}} _2 ^4}}

\I Bound the prob.\ that \m {\Nm {\VNm {\M{P} \V{u}} _2 ^2 -\VNm {\V{u}} _2 ^2} \geq \e \VNm {\V{u}} _2 ^2}, thus confirming the RIP of \m {\M{P}}

\I Substitute \m {\VNm {\V {g} _{\MC {K}}} _1} into the original Dantzig Selector Proof
}

\page[yes] % % % % % % % % % % % % % % % % % % % % % % % % % %

\Frame {\m {\V{g}} is Almost Sparse}
{
\I Largest \m {S} positions: \m {\V{g} _\MC {A} =\MC {S} \SB {\V{g}, S}}

\I Next largest \m {S} positions: \m {\V{g} _\MC {K} =\MC {C} \SB {\V{g}, S}}

\I Remaining positions: \m {V{g} _\MC {B} =\MC {S} \SB {\V{g} _\MC {K}, S}}

\I Claim:
\Disp {
R
:= \NC \VNm {\MC {C} \SB {\M {K}^\Adj \V {a} \SB {\f}, s}} _2
\leq \m {\F{1}{3 \R{N_H}}} \NR
}
}

\page[yes] % % % % % % % % % % % % % % % % % % % % % % % % % %

\Frame {Proof (1/2)}
{
\Disp {
\NC D \SB {\psi'}
:= \NC \sum_{n_H=0}^{N_H-1} \Ss {e}^{i n_H \psi'},\Q
-\pi \leq \psi' \leq \pi \NR
\NC \RB {\M {K}^\Adj \V {a} \SB {\f}} \DB {n_H}
=\NC \F {1}{N_H} D \SB {\psi \SB {\f, n_H}} \NR
\NC \Nm {D \SB {\psi'}}
= \NC \F {\Nm {\sin \SB {N_H \psi'/2}}}{\Nm {\sin \SB {\psi' /2}}} \NR
\NC \leq \NC B \SB {\psi'} := \F {48}{\Nm {\psi'^2 -24} \Nm {\psi'}} \NR
}

1. Definition 2. Trivial 3. Expansion around \m {\psi' =0}
}

\page[yes] % % % % % % % % % % % % % % % % % % % % % % % % % %

\Frame {Proof (2/2)}
{
\Disp {
\NC R^2
\leq \NC \F {1}{N_H^2} \D \F {N_H}{2\pi} \D 2 \int_{\pi s/N_H}^{\pi} B \SB {\psi'} ^2 d \psi' \NR
\NC = \NC \F {2304} {N_H \pi^6}
\int _{s /N_H} ^1 \F{1} {(24/\pi^2 -x'^2)^2 x'^2} dx' \NR
\NC =\NC \F{1} {2\pi^2 N_H}
\RB {
  -\F {8} {u}
+\F {4\pi^2 u} {24 -\pi^2 u^2}
+\R{6} \pi \tanh^{-1} \SB {\F {\pi u} {2\R{6}}}
}
\Bigg \| _{s/N_H} ^1 \NR
}

1. Approximating sum of largest \m {S} by integral 2. Direct calculation
}

\page[yes] % % % % % % % % % % % % % % % % % % % % % % % % % %

\Frame {\m {\M{G}} is Almost Sparse}
{
\I Claim:
\Disp {
R
:= \NC \VNm {\MC {C} \SB {\M {G}, s^2 L}} _2
\leq \m {\F{1}{3 \R{N_H}}} \NR
}
}

\page[yes] % % % % % % % % % % % % % % % % % % % % % % % % % %

\Frame {Proof}
{
\Disp{
\NC \VNm {\sum _{l=0} ^{L-1} \a_l \V {a} \SB {\f_l} \V {a} \SB {\th_l} ^\Adj } _F
\leq \NC \sum _{l=0} ^{L-1} \Nm {\a_l}\VNm {\V {a} \SB {\f_l} \V {a} \SB {\th_l} ^\Adj } _F \NR
\NC \leq \NC
\sum _{l=0} ^{L-1}
\Nm {\a_l} \VNm {\V {a} \SB {\f_l}} _2
\VNm {\V {a} \SB {\th_l}} _2 \NR
\NC \MB {E} \SB {\VNm {\MC {C} \SB {\sum _{l=0} ^{L-1} \a_l \V {a} \SB {\f_l} \V {a} \SB {\th_l} ^\Adj}} _F}
\leq \NC \F {1} {9 \R{N_H}} \MB {E} \SB {\sum _{l=0} ^{L-1} \Nm {\a_l}}
\leq \F {\R{\pi} L} {3 \R{N_H}} \NR
}

1. Triangle ineq.\ 2. Cauchy ineq.\ 3. Assuming \m {\a_l} to be i.i.d.\ Gaussian
}

\page[yes] % % % % % % % % % % % % % % % % % % % % % % % % % %

\Frame {Design of \m {\M{F}_B}}
{
\I Set each entry of \m {\M{F}_B} to be i.i.d.\ Gaussian r.v.\ with mean 0, standard deviation \m {1/2}, multiplied by \m {\l_B >0}

\I The magnitude \m {\M{F}_B \DB {n_R, n_Y}} follows Rayleigh distribution, having
\Disp {
\NC M_{B,2}
:=\NC \MB{E} \SB {\Nm {\M{F}_B \DB {n_R, n_Y}}^2}
=\l_R^2 \NR
\NC M_{B,4}
:=\NC \MB{E} \SB {\Nm {\M{F}_B \DB {n_R, n_Y}}^4}
=2 \l_R^4 \NR
}

\I We have \m {\MB{E} \SB {\M{F}_B \DB {n_R, n_Y}^2} =0}
}

\page[yes] % % % % % % % % % % % % % % % % % % % % % % % % % %

\Frame {Design of \m {\M{F}_R}}
{
\I Let \m {\M{F}_R \DB {n_H, n_R}} be uniformly distributed on the unit circle on the complex plane, giving
\Disp{
\NC M_{R,2}
:=\NC \MB{E} \SB {\Nm {\M{F}_R \DB {n_H, n_R}}^2}
=\l_R^2 \NR
\NC M_{R,4}
:=\NC \MB{E} \SB {\Nm {\M{F}_R \DB {n_H, n_R}}^4}
=\l_R^4 \NR
}

\I We have \m {\MB{E} \SB {\M{F}_R \DB {n_H, n_R}^2} =0}
}

\page[yes] % % % % % % % % % % % % % % % % % % % % % % % % % %

\Frame {Finding \m {\MB {E} \SB {\VNm {\M{P} \V{u}} _2 ^2}}}
{
}

\page[yes] % % % % % % % % % % % % % % % % % % % % % % % % % %


\Frame {Finding \m {\MB {E} \SB {\VNm {\M{P} \V{u}} _2 ^4}}}
{
}

\page[yes] % % % % % % % % % % % % % % % % % % % % % % % % % %
