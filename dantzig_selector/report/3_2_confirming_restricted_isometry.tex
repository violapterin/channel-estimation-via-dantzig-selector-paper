\startsection [title={Confirming Restricted Isometry of Beamformer}]
\startsubsection [title={Design of Digital Beamformer Entries}]

Our plan is setting the four precoder matrices to be i.i.d.\ random matrix, hoping that the resulting \m {P} has RIP.
Indeed, \m {\d_s \SB {\M{F}_B} \SB {\o_t}} may be found, along similar lines with Achlioptas (2001) and Baraniuk et.\ al.\ (2008).
We prove a concentration inequality, as explained below, and invoke Chebyshev inequality.
But before that, we have to investigate the moments of each entry of \m {\M{F}_B \SB {\o_t}} (resp.\ \m {\M{W}_B \SB {\o_t}}) and that of \m {\M{F}_R \SB {\o_t}} (resp.\ \m {\M{W}_R \SB {\o_t}}).

Let us make it simple and set each entry of \m {\M{F}_B} to be i.i.d.\ Gaussian r.v.\ with mean 0, standard deviation \m {1/2}, multiplied by a normalizing constant \m {\l_B >0}.
They are (with a slight of abuse of meaning of \m {\o_t} and so on)
\Disp{
\NC d \o_t
=\NC \F {1} {\R {\pi} \l} \exp
  \RB {-\F{1}{\l^2} \MF{Re} \SB {\M{F}_B \DB {n_R, n_Y}} ^2}
d \MF{Re} \SB {\M{F}_B \DB {n_R, n_Y}} \NR[+]
\NC d \o_t
=\NC \F {1} {\R {\pi} \l} \exp
  \RB {-\F{1}{\l^2} \MF{Im} \SB {\M{F}_B \DB {n_R, n_Y}} ^2}
d \MF{Im} \SB {\M{F}_B \DB {n_R, n_Y}} \NR[+]
\NC n_R
= \NC 0, 1, 2, \ldots, N_R -1 \NR
\NC n_Y
= \NC 0, 1, 2, \ldots, N_Y -1 \NR
}

We know that the magnitude \m {\M{F}_B \DB {n_R, n_Y}} follows Rayleigh distribution.
Set for shorthand
\Disp {
\NC M_{B,k}
:=\NC \MB{E} \SB {\Nm {\M{F}_B \DB {n_R, n_Y} \SB {\o_t}}^k} \NR
\NC k =\NC 1, 2, 3, \dots \NR
}

It is known that
\Result
{Lemma}
{
\Disp {
\NC M_{B,k}
=\NC \Gamma \SB {\F{k}{2} +1} \NR[+]
}
}

In particular,
\Disp {
\NC M_{B,2}
=\NC \l_B^2 \NR[+]
\NC M_{B,4}
=\NC 2 \l_B^4 \NR[+]
}

\m {\M{W}_B \DB {n_Y, n_R}} is set in an entirely analolous manner.

\stopsubsection

\startsubsection [title={Design of Analog Beamformer Entries}]

Let \m {\M{F}_R \DB {n_H, n_R}} be uniformly distributed on the unit circle on the complex plane, which gives probability density
\Disp{
\NC d \o_t
= \NC \F {1} {\pi} \RB {1 -\RB {\MF{Re} \SB {\M{F}_R \DB {n_H, n_R}}} ^2}^{-1/2}
d \MF{Re} \SB {\M{F}_R \DB {n_H, n_R}} \NR[+]
\NC d \o_t
= \NC \F {1} {\pi} \RB {1 -\RB {\MF{Im} \SB {\M{F}_R \DB {n_H, n_R}}} ^2}^{-1/2}
d \MF{Im} \SB {\M{F}_R \DB {n_H, n_R}} \NR[+]
}
Set
\Disp {
\NC M_{R,k}
:=\NC \MB{E} \SB {\Nm {\M{F}_R \DB {n_H, n_R} \SB {\o_t}}^k} \NR
\NC k =\NC 1, 2, 3, \dots \NR
}
Then it is trivial that
\Disp {
\NC M_{R,k}
=\NC \l_R^k \NR[+]
}
In fact, it seems to be possible, from analysis as follows, to require that \m {\M{F}_R \DB {n_H, n_R}} be uniformly distributed only on the discrete set of \m {p}-th roots of unity if \m {p} is big enough.
The point may be exploited in practical application when it is decided to avoid the overhead of generation of random phase.
But we choose such continuous distribution to keep the analysis simple.

\m {\M{W}_B \DB {n_Y, n_R}} is set in an entirely analolous manner.

\stopsubsection

\startsubsection [title={Suffice It to Ignore DFT Matrix}]

\Result
{Definition}
{
Let \m {P} be fixed.
For \m {\MC {T}, \MC {T}' \subset \MC {T} \RB {N_p}}, define the \m {s, s'}-restricted orthogonality constant \m {\tau_{s,s'} \SB {P} >0} to be the smallest number such that
\Disp {
\NC \Nm {\IP {P _{\MC {T}} h, P _{\MC {T}'} h'}}
\leq \NC \tau_{s, s'} \SB {P} \cdot \VNm {h} _2 \VNm {h'} _2 \NR
}
}

From \quotation {Decoding from Linear Programming} (Cand\`es and Tao 2005), Lemma 1.1:

\Result
{Lemma}
{
Let \m {\M{P}} be fixed.
Then \m {\tau_{s, s'} \SB {\M{P}}} is bounded in both direction as follows,
\Disp {
\NC \d_{s+s'} \SB {\M{P}} -\max \SB {\CB {\d_s \SB {\M{P}}, \d_{s'} \SB {\M{P}}}}
\leq \NC \tau_{s, s'} \SB {\M{P}} \NR
\NC \leq \NC \d_{s+s'} \SB {\M{P}} \NR[+]
}
}

Thus \m {\d_s \SB {\M{P}}}, which defines how much the deformation of norm is, also tells us how much the inner product is deformed.
We may well keep track of \m {\d_s \SB {\M{P}}} only.
When \m {\M{P}} is clear, we may suppress it.

But each entry of \m {P} is a linear combination of products, and such reasoning does not work.
If only \m {P} were also an i.i.d.\ random matrix, the resulting proof would be easier.
Is the approach all but lost?

Observe
\Disp {
\NC \M{P} ^\Adj \M{P}
=\NC \RB {
   \RB {\M{K} ^\Tr \M {F}_R^\ast \M {F}_B^\ast}
   \otimes \RB {\M {K} \M {W}_R \M {W}_B}
}
\RB {
   \RB {\M {F}_B^\Tr \M {F}_R^\Tr \M{K}^\ast}
   \otimes \RB {\M {W}_B \M {W}_R \M {K}}
} \NR
\NC =\NC
\RB {\M {F}_B^\Tr \M {F}_R^\Tr \M{K}^\ast \M{K}^\Tr \M {F}_R^\ast \M {F}_B^\ast}
\otimes \RB {\M {W}_B \M {W}_R \M{K} \M{K}^\Adj \M {W}_R^\Adj \M {W}_B^\Adj} \NR
\NC =\NC
\RB {\M {F}_B^\Tr \M {F}_R^\Tr \M {F}_R^\ast \M {F}_B^\ast}
\otimes \RB {\M {W}_B \M {W}_R \M {W}_R^\Adj \M {W}_B^\Adj} \NR
\NC =\NC \RB {
   \RB {\M {F}_R^\ast \M {F}_B^\ast}
   \otimes \RB {\M {W}_R \M {W}_B}
}
\RB {
   \RB {\M {F}_B^\Tr \M {F}_R^\Tr}
   \otimes \RB {\M {W}_B \M {W}_R}
} \NR
\NC =\NC \M{Q} ^\Adj \M{Q} \NR
}
where as we defined before \m {\M{Q} :=\RB {\M {F}_B^\Tr \M {F}_R^\Tr} \otimes \RB {\M {W}_B \M {W}_R}}.

This implies
\Disp {
\NC \V{u} ^\Adj \M{P} ^\Adj \M{P} \V{u}
= \NC \V{u} ^\Adj \M{Q} ^\Adj \M{Q} \V{u} \NR[+]
}
or
\Result
{Lemma}
{
For any instance of \m {\M{P} \SB {\o_t, \o_r}} and \m {\M{Q} \SB {\o_t, \o_r}},
\Disp{
\NC \VNm {\M{P} \V{u}} _2
= \NC \VNm {\M{Q} \V{u}} _2 \NR[+]
}
}
\stopsubsection

\startsubsection [title={Expectation of Products of Matrix Entry}]

To make expressions more compact, introduce the indication function \m {\i} so that it equals 1 only if all (positive integer) argument it receives are unequal when they are seperated by semicolons between them, and equal when they are not.
For example, \m {\i \SB {7, 7, 7, 7} =1}, and \m {\i \SB {0, 0; 3, 3, 3} =1}, but \m {\i \SB {5, 5; 4, 6} =0}.

Formally,
\Disp {
\NC \NC \i \SB {
x_{1,1}, x_{1,2}, \ldots, x_{1,M\SB{1}} \mid
\ldots \mid
x_{N,1}, x_{N,2}, \ldots, x_{N,M\SB{N}}
} \NR
\NC =\NC \startcases
\NC 1, \MC x_{1,1} =\ldots =x_{1,M\SB{1}}
   \neq \dots
   \neq x_{N,1} =\ldots =x_{N,M\SB{N}} \NR
\NC 0, \NC \Q \Rm {otherwise} \NR
\stopcases \NR[+]
}

If we introduce
\Disp{
\NC \NC I_2 \SB {x_1, y_1, x_2, y_2} \NR
\NC =\NC \i \SB {x_1, x_2} \D \i \SB {y_1, y_2} \NR[+]
\NC \NC I_4 \SB {x_1, y_1, x_2, y_2, x_3, y_3, x_4, y_4} \NR
\NC =\NC \i \SB {x_1, x_2, x_3, x_4} \D \i \SB {y_1, y_2, y_3, y_4} \NR[+]
\NC \NC I_{2,2} \SB {x_1, y_1, x_2, y_2, x_3, y_3, x_4, y_4} \NR
\NC =\NC \RB {\i \SB {x_1, x_2 \mid x_3, x_4} +\i \SB {x_1, x_3 \mid x_2, x_4}} \i \SB {y_1, y_2, y_3, y_4} \NR
\NC \NC \FourQ + \i \SB {x_1, x_2, x_3, x_4} \RB {\i \SB {y_1, y_2 \mid y_3, y_4} +\i \SB {y_1, y_3 \mid y_2, y_4}} \NR
\NC \NC \FourQ + \i \SB {x_1, x_2 \mid x_3, x_4} \i \SB {y_1, y_2 \mid y_3, y_4} \NR
\NC \NC \FourQ + \i \SB {x_1, x_4 \mid x_2, x_3} \i \SB {y_1, y_4 \mid y_2, y_3} \NR[+]
}

Clearly, by construction,
\Disp {
\NC \NC \MB{E} \SB {\M{F}_B ^\ast \DB {n_R, n_Y}  \M{F}_B \DB {n_R', n_Y'}} \NR
\NC = \NC \MB{E} \SB {\M{W}_B ^\Tr \DB {n_R, n_Y}  \M{W}_B ^\Adj \DB {n_R', n_Y'}} \NR
\NC = \NC I_2 \SB {n_R, n_Y, n_R', n_Y'} \D M_{B,2}, \NR[+]
%
\NC \NC \MB{E} \SB {\M{F}_R ^\ast \DB {n_H, n_R}  \M{F}_R \DB {n_H', n_R'}} \NR
\NC = \NC \MB{E} \SB {\M{W}_R ^\Tr \DB {n_H, n_R}  \M{W}_R ^\Adj \DB {n_H', n_R'}} \NR
\NC = \NC I_2 \SB {n_H, n_R, n_H', n_R'} \D M_{R,2}, \NR[+]
%
\NC \NC \MB{E} \SB {\M{F}_B ^\ast \DB {n_R, n_Y}  \M{F}_B ^\ast \DB {n_R', n_Y'}  \M{F}_B \DB {n_R'', n_Y''} \M{F}_B \DB {n_R''', n_Y'''}} \NR
\NC = \NC \MB{E} \SB {\M{W}_B ^\Tr \DB {n_R, n_Y}  \M{W}_B ^\Tr \DB {n_R', n_Y'}  \M{W}_B ^\Adj \DB {n_R'', n_Y''} \M{W}_B ^\Adj \DB {n_R''', n_Y'''}}, \NR
\NC = \NC I_4 \SB {n_R, n_Y, n_R', n_Y', n_R'', n_Y'', n_R''', n_Y'''} \D M_{B,4} \NR
\NC \NC \FourQ
I_{2,2} \SB {n_R, n_Y, n_R', n_Y', n_R'', n_Y'', n_R''', n_Y'''} \D M_{B,2}^2 \NR[+]
%
\NC \NC \MB{E} \SB {\M{F}_R ^\ast \DB {n_H, n_R}  \M{F}_R ^\ast \DB {n_H', n_R'}  \M{F}_R \DB {n_H'', n_R''} \M{F}_R \DB {n_H''', n_R'''}} \NR
\NC = \NC \MB{E} \SB {\M{W}_R ^\Tr \DB {n_H, n_R}  \M{W}_R ^\Tr \DB {n_H', n_R'}  \M{W}_R ^\Adj \DB {n_H'', n_R''} \M{W}_R ^\Adj \DB {n_H''', n_R'''}}, \NR
\NC = \NC I_4 \SB {n_H, n_R, n_H', n_R', n_H'', n_R'', n_H''', n_R'''} \D M_{R,4} \NR
\NC \NC \FourQ
I_{2,2} \SB {n_H, n_R, n_H', n_R', n_H'', n_R'', n_H''', n_R'''} \D M_{R,2}^2 \NR[+]
%
\NC n_Y, n_Y'
= \NC 0, 1, 2, \ldots, N_Y -1 \NR
\NC n_R, n_R'
= \NC 0, 1, 2, \ldots, N_R -1 \NR
\NC n_H, n_H'
= \NC 0, 1, 2, \ldots, N_H -1 \NR
}
It will be similarly understood below that \m {n_Y, n_R, n_H} (and primed variables) run through all possible values.

Now, denote for short
\Disp {
\NC \M{F} :=\NC \M {F}_R \M {F}_B \NR[+]
\NC \M{W} :=\NC \M {W}_B \M {W}_R \NR[+]
}
Then, according to definition
\Disp {
\NC \NC \M{F} ^\ast \DB {n_H, n_Y} 
\M{F} \DB {n_H', n_Y'} \NR
\NC =\NC \sum_{n_R, n_R' =0}^{N_R-1}
\M{F}_R ^\ast \DB {n_H, n_R} 
\M{F}_B ^\ast \DB {n_R, n_Y} 
\M{F}_R \DB {n_H', n_R'}
\M{F}_B \DB {n_R', n_Y'} \NR[+]
}
By above,
\Disp {
\NC \NC \MB{E} \SB {\M{F} ^\ast \DB {n_H, n_Y}  \M{F} \DB {n_H', n_Y'}} \NR
\NC = \NC \MB{E} \SB {\M{W}^\Tr \DB {n_Y, n_H}  \M{W}^\Adj \DB {n_Y', n_H'}} \NR
\NC = \NC I_2 \SB {n_H, n_Y, n_H', n_Y'} \D N_R M_{B,2} M_{R,2} \NR[+]
}

By the same token,
\Disp {
\NC \NC \M{F} ^\ast \DB {n_H, n_Y} 
\M{F} ^\ast \DB {n_H', n_Y'} 
\M{F} \DB {n_H'', n_Y''}
\M{F} \DB {n_H''', n_Y'''} \NR
\NC =\NC \sum_{n_R, n_R', n_R'', n_R''' =0}^{N_R-1}
\M{F}_R ^\ast \DB {n_H, n_R} 
\M{F}_B ^\ast \DB {n_R, n_Y} 
\M{F}_R ^\ast \DB {n_H', n_R'} 
\M{F}_B ^\ast \DB {n_R', n_Y'}  \NR
\NC \NC \SixQ \M{F}_R \DB {n_H'', n_R''}
\M{F}_B \DB {n_R'', n_Y''}
\M{F}_R \DB {n_H''', n_R'''}
\M{F}_B \DB {n_R''', n_Y'''} \NR[+]
}
And it can be seen that
\Disp {
\NC \NC \MB{E} \SB {
   \M{F} ^\ast \DB {n_H, n_Y} \M{F} ^\ast \DB {n_H', n_Y'} 
   \M{F} \DB {n_H'', n_Y''} \M{F} \DB {n_H''', n_Y'''}
} \NR
\NC =\NC \MB{E} \SB {
   \M{W} ^\Tr \DB {n_H, n_Y} \M{W} ^\Tr \DB {n_H', n_Y'} 
   \M{W} ^\Adj \DB {n_H'', n_Y''} \M{W} ^\Adj \DB {n_H''', n_Y'''}
} \NR
\NC = \NC I_4 \SB {n_H, n_Y, n_H', n_Y', n_H'', n_Y'', n_H''', n_Y'''}
\D \RB {N_R M_{R,4} M_{B,4} +N_R \RB {N_R-1} M_{R,2}^2 M_{B,2}^2} \NR
\NC \NC \FourQ
I_{2,2} \SB {n_H, n_Y, n_H', n_Y', n_H'', n_Y'', n_H''', n_Y'''}
\D \F{1}{2} N_R \RB {N_R+1} M_{R,2}^2 M_{B,2}^2 \NR[+]
}

Furthermore, by the nature of Kronecker product, and the symmetry between \m {\M{F} ^\Tr} and \m {\M{W}}, we realize that the expectation of such product of entries of \m {\M{Q}} is just square of the product in such manner of entries of \m {\M{F} ^\Tr} (or \m {\M{W}}).
\Result
{Lemma}
{
\Disp {
\NC \NC \MB{E} \SB {\M{Q} ^\ast \DB {n_y, n_h}  \M{Q} \DB {n_y', n_h'}} \NR
\NC = \NC I_2 \SB {n_y, n_h, n_y', n_h'} \D N_R^2 M_{B,2}^2 M_{R,2}^2 \NR[+]
}
}
And
\Result
{Lemma}
{
\Disp {
\NC \NC \MB{E} \SB {
   \M{Q} ^\ast \DB {n_y, n_h} \M{Q} ^\ast \DB {n_y', n_h'} 
   \M{Q} \DB {n_y'', n_h''} \M{Q} \DB {n_y''', n_h'''}
} \NR
\NC = \NC I_4 \SB {n_y, n_h, n_y', n_h', n_y'', n_h'', n_y''', n_h'''} \D N_R^2 M_{R,4}^2 M_{B,4}^2 \NR
\NC \NC \FourQ
I_{2,2} \SB {n_y, n_h, n_y', n_h', n_y'', n_h'', n_y''', n_h'''} \D N_R^4 M_{R,2}^4 M_{B,2}^4 \NR[+]
}
}

Now, fix any test vector \m {\V{u} \in \MB{V}_\MB{C} \SB{N_Y}}, we shall study the probability concentration of
\Disp {
\NC \rho \SB {\o_t, \o_r}
:=\NC \VNm {\M{Q} \SB {\o_t, \o_r} \V{u}}_2^2 \NR[+]
}
by exploiting Chernoff bound.

\stopsubsection

\startsubsection [title={The First Moment}]

The Chernoff bound is most effective if all higher moments are known.
To begin with, study \m {\MB{E} \SB {\rho \SB {\o_t, \o_r}}}.
The calculation can be simplified if we expand the square of norm and pass through the expectation sign.
\Disp{
\NC \NC \MB{E} \SB {\rho \SB {\o_t, \o_r}} \NR
\NC = \NC \MB{E} \SB {
  \sum_{n_y =0}^{N_y-1}
  \Nm {\sum_{n_h =0}^{N_h-1}
  \M{Q} ^\ast \DB {n_y, n_h}  \V{u} \DB{n_h}} ^2
} \NR
\NC = \NC
\MB{E} \SB {
  \sum_{n_y =0}^{N_y-1}
  \sum_{n_h, n_h' =0}^{N_h-1}
  \M{Q} ^\ast \DB {n_y, n_h} \V{u} \DB{n_h}
  \M{Q} \DB {n_y, n_h'}  \V{u} ^\ast \DB{n_h'}
} \NR
\NC = \NC
\sum_{n_y =0}^{N_y-1}
\sum_{n_h, n_h' =0}^{N_h-1}
\MB{E} \SB {\M{Q} ^\ast \DB {n_y, n_h} \M{Q} \DB {n_y, n_h'} }
\V{u} \DB{n_h} \V{u} ^\ast \DB{n_h'} \NR
\NC = \NC N_R^2 M_{B,2}^2 M_{R,2}^2 N_y
\sum_{n_h, n_h' =0}^{N_h-1}
\i \SB {n_h, n_h'} \D \V{u} \DB{n_h} \V{u} ^\ast \DB{n_h'} \NR
\NC = \NC N_R^2 M_{B,2}^2 M_{R,2}^2 N_y \D \VNm {\V{u}} _2 ^2 \NR[+]
}
Thus, we may set
\Disp{
\NC \l_B
=\NC \F {1} {\R {N_Y}} \NR[+]
\NC \l_R
=\NC \F {1} {\R {N_R}} \NR[+]
}
To make
\Disp{
\NC \MB{E} \SB {\rho}
=\NC \VNm {\V{u}} _2 ^2
}

\stopsubsection

\startsubsection [title={The Second Moment}]

Similarly, we have to find, in advance, the second moment.
Pay attention to the order of indices in the product of entries of \m {\V{u}} and that of \m {\M{Q}}.
\Disp{
\NC \NC \MB{E} \SB {\rho \SB {\o_t, \o_r} ^2} \NR
%
\NC =\NC \MB{E} \SB {
  \RB {
    \sum_{n_R =0}^{N_R-1}
    \Nm {\sum_{n_Y =0}^{N_Y-1} \M{Q} ^\ast \DB {n_R, n_Y}  \V{u} \DB{n_Y}} ^2
  } ^2
} \NR
%
\NC = \NC
\MB{E} \Bigg[
  \sum_{n_y, n_y =0}^{N_y-1}
  \sum_{n_h, n_h', n_h'', n_h''' =0}^{N_h-1}
  \M{Q} ^\ast \DB {n_y, n_h} \M{Q} \DB {n_y, n_h'}
  \V{u} \DB{n_h} \V{u} ^\ast \DB{n_h'} \NR
\NC \NC \SixQ
  \M{Q} ^\ast \DB {n_y', n_h''} \M{Q} \DB {n_y', n_h'''}
  \V{u} \DB{n_h''} \V{u} ^\ast \DB{n_h'''}
\Bigg] \NR
%
\NC = \NC
  \sum_{n_y, n_y' =0}^{N_y-1}
  \sum_{n_h, n_h', n_h'', n_h''' =0}^{N_h-1}
  \MB{E} \SB{
    \M{Q} ^\ast \DB {n_y, n_h} \M{Q} ^\ast \DB {n_y', n_h''}
    \M{Q} \DB {n_y, n_h'} \M{Q} \DB {n_y', n_h'''}
  } \NR
\NC \NC \SixQ
\D \V{u} ^\ast \DB{n_h'} \V{u} ^\ast \DB{n_h'''}
\V{u} \DB{n_h} \V{u} \DB{n_h''} \NR
%
\NC = \NC
\sum_{n_y, n_y' =0}^{N_y-1}
\sum_{n_h, n_h', n_h'', n_h''' =0}^{N_h-1}
  \Big(
    I_4 \SB {n_y, n_h, n_y', n_h'', n_y, n_h', n_y', n_h'''} \D N_R^2 M_{R,4}^2 M_{B,4}^2 \NR
\NC \NC \SixQ
    +I_{2,2} \SB {n_y, n_h, n_y', n_h'', n_y, n_h', n_y', n_h'''} \D N_R^4 M_{R,2}^4 M_{B,2}^4
  \Big) \NR
\NC \NC \SixQ
  \D \V{u} ^\ast \DB{n_h'} \V{u} ^\ast \DB{n_h'''}
  \V{u} \DB{n_h} \V{u} \DB{n_h''} \NR
%
\NC = \NC \RB {N_R M_{R,4} M_{B,4} +N_R \RB {N_R-1} M_{R,2}^2 M_{B,2}^2} ^2 \NR
\NC \NC \SixQ \D \sum_{n_y, n_y' =0}^{N_y-1}
\sum_{n_h, n_h', n_h'', n_h''' =0}^{N_h-1}
   \i \SB {n_y, n_y'} \i \SB {n_h, n_h'', n_h', n_h'''} \NR
  \NC \NC \SixQ \D \V{u} ^\ast \DB{n_h'} \V{u} ^\ast \DB{n_h'''}
  \V{u} \DB{n_h} \V{u} \DB{n_h''} \NR
\NC \NC \FourQ + \RB {N_R^2 M_{R,2}^2 M_{B,2}^2} ^2
\sum_{n_y, n_y' =0}^{N_y-1}
\sum_{n_h, n_h', n_h'', n_h''' =0}^{N_h-1} \NR
\NC \NC \SixQ \Bigg(
  \i \SB {n_y, n_y'} \i \SB {n_h, n_h'' \mid n_h', n_h'''}
  +\i \SB {n_y, n_y'} \i \SB {n_h, n_h' \mid n_h'', n_h'''} \NR
  \NC \NC \SixQ +\i \SB {n_y \mid n_y'} \i \SB {n_h, n_h', n_h'', n_h'''}
  +\i \SB {n_y \mid n_y'} \i \SB {n_h, n_h' \mid n_h'', n_h'''}
\Bigg) \NR
  \NC \NC \SixQ \D \V{u} ^\ast \DB{n_h'} \V{u} ^\ast \DB{n_h'''}
  \V{u} \DB{n_h} \V{u} \DB{n_h''} \NR
%
\NC = \NC \RB {N_R M_{R,4} M_{B,4} +N_R \RB {N_R-1} M_{R,2}^2 M_{B,2}^2} ^2
\D N_y \sum_{n_h =0}^{N_h-1} \Nm {\V{u} \DB{n_h}} ^4 \NR
\NC \NC \FourQ
+ \RB {N_R^2 M_{R,2}^2 M_{B,2}^2} ^2 \NR
\NC \NC \SixQ \D \Bigg(
  N_y \sum_{n_h, n_h' =0}^{N_h-1} \i \SB {n_h \mid n_h'} \V{u} ^\ast \DB{n_h} ^2 \V{u} \DB{n_h'} ^2 \NR
  \NC \NC \SixQ +N_y \sum_{n_h, n_h' =0}^{N_h-1} \i \SB {n_h \mid n_h'} \Nm {\V{u} \DB{n_h}} ^2 \Nm {\V{u} \DB{n_h'}} ^2 \NR
  \NC \NC \SixQ +\RB {N_y^2 -N_y} \sum_{n_h =0}^{N_h-1} \Nm {\V{u} \DB{n_h}} ^4 \NR
  \NC \NC \SixQ +\RB {N_y^2 -N_y} \sum_{n_h, n_h' =0}^{N_h-1} \i \SB {n_h \mid n_h'} \Nm {\V{u} \DB{n_h}} ^2 \Nm {\V{u} \DB{n_h'}} ^2
\Bigg) \NR[+]
}

The third term is \m {\VNm {\V{u}} _4 ^4}.
The second and fourth terms are exactly \m {\VNm {\V{u}} _2 ^4 -\VNm {\V{u}} _4 ^4}.
The first term in the big parentheses can also be bounded by triangle inequality to be \m {\VNm {\V{u}} _2 ^4 -\VNm {\V{u}} _4 ^4}.
\Disp {
\NC \NC \sum_{n_h, n_h' =0}^{N_h-1}
\i \SB {n_h \mid n_h'} \V{u} ^\ast \DB{n_h} ^2 \V{u} \DB{n_h'} ^2 \NR
\NC \leq \NC \Nm {\sum_{n_h =0}^{N_h-1} \V{u} ^\ast \DB{n_h} ^2} ^2
-\sum_{n_h =0}^{N_h-1} \V{u} ^\ast \DB{n_h} ^4 \NR
\NC \leq \NC \VNm {\V{u}} _2 ^4 -\VNm {\V{u}} _4 ^4 \NR[+]
}

Thus,
\Disp {
\NC \NC \MB{E} \SB {\rho ^2} \NR
\NC = \NC \RB {N_R M_{R,4} M_{B,4} +N_R \RB {N_R-1} M_{R,2}^2 M_{B,2}^2} ^2
\D N_y \VNm {\V{u}} _4 ^4 \NR
\NC \NC \FourQ + \RB {N_R^2 M_{R,2}^2 M_{B,2}^2} ^2 \NR
\NC \NC \SixQ \Bigg(
  N_y \RB {\VNm {\V{u}} _2 ^4 -\VNm {\V{u}} _4 ^4}
  +N_y \RB {\VNm {\V{u}} _2 ^4 -\VNm {\V{u}} _4 ^4} \NR
  \NC \NC \SixQ +\RB {N_y^2 -N_y} \VNm {\V{u}} _4 ^4
  +\RB {N_y^2 -N_y} \RB {\VNm {\V{u}} _2 ^4 -\VNm {\V{u}} _4 ^4}
\Bigg) \NR
%
\NC =\NC \RB {1 +N_Y^{-2}} \VNm {\V{u}} _2 ^4
+\RB {-2 N_Y^2 +\RB {1 +2N_R^{-1} +N_R^{-2}} N_Y^{-2}} \VNm {\V{u}} _4 ^4 \NR
\NC \leq \NC \RB {1 +N_Y^{-2}} \VNm {\V{u}} _2 ^4 \NR[+]
}
In simplifying we used \m {N_H \gg N_R \gg N_Y \gg 1}. and the fact \m {\VNm {\V{u}} _4 \geq N_H^{-1/2} \VNm {\V{u}} _2}.

\stopsubsection

\startsubsection [title={The Higher Moments}]

In bounding the higher moments of \m {\rho}, let us be more generous.
Fix \m {k =3, 4, 5, \dots}.
If we introduce the change of variables
\Disp {
\NC l_Y :=\NC \Fl {n_Y /N_Y} \NR
\NC m_Y :=\NC n_Y \; \Rm{Mod}\; N_Y \NR
\NC l_H :=\NC \Fl {n_H /N_H} \NR
\NC l_H' :=\NC \Fl {n_H' /N_H} \NR
\NC m_H :=\NC n_H \; \Rm{Mod}\; N_H \NR
\NC m_H' :=\NC n_H' \; \Rm{Mod}\; N_H \NR
}
we may split the expression of \m {\rho} as follows.
Now, respectively: by Minkowski inequality; by the fact that \m {\M{F}_R} and \m {\M{F}_R} entries both have unity magnitude, by the identical distribution and by the definition of \m {M_{B,k}}; by Cauchy inequality: we have

\Disp{
\NC \NC \MB{E} \SB {\rho \SB {\o_t, \o_r} ^k} \NR
\NC =\NC \MB{E} \Bigg[ \Big(
  \sum_{l_Y, m_Y =0}^{N_Y-1}
  \sum_{n_R, n_R =0}^{N_R-1}
  \sum_{l_H, l_H', m_H', l_H'' =0}^{N_H-1} \NR
    \NC \NC \SixQ
    \M{F}_R ^{\Adj} \DB {l_H, n_R}
    \M{F}_R ^{\Tr} \DB {l_H', n_R'}
    \M{F}_B ^{\Adj} \DB {n_R, l_Y}
    \M{F}_B ^{\Tr} \DB {n_R', l_Y} \NR
    \NC \NC \SixQ
    \M{W}_R ^{\ast} \DB {m_H, n_R}
    \M{W}_R \DB {m_H', n_R'}
    \M{W}_B ^{\ast} \DB {n_R, m_Y}
    \M{W}_B \DB {n_R', m_Y} \NR
  \NC \NC \SixQ \V{u} \DB{l_H N_H +m_H}
  \V{u} ^\ast \DB{l_H' N_H +m_H'}
\Big) ^k \Bigg] \NR
%
\NC \leq \NC \Bigg(
  \sum_{l_Y, m_Y =0}^{N_Y-1}
  \sum_{n_R, n_R =0}^{N_R-1}
  \sum_{l_H, l_H', m_H', l_H'' =0}^{N_H-1} \NR
    \NC \NC \SixQ
    \Big(
      \MB{E} \SB {\Nm {\M{F}_B ^{\Adj} \DB {n_R, l_Y}} ^k}
      \MB{E} \SB {\Nm {\M{F}_B ^{\Tr} \DB {n_R', l_Y}} ^k} \NR
      \NC \NC \SixQ
      \MB{E} \SB {\Nm {\M{W}_B ^{\ast} \DB {n_R, m_Y}} ^k}
      \MB{E} \SB {\Nm {\M{W}_B \DB {n_R', m_Y}} ^k}
    \Big) ^{1/k} \NR
  \NC \NC \SixQ
  \Nm {\V{u} \DB{l_H N_H +m_H}}
  \Nm {\V{u} ^\ast \DB{l_H' N_H +m_H'}}
\Bigg) ^k \NR
%
\NC =\NC
  \Big(
    \sum_{l_Y, m_Y =0}^{N_Y-1}
    \sum_{n_R, n_R =0}^{N_R-1}
    \sum_{l_H, l_H', m_H', l_H'' =0}^{N_H-1}
      M_{B,k} ^{4/k}
    \V{u} \DB{l_H N_H +m_H}
    \V{u} ^\ast \DB{l_H' N_H +m_H'}
  \Big) ^k \NR
%
\NC = \NC
  M_{B,k} ^4
  \Nm {
    N_Y^2 N_R^2
    \sum_{n_h, n_h' =0}^{N_h-1}
    \V{u} \DB{n_h} \V{u} ^\ast \DB{n_h'}
  } ^k \NR
\NC \leq \NC \Gamma \SB {\F{k}{2} +1} ^4 N_Y^{2k} N_R^{2k} \VNm {\V{u}} _2 ^{2k} \NR
}

\stopsubsection

\startsubsection [title={Using Chernoff Bound}]

Without loss of generality, let us assume
\Disp {
\NC \VNm {\V{u}}
=\NC 1 \NR[+]
}
Imitating the Markov inequality argument in classical probability, we have, 
for any \m {s_{+} >0} so that, supposedly, both sides \m {< \infty},
\Disp {
\NC \NC \MB{P} \SB {\rho \geq \RB {1 +\e} \MB{E} \SB {\rho}} \NR
\NC =\NC \MB{P} \SB {\rho \geq \RB {1 +\e}} \NR
\NC =\NC \MB{P} \SB {\Ss {e} ^{\R {s_{+} \rho}} \geq \Ss {e} ^{\R {s_{+} \RB {1 +\e}}}} \NR
\NC =\NC \Ss {e} ^{-s_{+} \RB {1 +\e}} \MB{E} \SB {\Ss {e} ^{\R {s_{+} \rho}}} \NR[+]
}
Assuming the convergence of power series of \m {\Ss {e} ^{\R {s_{+} \rho}}} for a moment, by various result in previous subsections, we get
\Disp {
\NC \NC \MB{E} \SB {\Ss {e} ^{\R {s_{+} \rho}}} \NR
\NC \leq \NC 1
+s_+ ^{1/2} \MB{E} \SB {\rho ^{1/2}}
+\F{s_+ ^{1/2}} {2} \MB{E} \SB {\rho}
+\F{s_+ ^{3/2}} {6} \MB{E} \SB {\rho ^{3/2}}
+\F{s_+ ^2} {24} \MB{E} \SB {\rho ^2}
+\sum_{k=5}^\infty \F{1} {k!} \MB{E} \SB {s_+ ^{k/2} \rho ^{k/2}} \NR[+]
}
By power mean inequality,
\Disp {
\NC \leq \NC 1
+\VNm {\V{u}} _2 s_{+} ^{1/2}
+\F{1} {2} \VNm {\V{u}} _2 ^2 s_{+}
+\F{1} {6} \RB {1 +N_Y^{-2}} ^{3/4} \VNm {\V{u}} _2 ^3 s_{+} ^{3/2} \NR
\NC \NC \SixQ +\F{1} {24} \RB {1 +N_Y^{-2}} \VNm {\V{u}} _2 ^4 s_{+} ^2
+\sum_{k=5}^\infty \F{s_{+} ^{k/2}} {k!}
  \Gamma \SB {\F{k}{2} +1} ^4 N_Y^{2k} N_R^{2k}
  \VNm {\V{u}} _2 ^k \NR
\NC \leq \NC 1
+s_{+} ^{1/2}
+\F{1} {2} s_{+}
+\F{1} {6} \RB {1 +N_Y^{-2}} ^{3/4} s_{+} ^{3/2} \NR
\NC \NC \SixQ +\F{1} {24} \RB {1 +N_Y^{-2}} s_{+} ^2
+\sum_{k=5}^\infty \F{s_{+} ^{k/2}} {k!}
  \Gamma \SB {\F{k}{2} +1} ^4 N_Y^{2k} N_R^{2k} \NR[+]
}

Meanwhile, we recall the Stirling's Approximation
\Result
{Lemma}
{
For \m {n>1},
\Disp {
\NC \R {2 \pi} n ^{n+1/2} \Ss {e} ^{-n+1/(12n+1)}
<\NC \Gamma \SB {n+1} \NR
\NC <\NC \R {2 \pi} n ^{n+1/2} \Ss {e} ^{-n+1/12n} \NR
}
}

Thus,
\Disp {
\NC \F {\RB {\Gamma \SB {k/2+1}}^4} {\Gamma \SB {k+1}}
\leq \NC \F {\RB {\R {2\pi} \RB {k/2} ^{k/2+1/2} \Ss {e} ^{-k/2 +1/6k}} ^4}
  {\R {2\pi} \; k ^{k+1/2} \; \Ss {e} ^{-k +1/(12k+1)}} \NR
\NC =\NC \RB {2\pi} ^{3/2} \; k ^{3/2} \; 2 ^{-2k-4} \; \Ss {e} ^{5/4k} \NR
}

Setting
\Disp {
\NC C\SB{s}
=\NC \F{1}{4} N_Y N_R \R{s} \NR
\NC c_0
=\NC \F{1}{16} \RB {2\pi} ^{3/2} \Ss {e} ^{1/4} \NR
}
we may bound the last series summation by the integration
\Disp {
\NC \NC \F{1}{c_0} \sum_{k=5}^\infty \F{s_{+} ^{k/2}} {k!}
  \Gamma \SB {\F{k}{2} +1} ^4 N_Y^{2k} N_R^{2k} -1
  \VNm {\V{u}} _2 ^k \NR
\NC \leq \NC \int _{5} ^{\infty} C ^x x ^{3/2} \Ss {d} x \NR
\NC = \NC \F{5^{3/2} C^5} {-\log C}
+\F{3 \D C^5 \R{5}} {2 \RB {-\log C}^2}
+\F{3 \D C^5} {4 \R{5} \RB {-\log C}^3} \NR
}

% % % % % % % % % % % % % % % % % % % % % % % % % % % % % % % %

Similarly, for some \m {s_{-} >0}.
\Disp {
\NC \NC \MB{P} \SB {\rho \leq \RB {1 -\e} \MB{E} \SB {\rho}} \NR
\NC =\NC \Ss {e} ^{s_{-} \RB {1 -\e} \VNm {\V{u}} _2 ^2}
\RB {1 -\VNm {\V{u}} _2 ^2 +\F{1} {2} \VNm {\V{u}} _2 ^4 +\sum_{k=3}^\infty \F{\RB {-1}^{k}} {k!} \MB{E} \SB {\rho^k}} \NR
}

Since the quantity \m {\Nm {\rho -\MB{E} \SB {\rho}}} is linear in \m {\VNm {\V{u}} _2}, by a scaling argument we may set
\Disp {
\NC \VNm {\V{u}} _2
=\NC 1 \NR
}

It follows that
\Disp{
\NC \Ss{Var} \SB {\VNm {\M{P} \V{u}} _2 ^2}
= \NC \Ss{Var} \SB {\VNm {\M{Q} \V{u}} _2 ^2} \NR
\NC \leq \NC N_Y^{-2} \VNm {\V{u}} _2 ^4 \NR[+]
}
Thus, by Chebyshev bound,
\Result
{Theorem}
{
Let \m {\M{P}} be generated randomly according to ().
Then, for any fixed \m {\V{u} \in \MB {V}_{\MB {C}} \SB {N_Y}}, and for any \m {\e >0},
\Disp{
\NC \MB{P}
\SB {
  \Nm {\VNm {\M{P} \V{u}} _2 ^2 -\VNm {\V{u}} _2 ^2}
  \geq \e \VNm {\V{u}} _2 ^2
}
\leq \NC \F {1} {\e^2} N_Y^{-2} \NR[+]
}
}

According to Lemma 5.1 in Baraniuk et.\ al.\ (2008), substitution of relevent quantities yields
\Result
{Lemma}
{
Suppose the probability, according to event space \m {\o_t, \o_r}, that RIP holds for \m {\M{P}} w.r.t.\ \m {\d_s} is \m {p}, then
\Disp{
\NC 1 -p
\leq \NC 12^s \D \d_s^{-\RB {s+2}} N_Y^{-2} \NR[+]
}
}

\stopsubsection
\stopsection



