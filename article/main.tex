
\documentclass[journal]{IEEEtran}

\begin{document}

\title{Dantzig Selector Applied on mm-Wave MIMO Channel Estimation and and Error Analysis}

\author{Tzu-Yu Jeng
        Hsuan-Jung Su%
\thanks{Hsuan-Jung Su is with the Department
of Electrical Engineering, National Taiwan University.}
\thanks{Manuscript received April 1, 2020; revised April 1, 2021.}}

\markboth{IEEE Journal of whatever,~Vol.~0, No.~0, June~2020}%
{Jeng and Su: Dantzig Selector Applied on mm-Wave MIMO Channel Estimation}

\maketitle

\begin{abstract}
Multiple-input multiple-output communication systems in the millimeter-wave band are gradually being adopted.
A larger antennae array makes channel estimation more difficult.
Meanwhile, hybrid beamforming, which utilizes fewer RF chains, is applied too, so that new estimation methods have to be designed.
This can be seen as a compressive sensing problem, for which Orthogonal Matching Pursuit is usually used.

We consider a hybrid structure with a uniform linear array with both precoders and combiners on each side.
We generate random beamforming matrices and use Dantzig Selector to estimate the channel in the space frequency domain.
Then, we give a quantitative bound (which holds for high probability) on the expected error norm.
To reduce the complexity, we cast it as a linear program, and suggest the basis pursuit denoising form.
Numerical results show that DS gives a superb regularization, especially when the sample is less sufficient and the noise level is higher.
We therefore propose that DS may be used, when the number of RF chains is more limited or when fewer stages of estimation are possible.
\end{abstract}

\begin{IEEEkeywords}
channel estimation, compressive sensing, sparsity, Dantzig Selector, regularization, restricted isometry
\end{IEEEkeywords}



\section{Introduction}

I wish you the best of success.
%\IEEEPARstart{T}{his} demo file is intended to serve as a ``starter file''
%for IEEE journal papers produced under \LaTeX\ using
%IEEEtran.cls version 1.8b and later.

\hfill mds
 
\hfill August 26, 2015

\subsection{Subsection Heading Here}
Subsection text here.



\section{Conclusion}
The conclusion goes here.



\appendices
\section{Proof of the First Zonklar Equation}
Appendix one text goes here.

\section{Proof of the First Zonklar Equation}
Appendix two text goes here.


\section*{Acknowledgment}

The authors would like to thank...


% Can use something like this to put references on a page
% by themselves when using endfloat and the captionsoff option.
\ifCLASSOPTIONcaptionsoff
  \newpage
\fi

\begin{thebibliography}{1}

\bibitem{IEEEhowto:kopka}
H.~Kopka and P.~W. Daly, \emph{A Guide to \LaTeX}, 3rd~ed.\hskip 1em plus
  0.5em minus 0.4em\relax Harlow, England: Addison-Wesley, 1999.

\end{thebibliography}


\end{document}


