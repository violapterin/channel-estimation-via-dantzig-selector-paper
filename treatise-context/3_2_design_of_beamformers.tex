\startsection [title={Design of beamformers}]

\startsubsection [title={Design of beamformers}]

We want to set the four beamformers to be i.i.d.\ random matrix, hoping that the resulting \m {P} has RIP.
Let us simply set each entry of \m {\M {F} _B} to be i.i.d.\ normal random variables with mean \m {0}, standard deviation \m {1/2}, multiplied by a normalizing constant \m {\l_B >0}.
We know that the magnitude \m {\RB {\M {F} _B} _{\SB {n_R, n_B}}} follows Rayleigh distribution.
\m {\RB {\M {W} _B ^\Tr} _{\SB {n_B, n_R}}} is set similarly.
In addition, let \m {\RB {\M {F} _R} _{\SB {n_H, n_R}}} be uniformly distributed on the unit circle on the complex plane, multiplied by a normalizing constant \m {\l_R >0}.
\m {\RB {\M {W} _R ^\Tr} _{\SB {n_R, n_H}}} is set similarly.

It turns out we have results that guarantee that both \m {\M {F} _B ^\Tr} and \m {\M {F} _R ^\Tr} have RIP, and the same can be said of \m {\M {W} _B} and \m {\M {W} _R}.
Consider a sparse \m {\V {u}} with unity norm, then we are sure that \m {\M {F} _R ^\Tr \V {u}} has almost unity norm.
If, furthermore, \m {\M {F} _R ^\Tr \V {u}} is sparse, we see that \m {\M {F} _B ^\Tr \M {F} _R ^\Tr \V {u}} has almost unity norm, and the RIP of \m {\M{P}} also follows.

However, we cannot be sure that \m {\M {F} _B ^\Tr \M {F} _R ^\Tr}, the product of two RIP matrices, has RIP, nor do we know that \m {\M {P}}, the result of Kronecker product, has RIP.
In fact, the statement seems to be false in general.
What now?
Can we argue instead that \m {\M {F} _B ^\Tr \M {F} _R ^\Tr \V {u}} has similar norm with \m {\M {F} _R ^\Tr \V {u}}?

\stopsubsection

\startsubsection [title={Operator norm of digital beamformers}]

Use \m {\chi _{i,j}} to denote independent complex standard normal random variables.
It follows that
%
\DispNum {C':C':WB:j2} {
\NC \VNm {\M {W} _B \V {u}} _2
=\NC \R {
   \sum _{i=0} ^{N_B-1} \VNm {\sum _{j=0} ^{N_R-1} \F {1} {\R{N_B}} \chi _{i,j} \V {u} _{j}} _2 ^2} \NR
}
%
We need a large deviation result for i.i.d.\ sum of chi-square random variables.
To do so, we apply Lemma \m {1} in Laurent and Massart \cite [LaM00] for equal weights, with \m {x > 0} below.
By some calculation, we have the upper tail inequality
%
\DispNum {P:x:ex:ex} {
\NC \MB {P} \SB {\VNm {\M {W} _B \V {u}} _2 ^2 \geq 1 + 2 \R {\F {x} {N_B}} + \F {2x} {N_B}}
\leq \NC \Ss {e} ^{-x} \NR
}
%
and the lower tail inequality
%
\DispNum {P:x:ex:ex!} {
\NC \MB {P} \SB {\VNm {\M {W} _B \V {u}} _2 ^2 \leq 1 - 2 \R {\F {x} {N_B}}}
\leq \NC \Ss {e} ^{-x} \NR
}
%
With substitution \m {\d_s =16x^2 /N_B^2}, the upper tail can be simplified with \m {x <N_B}, and we get the following bound for overall failure probability.

\Result
{Proposition}
{
For \m {0 <\d_s <1}, it holds that
%
\DispNum {W:1:ds:ds} {
\NC \Nm {\VNm {\M {W} _B \V {u}} _2 ^2 - 1}
\geq \NC \d_s \NR
}
for probability \m {p}, with
\DispNum {1:p:2e:s4} {
\NC 1 -p
\leq \NC 2 \Ss {e} ^{-N_B \R {\d_s} /4} \NR
}
}

\stopsubsection

\startsubsection [title={Operator norm of analog beamformers}]

In Haviv and Regev, \quotation {The restricted isometry property of subsampled Fourier matrices} \cite [KlM17], Theorem \m {4.5} states a DFT submatrix has \m {\d_s} RIP with failure probability lower than a certain Big-O bound.
We argue that their argument can be refined to give precise constants, by Chernoff-Hoeffding bounds and their theorem \m {2.2} with (say) \m {C \leftarrow 1/3}.
The conclusions, 
Their Theorem \m {3.7} and Theorem \m {4.5} may be changed accordingly, so that the relation below gives a bound for failure probability for \m {\d_s /N_H s} RIP.
%
\DispNum {N:R:sd':NH} {
\NC N_R
\geq \NC \F {s} {\d_s^2} \RB {\log \F{1}{\d_s}}^2 \RB {\log \F{s}{\d_s}}^2 \log N_H \NR
}
%
Drop \m {\log \SB {1 /\d_s}^2} and the denominator \m {\d_s} in \m {\log \SB {s /\d_s}^2} in \Rf {N:R:sd':NH}, to get the statement below.

\Result
{Theorem}
{
Suppose
%
\DispNum {N:R:sd'NH} {
\NC N_R
\geq \NC \F {s} {\d_s^2} \RB {\log s}^2 \log N_H \NR
}
%
Then \m {\M {F}_R} (respectively \m {\M {W} _R}) has \m {\d_s} RIP for probability \m {p}, with
\DispNum {1:p:ex:d's} {
\NC 1 -p
\leq \NC \RB {\F {\d_s} {N_H s}} ^{1/3} \NR
}
}

\stopsubsection

\startsubsection [title={Confirming the restricted isometry}]

So, how do we relate RIP of \m {\M {F} _R ^\Tr}, \m {\M {F} _B ^\Tr}, \m {\M {W} _R}, and \m {\M {W} _B} to RIP of \m {\M {P}}?
Notice
\DispNum {P:2:FB:WR} {
\NC \VNm {P} _2
=\NC \VNm {\M {F}_B^\Tr \M {F}_B^\Tr} _2 \VNm {\M {W}_B \M {W}_R} _2 \NR
}
Since both \m {\M {F}_B^\Tr \M {F}_B^\Tr} and \m {\M {F}_B^\Tr \M {F}_B^\Tr} are \m {\d_s} RIP, it does appear, if we ignore for a moment the sparsity constraint, that \m {\M {P}} has approximately \m {2\d_s} RIP.
However, it is doubtful to assume that their Kronecker product still has RIP.

We argue as follows.
If all angles \m {\th_l, \f_l} are exactly the multiples of \m {2 \pi/N_H}, then
\DispNum {G:G:Ud:dU} {
\NC \M {G}
=\NC \M {U} \Rm {diag} \SB {\V {d}'} \M {U} ^\Adj \NR
}
for some \m {S}-sparse \m {\V {d}'} and unitary \m {\M {U}}.
Clearly,
\DispNum {G:2:g2:g2} {
\NC \VNm {\M {G}} _2
\NC =\VNm {\Rm {abs} \SB {\V {g}}} _2 \NR
}
where \m {\Rm {abs}} denotes entrywise absolute value.
Then if we notice that \m {\M {W} _B \M {W} _R \M {K} \M {U}} and \m {\M {U} ^\Adj \M {K}^\Adj \M {F} _R^\Adj \M {F} _B^\Adj} has \m {\d_s} RIP too, we see
\DispNum {W:2:WB:d's} {
\NC \NC \VNm {\M {W} _B \M {W} _R \M {H} \M {F} _R \M {F} _B} _2 \NR
%
\NC =\NC \VNm {
   \M {W} _B \M {W} _R \M {K} \M {U} \D
   \Rm {diag} \SB {\Rm {abs} \SB {\V {d}'}} \D
   \M {U} ^\Adj \M {K} ^\Adj \M {F} _R \M {F} _B} _2 \NR
%
\NC =\NC \VNm {
   \M {W} _B \M {W} _R \M {K} \M {U}
   \R {\Rm {abs} \SB {\V {d}'}} \D
   \R {\Rm {abs} \SB {\V {d}'}}
   \M {U} ^\Adj \M {K} ^\Adj \M {F} _R \M {F} _B} _2 \NR
%
\NC =\NC \VNm {
   \M {W} _B \M {W} _R \M {K} \M {U}
   \R {\Rm {abs} \SB {\V {d}'}}} _2 \D
   \VNm {\M {F} _B ^\Adj \M {F} _R ^\Adj \M {K} \M {U}
   \R {\Rm {abs} \SB {\V {d}'}}} _2 \NR
%
\NC \eqsim \NC \VNm {\M {G}} _2 \RB {1 +2\d_s} \NR
}
Thus, even we cannot make sure that \m {\M {P}} is \m {\d_s} RIP with respect to all possible \m {\V {g}}'s, at least \m {\M {P}} has RIP with respect to relevant values of \m {\V {g}} according to the form \Rf {H:H:l0:ql}.

\stopsubsection

\stopsection
