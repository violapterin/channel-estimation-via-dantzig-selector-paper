\documentclass[12pt]{article}

\usepackage{amsmath,amssymb}
\usepackage{graphicx}

\author{Tzu-Yu Jeng}
\date{\today}
\title{Proposal (Sketch)}
\begin{document}
\maketitle

\section{Introduction}

Multiple-input multiple-output (MIMO) systems with a large number of antennae of both sides (hereafter massive MIMO) is expected to improve spectral efficiency, but presents new obstacles as well.
The hardware overhead due to large number of antennas increases complexity and power consumption, and new precoders are being invented to take account of this.
It is important to devise feasible simulation method, as the systems themselves are growing more complicated, and the cost of simulations is large.

Though the problem of finding precoder is old, often times we encounter new settings that suggests us to use a technique that is more agnostic to the actual traits of the channel.
For example, when 3D channel is considered with 2D active antenna arrays used, new precoding technique is required.
3D channel considers not only the horizontal position, but also the height, and using expressions that involves longitudinal as well as azimuthal angles.

We apply a precoding matrix \(W\) chosen among the Fourier-kernel-like matrix, which has entry that solely consists of roots of unity, and study the sum rate with respect to \(HW\).
Despite the extensive discussion of precoding matrices out of the academic interest, in reality usually the Fourier-kernel-like rows and columns are used.
In short, every row of the precoding matrix is a vector consisting of the roots of unity with equally spaced arguments.
This is true according to (for example) 3GPP Release 14, justifying our decision.

We choose the sum rate to be the target function.
The reader may think that the channel capacity, which let us call \(C\), is the quantity to be maximized, but in fact the explicit expression of \(C\) in 
nor is it clear what it means by ``capacity''.
In fact, we usually say ``capacity regions'', and no single number can represent for that.
As a result, researchers often use sum rate \(\log\det(I+H)\) for normalized AWGN case.

Previous literature, I believe, usually simplifies the scenario in two aspects, that is, the consideration of a large number \(K\) of subcarriers, and the two-staged precoding.

Love et.\ al.\ (2003) transform the precoding problem as a line packing problem on Grassmannian manifold with respect to a self-defined distance function, 
Sohrabi and Yu (2016a) consider a single given instance of the channel matrix.
And another paper by Sohrabi and Yu (2016b), as a follow up of their previous work, deals moreover with the case of several subcarriers, but pulls them out and did not exploit more possibility of much more optimization.
[....??]

Two-staged precoding is hardly seen fomulated in literature, which is something I don't understand.
Two-staged precoding is actually considered in 3GPP release 13, and is beneficial 
If we set (for example in Li et.\ al.\ 2016)
\begin{gather*}
W =W_2 W_1
\end{gather*}
Such distinction do not add complexity without reason, but \(W_2\) can take account of slow-varying channel traits, and \(W_1\) of fast-varying ones.

These are both extremely realistic concerns, and it seems not to be fully investigated.

Since it not easy to derive analytical result, I gather maybe metaheuristic algorithms may be used.

\section{Problem formulation}

For convenience, denote by \(\mathbb{M}(N_1',N_2')\) family of matrices having dimension \(N_1'\) by \(N_2'\).
Let \(0 \leq N_D \leq N_H\), where \(N_D,N_H \in \mathbb{N}\).

Define matrices
\begin{gather*}
W_1(m), W_2(m,n) \in \mathbb{M}(N_D,N_H)
\end{gather*}
here, for predefined set of matrices \(\mathcal{W}_1\) and \(\mathcal{W}_2\),
\begin{gather*}
W_1(m) \in \mathcal{W}_1,\quad
W_2(m,n) \in \mathcal{W}_2.
\end{gather*}

Moreover, let random instances of channel matrices for different subcarriers be given,
\begin{gather*}
H(m,n,l) \in M(N_H,N_H),\; m=1,\dotsc,M,\; n=1,\dotsc,N,\; l=1,\dotsc,L
\end{gather*}
\(H(m,n,l)\)'s may in general be correlated.

Then define the problem
\begin{align*}
& \hat{W}_1(m),\; \hat{W}_2(m,n),\; m=1,\dotsc,M,\; n=1,\dotsc,N \\
\leftarrow &\underset{\in \mathcal{W}_1(m) \in \mathcal{W}_1,\; W_2(m,n) \in \mathcal{W}_2}{\mathrm{argmax}}
\sum_{m=1}^M \sum_{n=1}^N \sum_{l=1}^L \log \det (I +G(m,n,l) G(m,n,l)^\dagger),\\
& G(m,n,l) =H(m,n,l) W_2(m,n) W_1(m).
\end{align*}

Question: Are there algorithms that, when averaged through instances of \(H(m,n,l)\), find the optimal \(\hat{W}_1(m), \hat{W}_2(m,n)\) with low complexity?

As realisitic setting is, say, that there are 1200 subcarriers for every symbol.
Among them, there are 100 subcarriers for every resource block (giving 12 resource blocks indexed by \(m\)), and 4 subcarriers for every subbands (giving 25 subbands indexed by \(n\) every resource block).

\section{Plan}

I want to simulate the two-stage precoding problem for randomly given channel instances, using metaheuristic methods like simulated annealing.
Although the choice of precoding matrix is confined within a finitely-many set, the search is highly non-trivial, considering the complexity would be large 
It may then be necessary to design, in addition, a data structure to facilitate such search.

I will look up analysis of such algorithm, and compare theoretical prediction to performance.
However, it is hard to imagine there will be closed form solution, and thus, it is not clear what quantitative justification that matches simulation can be given.

It is also natural to extend such technique if successful for dual-stage precoding to hybrid precoding, and that may be easier to be published.

\section{References}

\begin{enumerate}

\item F Sohrabi and W Yu, ``Hybrid Digital and Analog Beamforming Design for Large-Scale Antenna Arrays'', \textit{IEEE Journal of Selected Topics in Signal Processing}, Vol. 10, No. 3, April 2016.

\item F Sohrabi and W Yu, ``Hybrid Analog and Digital Beamforming for OFDM-Based Large-Scale MIMO Systems''. 2016 IEEE 17th International Workshop on Signal Processing Advances in Wireless Communications (SPAWC).

\item A Goldsmith, S A Jafar, N Jindal, and S Vishwanath, ``Capacity Limits of MIMO Channels'', \textit{IEEE Journal on Selected Areas in Communications}, Vol. 21, No.5, June 2003.

\item R W Heath Jr, N González-Prelcic, S Rangan, W Roh, and A M Sayeed, ``An Overview of Signal Processing Techniques for Millimeter Wave MIMO Systems''. \textit{IEEE Journal of Selected Topics in Signal Processing}, Vol. 10, No. 3, April 2016.

\item H Li, Q Gao, R Chen, R Tamrakar, S Sun, and W Chen, ``Codebook Design for Massive MIMO Systems in LTE'', \textit{2016 IEEE 83rd Vehicular Technology Conference (VTC Spring)}.

\item 3GPP Specifications, TS 36.213 V14.7.0. Retrived from: \\ ``http://www.3gpp.org/specifications''.

\end{enumerate}

\end{document}
