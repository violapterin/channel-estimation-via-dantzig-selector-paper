\startsection [title={Restricted Isometry of Effective Beamformer}]

\startsubsection [title={Design of Beamformer Matrices}]

Our plan is setting the four precoder matrices to be i.i.d.\ random matrix, hoping that the resulting \m {P} has RIP.

Let us make it simple and set each entry of \m {\M {F} _B} to be i.i.d.\ Normal r.v.\ with mean 0, standard deviation \m {1/2}, multiplied by a normalizing constant \m {\l_B >0}.
We know that the magnitude \m {\M {F} _B \DB {n_R, n_Y}} follows Rayleigh distribution.
\m {\M {W} _B ^\Tr \DB {n_Y, n_R}} is set in an entirely analolous manner.

In addition, let \m {\M {F} _R \DB {n_H, n_R}} be uniformly distributed on the unit circle on the complex plane.
\m {\M {W} _R ^\Tr \DB {n_Y, n_R}} is set in an entirely analolous manner.

Existent result () guarantee that both \m {\M {F} _B ^\Tr} and \m {\M {F} _R ^\Tr} are RIP, and same can be said of \m {\M {W} _B} and \m {\M {W} _R}.
Consider a sparse \m {\V {u}} or unity norm, then existent result guarantees that \m {\M {F} _R ^\Tr \V {u}} has almost unity norm.
If, furthermore, \m {\M {F} _R ^\Tr \V {u}} is sparse, we see that \m {\M {F} _B ^\Tr \M {F} _R ^\Tr \V {u}} has almost unity norm, and the RIP of \m {\M{P}} also follows by construction.

However, it is not clear that \m {\M {F} _B ^\Tr \M {F} _R ^\Tr}, the product of two RIP matrices, have RIP, nor that, \m {\M {P}}, the result of Kronecker product, is RIP.
In fact, the statement seems to be false in general.
Indeed, since a \m {\M {F} _R ^\Tr} of the same \m {\d} is sparse, up to a factor of isometry matrix, and \m {\M {F} _R ^\Tr} is generated randomly, it would seem to us that the probability that \m {\M {F} _R ^\Tr \V {u}} be sparse is exceedingly small.

But all is not lost.
At least we have made sure that \m {\M {F} _R ^\Tr \V {u}} has norm close to unity.
If we have a stronger result, that \m {\M {F} _B ^\Tr \V {u}} is almost unitary even for \It {not-sparse} vectors, then we are done.
That does sound somehow plausible, considering the Normal entries of \m {\M {F} _B ^\Tr} would introduce a Central-limit-theorem-like sum, though the summands are not independent, but weakly dependent.

But how does we relate RIP of \m {\M {F} _R ^\Tr}, \m {\M {F} _B ^\Tr}, \m {\M {W} _R}, and \m {\M {W} _B} to that of \m {\M {P}}?
Indeed, if it were not for the sparsity constraint, we would have
%
\DispNum {P,2;FB,R2} {
\NC \VNm {\M {P}}_2
=\NC \VNm {\M {F}_B^\Tr \M {F}_R^\Tr} _2 \VNm {\M {W}_B \M {W}_R} _2 \NR
}
%
However, it is not true that \m {\VNm {\M {P}} _2 \approx 1 \pm \d}.
Nevertheless, if columns of \m {\M {G}} is also sparse, RIP follows respectively.
Thus we would like to assume that all \m {N_H} but \m {\R {s}} columns of \m {\M {G}} are all-zero vector.
This would seem more likely if \m {s} is chosen to be less than \m {\R {N_H}}, and we decide that
%
\DispNum {s,s;2N,NH} {
\NC s
=\NC 2 \log N_H \NR
}
%
And
%
\Disp {
\NC q
\sim \NC 2 \RB {q _{\Rm {RIP}, B} +q _{\Rm {RIP}, R}} +q _{\s} +q _{\Rm {Sp}} \NR
\NC \sim \NC N_H ^{-1/2}
+N_H^{-2/3} \RB {\log N_H} ^{1/3}
+\RB {\log N_H} ^{-3}
+N_H ^{-1/2} \NR
}

\stopsubsection

\startsubsection [title={Operator Norm of Digital Beamformers}]

To start, use \m {\chi _{i,j}} to denote Complex Standard Normal random variables, and to simplify notation, let them be independent if they have different arguments.
With this,
%
\DispNum {C',C';WB,j2} {
\NC \VNm {\M {W} _B \V {u}}
=\NC \R {
   \sum _{i=0} ^{N_B-1} \VNm {\sum _{j=0} ^{N_R-1} \F {1} {\R{N_B}} \chi _{i,j} \V {u} _{j}} ^2} \NR
}
%
Essentially, we seek for a large-deviation result for i.i.d.\ sum of chi-square random variables now.
To so so, we may apply Lemma 1 in Laurent and Massart \cite [LaM00] for equal weights.
By some calculation, we have upper tail inequality, for \m {x > 0},
%
\DispNum {P,x;ex,ex} {
\NC \MB {P} \SB {\VNm {\M {W} _B \V {u}}^2 \geq 1 + 2 \R {\F {x} {N_B}} +\F {2x} {N_B}}
\leq \NC \Ss {e} ^{-x} \NR
}
%
And lower tail inequality
%
\DispNum {P,x;ex,ex!} {
\NC \MB {P} \SB {\VNm {\M {W} _B \V {u}}^2 \leq 1 + 2 \R {\F {x} {N_B}}}
\leq \NC \Ss {e} ^{-x} \NR
}
%
With substitution
%
\Disp {
\NC \d
= \NC \F {16x^2} {N_B^2} \NR
}
%
naturally \m {\d <1}, and the upper tail can be simplified with \m {x <N_B}, and we get the following lemma.

\Result
{Lemma}
{
For \m {0 <\d <1},
%
\DispNum {P,d;eN,d4} {
\NC \MB {P} \SB {\Nm {\VNm {\M {W} _B \V {u}}^2 - 1} \geq \d}
\leq \NC 2 \Ss {e} ^{-N_B \R {\d} /4} \NR
}
}

\stopsubsection

\startsubsection [title={Operator Norm of Analog Beamformers}]

Here, we consult Haviv and Regev, ``The restricted isometry property of subsampled Fourier matrices'' in \cite [KlM17].
We need their Thm.\ 4.5; it gives a RIP guarantee w.r.t.\ \m {\d}, and probability lower bound, for DFT submatrix.
However, the probability bound they give involve is expressed in terms of big-\m {\O} and big-\m {O}, and since our methodology is 
Fortunately, it can be seen that their argument allows precise constants, and they just did not spell out, which we briefly explain.
Indeed, their Thm.\ 2.2 is a form of hoeffding bound, and the constant \m

So, we have the following modified statement, which has been conformed to our setting.

\Result
{Theorem}
{


}

\stopsubsection

\startsubsection [title={Confirming the Restricted Isometry}]

\stopsubsection

\stopsection
