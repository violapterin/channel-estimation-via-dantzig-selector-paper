\startsection [title={Restricted Isometry of Effective Beamformer}]
\startsubsection [title={Design of Beamformer Matrices}]


Our plan is setting the four precoder matrices to be i.i.d.\ random matrix, hoping that the resulting \m {P} has RIP.

Let us make it simple and set each entry of \m {\M {F} _B} to be i.i.d.\ Normal r.v.\ with mean 0, standard deviation \m {1/2}, multiplied by a normalizing constant \m {\l_B >0}.
We know that the magnitude \m {\M {F} _B \DB {n_R, n_Y}} follows Rayleigh distribution.
\m {\M {W} _B ^\Tr \DB {n_Y, n_R}} is set in an entirely analolous manner.

In addition, let \m {\M {F} _R \DB {n_H, n_R}} be uniformly distributed on the unit circle on the complex plane.
\m {\M {W} _R ^\Tr \DB {n_Y, n_R}} is set in an entirely analolous manner.

Existent result () guarantee that both \m {\M {F} _B ^\Tr} and \m {\M {F} _R ^\Tr} are RIP, and same can be said of \m {\M {W} _B} and \m {\M {W} _R}.
Consider a sparse \m {\V {u}} or unity norm, then existent result guarantees that \m {\M {F} _R ^\Tr \V {u}} has almost unity norm.
If, furthermore, \m {\M {F} _R ^\Tr \V {u}} is sparse, we see that \m {\M {F} _B ^\Tr \M {F} _R ^\Tr \V {u}} has almost unity norm, and the RIP of \m {\M{P}} also follows by construction.

However, it is not clear that \m {\M {F} _B ^\Tr \M {F} _R ^\Tr}, the product of two RIP matrices, have RIP, nor that, \m {\M {P}}, the result of Kronecker product, is RIP.
In fact, the statement seems to be false in general.
Indeed, since a \m {\M {F} _R ^\Tr} of the same \m {\d} is sparse, up to a factor of isometry matrix, and \m {\M {F} _R ^\Tr} is generated randomly, it would seem to us that the probability that \m {\M {F} _R ^\Tr \V {u}} be sparse is exceedingly small.

But all is not lost.
At least we have made sure that \m {\M {F} _R ^\Tr \V {u}} has norm close to unity.
If we have a stronger result, that \m {\M {F} _B ^\Tr \V {u}} is almost unitary even for \It {not-sparse} vectors, then we are done.
That does sound somehow plausible, considering the Normal entries of \m {\M {F} _B ^\Tr} would introduce a Central-limit-theorem-like sum, though the summands are not independent, but weakly dependent.

\startsubsection [title={Confirming the Restricted Isometry}]



To start, use \m {\chi _{i,j}} to denote Complex Standard Normal random variables, and to simplify notation, let them be independent if they have different arguments.
With this,
\PF {cc';w-bb,u}
\DispN {
\NC \Xi
:= \NC \VNm {\M {W} _B \V {u}} \NR
\NC =\NC \R {
   \sum _{i=0} ^{N_B-1} \VNm {\sum _{j=0} ^{N_R-1} \F {1} {\R{N_B}} \chi _{i,j} \V {u} _{j}} ^2} \NR
}
By looking up well known properties of Chi distribution, we get the expression of its PDF
\Disp {
\NC f_{\Xi} \SB {\xi}
=\NC \F {\xi ^{N_B /2 -1} \Ss {e} ^{-\xi /2}}
{2 ^{N_B/2} \G \SB {\F {N_B} {2}}} \NR
}
By Lemma 1 in Laurent and Massart \cite [LaM00],

\Disp {
\NC \MB {P} \SB {\Xi^2 -N_B \geq 2 \R {N_B x} +2x}
\leq \NC \Ss {e} ^{-x} \NR
}
And
\Disp {
\NC \MB {P} \SB {N_B -\Xi^2 \geq 2 \R {N_B x}}
\leq \NC \Ss {e} ^{-x} \NR
}
It can be seen that, with substitution \m {\d = 4x/N_B} and some substitution and loosening, we get


\Result
{Lemma}
{
For \m {\Xi} defined in \Rf {cc';w-bb,u}, we have
\DispN {
\NC \MB {P} \SB {\Xi^2 \geq 1 -\d}
\leq \NC \Ss {e} ^{-N_B \d /4} \NR
}
\DispN {
\NC \MB {P} \SB {\Xi^2 \geq 1 -\d}
\leq \NC \Ss {e} ^{-N_B \d /4} \NR
}
}



