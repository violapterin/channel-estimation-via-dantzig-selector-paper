\startsection [title={Almost-Sparsity of Angular Channel Response}]
\startsubsection [title={Norm of Array Response}]

Let \m {\hat {\V {g}}} be the Dantzig Selector, and let the sparsity index \m {s} be given.
Split \m {\V {g}} into two parts: \m {\V {g} _{\SB{\MC {A}}}}, the largest \m {s} components of \m {\V {g}}, \m {\V {g} _{\SB{\MC {A}}}} the next \m {s} largest components of \m {\V {g}}, and \m {\V {g} _{\SB{\MC {C}}}} are the components complement to \m {\V {g} _{\SB{\MC {A}}}}.
For example, if \m {\V {g} =\IP {-1,3,-4,2,8}}, and \m {S=2}, then \m {\V {g} _{\SB{\MC {A}}} =\IP {0,3,0,0,8}}, \m {\V {g} _{\SB{\MC {B}}} =\IP {-1,0,0,2,0}}, and \m {\V {g} _{\SB{\MC {A}}} =\IP {-1,0,-4,2,0}}.
In this section, the subscripts \m {\MC {A}, \MC {B}, \MC {C}} will keep bearing analogous meaning.

We hope that \m {\VNm {\V {g} _{\SB{\MC {C}}}} _1} is small.
If so, we shall substitute the quantity into the expected square error of DS, thus generalizing the original argument.
Recall that \m {\M {G}} is just the vectorization of \m {\V {g}}, and \m {\M {G}} is the function of \m {\V {a}}, so we seek to establish that \m {\V {a}} is almost sparse.
We define, similarly, \m {\V {a} _{\SB{\MC {A}}}} and \m {\V {a} _{\SB{\MC {C}}}}, with undetermined sparsity index
%
\Disp {
\NC S
\leq \NC \R {N_H}. \NR
}
%
If \m {N_H} is large, we need not bother ourself with the requirement that \m {S} is an integer.

First, we investigate
%
\Disp {
\NC \VNm {\V {a} \SB {\f} _{\SB{\MC {C}}}} _2 ^2
=\NC \VNm {\RB {\M {K}^\Adj \V {a} \SB {\f}} _{\SB{\MC {C}}}} _2 ^2 \NR
}
%
Let \m {\f} be fixed.
Towards that end, 
Introduce
%
\Disp {
\NC \psi \SB {\f, n_H}
:=\NC \RB {
   \RB {
      \RB {\f \; \Rm {Mod}\; \F {2\pi} {N_H}}
      +\F {2 \pi n_H} {N_H}
      +\pi
   } \;
   \Rm {Mod}\; \RB {2\pi}
}
-\pi \NR
}
%
Note that by construction
%
\Disp {
\NC \Nm {\psi \SB {\f, n_H}}
\leq \NC \pi \NR
}
%
And define the so-called Dirichlet kernel
%
\Disp {
\NC D \SB {\psi'}
:= \NC \sum_{n_H=0}^{N_H-1} \Ss {e}^{i n_H \psi'} \NR
}
%
Then observe
%
\Disp {
\NC \RB {\M {K}^\Adj \V {a} \SB {\f}} _{\SB {n_H}}
=\NC \F {1} {N_H} D \SB {\psi \SB {\f, n_H}} \NR
}

Now, by the nature of alternating series, for \m {-\pi \leq x < \pi}, it holds that
%
\Disp {
\NC \Nm {x - \F {x^3} {6}} \leq \NC \Nm {\sin x}. \NR
}

Applying Lemma () to the denominator of () and bounding the nominator by 1, we have
%
\Disp {
\NC \Nm {D \SB {\psi'}}
= \NC \F {\Nm {\sin \SB {N_H \psi'/2}}} {\Nm {\sin \SB {\psi' /2}}} \NR
\NC \leq \NC B \SB {\psi'} \NR
\NC := \NC \F {48} {\Nm {\psi'^2 -24} \Nm {\psi'}} \NR
\NC -\pi \leq \NC \psi' < \pi. \NR
}
%
Thus,
%
\Disp {
\NC \VNm {\V {a} \SB {\f} _{\SB{\MC {C}}}} _1
=\NC \F {1} {N_H}
\VNm {
\RB {
   \sum_{n_H' =0}^{N_H -1}
      B \SB {\psi \SB {\f, n_H'}}
      \V {u} _{n_H'}
} _{\SB{\MC {C}}}
} _1
\NR
}
%
Note that \m {\Nm {B \SB {\psi'}}} is strictly decreasing in \m {\SB {0,\pi}}.
We seek to bound the \quotation {rectangulars} with an integral, and we have to split the cases that \m {N_H} is odd and even.
Anyway, a moment's reflection shows
%
\Disp {
\NC \VNm {\V {a} \SB {\f} _{\SB{\MC {C}}}} _1
\leq \NC \F {1} {N_H} \D \F {N_H} {2\pi} \D 2 \int_{\pi s/N_H}^{\pi} B \SB {\psi'} ^2 d \psi' \NR
\NC = \NC \F {2304} {N_H \pi^6}
\int _{s /N_H} ^1 \F {1} {(24/\pi^2 -x'^2)^2 x'^2} dx' \NR
\NC =\NC \F {1} {2\pi^2 N_H}
\RB {
  -\F {8} {u}
+\F {4\pi^2 u} {24 -\pi^2 u^2}
+\R {6} \pi \tanh^{-1} \SB {\F {\pi u} {2\R {6}}}
}
\Bigg \| _{s/N_H} ^1 \NR
\NC \leq \NC \F {4} {\pi^2} \D \F {1} {s} +\F {0.032670} {N_H} \NR
\NC \approx \NC \F {4} {\pi^2 s} \NR
}


% % Deprecated!
% \Disp {
% \NC R^2
% \leq \NC \F {1} {N_H^2} \D \F {N_H} {2\pi} \D 2 \int_{\pi s/N_H}^{\pi} B \SB {\psi'} ^2 d \psi' \NR
% \NC = \NC \F {2304} {N_H \pi^6}
% \int _{s /N_H} ^1 \F {1} {(24/\pi^2 -x'^2)^2 x'^2} dx' \NR
% \NC =\NC \F {1} {2\pi^2 N_H}
% \RB {
%   -\F {8} {u}
% +\F {4\pi^2 u} {24 -\pi^2 u^2}
% +\R {6} \pi \tanh^{-1} \SB {\F {\pi u} {2\R {6}}}
% }
% \Bigg \| _{s/N_H} ^1 \NR
% \NC \leq \NC \F {4} {\pi^2} \D \F {1} {s} +\F {0.032670} {N_H} \NR
% \NC \approx \NC \F {4} {\pi^2 s} \NR
% }
% In the last step we bound the lower limit of the second and third term with \m {x' =0}, and leave the dominating \m {1/x'}.

\Result
{Lemma}
{
Let \m {\f } be given, and linear array response \m {\V {a} \SB {\f}} defined as in ().
Suppose \m {N_H \geq 4}.
Then, for any instance of \m {\f},
%
\DispNum {a;f',C} {
\NC \VNm {\V {a} \SB {\f} _{\SB{\MC {C}}}} _2
\leq \NC \F {2} {\pi \R{s}} \NR
}
}

Since the value of \m {s} is at our disposal, we try
%
\DispNum {s;nn-hh,} {
\NC s
=\NC \R {\log N_H} \NR
}
%
so that
%
\DispNum {R;2,p'} {
\NC R
\leq \NC \F {2} {\pi \R{\log N_H}} \NR
}

\stopsubsection

\startsubsection [title={Norm of Angular Channel Response}]

Recall that, by \m {\ell_p}-norm inequality,
%
\Disp {
\NC \VNm {\M {g} _{\SB{\MC {C}}}} _2
\leq \NC \VNm {\M {g} _{\SB{\MC {C}}}} _1 \NR
}
%
And just by definition
%
\Disp {
\NC \VNm {\M {G} _{\SB{\MC {C}}}} _F
\leq \NC \VNm {\M {g} _{\SB{\MC {C}}}} _2 \NR
}
%
So it remains to bound \m {\VNm {\M {G} _{\SB{\MC {C}}}} _F}.

Note that
%
\DispNum {G;H,a'-l} {
\NC \VNm {\M {G} _{\SB{\MC {C}}}} _F
=\NC \VNm {\M {H} _{\SB{\MC {C}}}} _F \NR
}
%
And by definition
%
\Disp {
\NC \VNm {\M {H} _{\SB{\MC {C}}}} _F
=\NC \VNm {\sum _{l=0} ^{L-1} \a_l
\V {a} \SB {\f_l} \V {a} \SB {\th_l} ^\Adj} _F \NR
}
%
By triangle inequality, by the property of 2-norm and Frobenius norm,
%
\DispNum {;a'-l,a,f'-l} {
\NC \leq \NC \sum _{l=0} ^{L-1}
\Nm {\a_l}\VNm {\V {a} \SB {\f_l} \V {a} \SB {\th_l} ^\Adj } _F \NR
\NC \leq \NC
\sum _{l=0} ^{L-1} \Nm {\a_l}
\VNm {\V {a} \SB {\f_l}} _2
\VNm {\V {a} \SB {\th_l}} _2 \NR
}
%
It follows that
%
\Disp {
\NC \VNm {\M {H} _{\SB{\MC {C}}}} _F
\leq \NC
\sum _{l=0} ^{L-1} \Nm {\a_l}
\VNm {\V {a} \SB {\f_l} _{\SB{\MC {C}}}} _2
\VNm {\V {a} \SB {\th_l} _{\SB{\MC {C}}}} _2 \NR
\NC \leq \NC
\VNm {\V {a} \SB {\f_l} _{\SB{\MC {C}}}} _2
\VNm {\V {a} \SB {\th_l} _{\SB{\MC {C}}}} _2
\sum _{l=0} ^{L-1} \Nm {\a_l} \NR
\NC \leq \NC
\F {4} {\pi^2 \log N_H}
\sum _{l=0} ^{L-1} \Nm {\a_l} \NR
}

Observe that
%
\Disp {
\NC \MB {E} \SB {\sum _{l=0} ^{L-1} \Nm {\a_l}}
=\NC \F{\R {2} L} {\R {\pi}} \NR
%
\NC \Ss {Var} \SB {\sum _{l=0} ^{L-1} \Nm {\a_l}}
=\NC \RB {1 - \F{2}{\pi}} L \NR
}

We expect that
%
\Disp {
\NC \sum _{l=0} ^{L-1} \Nm {\a_l}
\leq \NC L \R {\F {2} {\pi}}
   +\R {1 -\F {2} {\pi}} \R {L} \RB {\log N_H} ^{3/2} \NR
\NC \approx \NC \R {1 -\F {2} {\pi}} \R {L} \RB {\log N_H} ^{3/2} \NR
}
%
Thus, by Chebyshev inequality,
%
\Disp {
\NC \MB {P} \SB {
   \sum _{l=0} ^{L-1} \Nm {\a_l}
   \geq \R {1 -\F {2} {\pi}} \R {L} \RB {\log N_H} ^{3/2}
}
\leq \NC \F {1} {\RB {\log N_H} ^3} \NR
}
%
Summarizing, we have the following result.

\Result
{Lemma}
{
Let \m {\f} and \m {\th} be uniformly, independently distributed in \m {[0,2\pi)},
and let \m {\V {g} \in \MB {C} ^{N_H^2}} be defined as in ().
Then the bound
%
\DispNum {g-C;4,p'-2} {
\NC \VNm {\M {g} _{\SB{\MC {C}}}} _1
\leq \NC \F {4} {\pi^2} \R {1 -\F {2} {\pi}} L N_H \R {\log N_H} \NR
}
%
holds with probability \m {p} such that
%
\DispNum {1,p;1} {
\NC 1 -p
\leq \NC \F {1} {\RB {\log N_H} ^3} \NR
}
}

\stopsubsection
\stopsection

