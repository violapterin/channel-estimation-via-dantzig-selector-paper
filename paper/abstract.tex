
\starttitle [title={Abstract}]

Multiple-input-multiple-output (MIMO) is widely expected to be used in the next-generation wireless communication.
However, reliable communication in such r\`egime often desires that the channel state information be known at the receiver, necessary for example in beamforming algorithm and channel calibration, but it is usually not easy to estimate real-world wireless channel.
Meanwhile, the millimeter wave channel, which is usually used in tandem with MIMO, often exhibits sparse properties, if the sparsity is exploited, few observations may suffice to estimate the channel.
Such algorithm falls under the study of Compressive Sensing, an active area that has recently emerged.
A promiment estimator is The Dantzig Selector (DS) proposed by Tao and Cand\'es, and it is DS which was first applied in channel estimation context, and as originally perceived, it ideal in certain senses.
However, Orthogonal Matching Pursuit (OMP) later emerges as a solution with lower complexity, and following-up literature, as a result, usually used OMP rather than DS.
In this article, we consider a hybrid structure at both the transmitter and receiver end, each of them having both digital and analog percoders and combiners.
We argue in favor of DS, since DS encompasses a more general setting than OMP, such as in its requirement of restricted isometry property (RIP) only, as well as in its tolerance of noise corruption.
Indeed, we shall refine Tao and Cand\'es's bound, and generalize under current settings, thus giving a viable solution for the estimation of millimeter wave channel.
Moreover, we show explicitly that DS may be cast as a second order cone problem (SOCP), and primal-dual technique is ready to be used.
Simulation is done with both DS and OMP, and we discuss their respective merits as our conclusion.


\stoptitle
