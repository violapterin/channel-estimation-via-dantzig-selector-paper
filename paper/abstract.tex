
\starttitle [title={Abstract}]

Multiple-input-multiple-output (MIMO) is expected to be the next-generation technology for wireless communication.
However, for MIMO system it is not easy to estimate the channel, while reliable communication often requires that the channel state information be known at the receiver, necessary e.g.\ in beamforming and channel calibration.
Meanwhile, the millimeter wave (mm-Wave) band, often used in MIMO context, is known to exhibit sparsity, and emerging work on compressive sensing suggests that if sparsity is exploited, fewer measurement suffices to estimate the channel.
Dantzig Selector (DS), proposed by Tao and Cand\'es, has been introduced early in channel estimation literature.
But Orthogonal Matching Pursuit (OMP) emerges, and having a lower complexity, it ends up being popular.
In this paper, we consider a mm-Wave MIMO channel, and a hybrid structure at both the transmitter and receiver end.
Using pilot signal and designed random beamformer, we employ DS to estimate the channel, arguing in favor of its greater generality, its tolerance of noise corruption, and its lower averaged error, despite of having greater complexity.
We then proved a bound for the expected square error of DS, as well as bounded probability of success, showing explicit dependency on the number of path and channel sparsity.
DS can be cast as a second order cone problem (SOCP), so that feasible code for convex programming may be applied.
Simulation is done on DS and other methods, and we discuss their respective merits.


\stoptitle
