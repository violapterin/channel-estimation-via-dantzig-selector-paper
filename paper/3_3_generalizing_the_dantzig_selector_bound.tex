\startsection [title={Generalizing the Dantzig Selector Bound}]
\startsubsection [title={Technical Lemmata in Complex Case}]

The third and last part of our task is to plug in the almost-sparsity found in Section 3.1, and plug in the probablity found in Section 3.2.
Set for short
%
\DispNum {d-d-gg-gg} {
\NC \V {d} 
= \NC \Hat {\V {g}} -\V {g} \NR
}
%
And our task is to bound the expectation \m {\MB {E} \SB {\V {d}}}.

We argue that \m {\V {g}} can essentially be seen as sparse, and for that purpose we ignore relevent terms.
Also, we do not seek a very tight bound and will approximate \m {\d_{S}} by \m {1}.
And, as shall be seen, it suffices to set
%
\DispNum {g'-g'-2N-Nh} {
\NC \g
= \NC \R {2 \log N_h} \NR
}
%
The lemmata below are taken from \cite [CaT07] and modified slightly in accordance to our settings, except possibly the generalization to complex vector.
%
\Result
{Lemma}
{
%
\DispNum {d-1-dA-C1} {
\NC \VNm {\V {d} _{\SB{\MC {C}}}} _1
\leq \NC \VNm {\V {d} _{\SB{\MC {A}}}} _1
+2\VNm {\V {g} _{\SB{\MC {C}}}} _1 \NR
}
}
%
To show this, observe that with triangle inequality applied respectively on \m {\MC {A}} and \m {\MC {C}}, while recalling that \m {\V {g} =\V {g} _{\SB{\MC {A}}}},
%
\Disp {
\NC \VNm {\V {g}} _1
-\VNm {\V {d} _{\SB{\MC {A}}}} _1
+\VNm {\V {d} _{\SB{\MC {C}}}} _1
-\VNm {\V {g} _{\SB{\MC {C}}}} _1
\leq \NC \VNm {\V {g} +\V {d}} _1 \NR
%
\NC =\NC \VNm {\hat {\V {g}}} _1 \NR
}
%
On the other hand, by construction that \m {\hat {g}} minimizes the \m {\ell_1}-norm,
%
\DispNum {--;g1,g1;} {
\NC \leq \NC \VNm {\V {g}} _1 \NR
}
%
\Result
{Lemma}
{
Let \m {\M {P}} be fixed, and instance of \m {\V {z}} be given.
Then
%
\DispNum {E-h-2N-Nh} {
\NC \MB {E} \SB {\Nm {\IP {\V {z} , \M {P} \DB {:, n_h}}}}
\leq \NC  2 \R {\log N_h} \NR
%
\NC n_h 
=\NC 1, \ldots, N_h. \NR
}
%
holds with probability \m {\geq 1 -N_h^{-1}}.
}
%
Indeed, recall the fact that \m {\M {P} \DB {:, n_g}} has unity \m {\ell_2}-norm, and on the randomness of \m {\V {z}}, the quantity \m {\Nm {\IP {\V {z} , \M {P} \DB {:, n_h}}}} observes standard Normal.
Recall the following bound for \m {Q} function
%
\DispNum {Q-x-12-22} {
\NC Q\SB {x}
\leq \NC \F {1} {2} \Ss {e}^{-x^2/2} \NR
}
%
Particularly, for \m {x =\R {2 \log N_h}},
%
\DispNum {Q-h-1N-Nh} {
\NC Q\SB {\R {2 \log N_h}}
=\NC \F {1} {N_h}. \NR
}
%
\Result
{Lemma}
{
If \m {\M {P}} is \m {\d_S}-RIP,
%
\DispNum {P-d-4N-Nh} {
\NC \VNm {\M {P}^\Adj \M {P} \V {d}} _\infty
\leq \NC  4 \R {\log N_h} \NR
}
%
holds with probability \m {\geq 1 -N_h^{-1}}.
}
%
To show this, by definition
%
\DispNum {z-y-Pg-ny} {
\NC \IP {\V {z} -\RB {\V {y} -\M {P} \hat {\V {g}}}, \M {P} \DB {:,n_y}}
= \NC \IP {\M {P} \hat {\V {g}} -\M {P} \V {g}, \M {P} \DB {:,n_y}} \NR
%
\NC = \NC \IP {\M {P} \V {d}, \M {P} \DB {:,n_y}} \NR
}
%
By construction
%
\DispNum {P-g-Py-Nh} {
\NC \VNm {\M {P} \DB {:,n_y}^\Adj \RB {\V {y} -\M {P} \hat {\V {g}}}} _\infty
\leq \NC \VNm {\M {P}^\Adj \RB {\V {y} -\M {P} \hat {\V {g}}}} _\infty \NR
%
\NC \leq \NC 2\R {\log N_h} \NR
}
%
By triangle inequality, together with \Rf {E-h-2N-Nh}, we have
%
\DispNum {P-d-zP-Nh} {
\NC \VNm {\M {P} \DB {:,n_y}^\Adj \M {P} \V {d}} _\infty
\leq \NC \VNm {\IP {\V {z}, \M {P} \DB {:,n_y}}} _\infty
+\IP {\V {y} -\M {P} \hat {\V {g}}, \M {P} \DB {:,n_y}} \NR
%
\NC \leq \NC 2 \R {\log N_h} +2 \R {\log N_h} \NR
}
%
which implies
%
\DispNum {P-d-4l-Nh} {
\NC \VNm {\M {P}^\Adj \M {P} \V {d}} _\infty
\leq \NC 4 \R {\log N_h}. \NR
}
%
Lastly, the following two lemmata is taken from \cite [CaT07], Lemma 1.
The generalization from real vector spaces to complex spaces is completely straightforward, while meaning of magnitude and inner product has to be changed.
\Result
{Lemma}
{
%
\DispNum {d-2-dA-12} {
\NC \VNm {\V {d}} _2^2
\leq \NC \VNm {\V {d} _{\MC {A} \MC {B}}} _2^2 +\F {1} {S} \VNm {\V {d} _{\SB{\MC {C}}}} _1^2 \NR
}
}
The other one reads
\DispNum {d-2-11-C1} {
\NC \VNm {\V {d} _{\SB{\MC {AB}}}} _2
\leq \NC \F {1} {1-\d_{2S}} \VNm {P _{\MC {A} \MC {B}}^\Tr P d} _2
+\F {\T {\d}_{S, 2S}} {\RB {1-\d_{2S}} \R {S}} \VNm {d_{\SB{\MC {C}}}} _1 \NR
}
Here \m {\T {\d} _{s_1, s_2}} is called the \m {s_1,s_2}-restricted isometry constant, and is defined in \cite [Can05] (which uses \m {\th}), as follows.
Let \m {\M {Q}'} be fixed.
For all \m {s_1}-sparse \m {\V {x}_1}, for all \m {s_2}-sparse \m {\V {x}_2}, whose positions for nonzero components are disjoint,
%
\Disp {
\NC \T {\d} _{s_1, s_2}
= \NC \inf \CB {
\d':\; \Nm {\IP {\M {Q}' \V {x}_1, \M {Q}' \V {x}_2}}
   \leq \d' \VNm {\V {x}_1} \VNm {\V {x}_2}
} \NR
}
%
Here we use a result in \cite [Can05] that relates \m {\T {\d} _{s_1, s_2}} to \m {\d_s}, that is
\DispNum {d-2-ds-s2} {
\NC \T {\d} _{s_1, s_2}
\leq \NC \d_{s_1+s_2} \NR
}
And we need another crude bound to relate \m {\d_{2S}} to \m {\d_{S}}.
By triangle inequality, by power-mean inequality,
\DispNum {Q-x-Qx-x2} {
\NC \VNm {Q' \RB {\V {x}_1 + \V {x}_2}}
%
\leq \NC \VNm {Q' \V {x}_1}
+\VNm {Q' \V {x}_2} \NR
%
\NC \leq \NC \RB {1 +\d _{s}} \RB {\VNm {\V {x}_1} +\VNm {\V {x}_2}} \NR
%
\NC \leq \NC \RB {1 +\d _{s}} \D \R{2} \VNm {\V {x}_1 +\V {x}_2} \NR
}
Therefore
\Disp {
\NC \d_{2s}
\leq \NC \R {2} \d_{s} \NR
}
By the same token,
%
\Result
{Lemma}
{
If \m {\M {Q}'} has \m {\d}-RIP, then, for \m {M=1,2,3,\dots} so that \m {\d _{Ms}} is meaningful,
%
\DispNum {d-s-Md'-d's} {
\NC \d_{Ms}
\leq \NC \R {M} \d_{s} \NR
}
}
%
Thus we have, in view of \Rf {d-2-11-C1},
\Result
{Lemma}
{
If \m {\M {P}} is \m {\d_S}-RIP,
%
\DispNum {d-2-11-C1} {
\NC \VNm {\V {d} _{\SB{\MC {AB}}}} _2
\leq \NC \F {1} {1-\R{2} \d_{S}} \VNm {P _{\MC {A} \MC {B}}^\Tr P d} _2
+\F {\R{3} \d_{S}} {\RB {1-\R{2} \d_{S}} \R {S}} \VNm {d_{\SB{\MC {C}}}} _1 \NR
}
}
%

\stopsubsection

\startsubsection [title={Main Result}]

We are going to combine previous lemmata and show the main result.
For simplicity, set
%
\DispNum {d'-S;18,18;} {
\NC \d_S
\leq \NC \F {1} {8} \NR
}
%
Thus
\DispNum {S-S-4N-NH} {
\NC S
=\NC L \log N_H \NR
}
%
\Result
{Theorem}
{
Let \m {\V {y}}, \m {\M {P}}, \m {\V {g}}, \m {\hat {\V {g}}}, \m {\V {d}} be defined as above.
Then the bound
%
\DispNum {d-2-42-NH} {
\NC \VNm {\V {d}} _2
\leq \NC 4 \R {2L} \log N_H \NR
}
}
%

We start to prove \Rf {d-2-42-NH}.
By the definition of truncation, by \m {\ell_p} norm inequality, by \Rf {P-d-4N-Nh},
%
\DispNum {P-2-PP-NH} {
\NC \VNm {\M {P} _{\MC {A} \MC {B}}^\Tr \M {P} \V {d}} _2
\leq \NC \VNm {\M {P}^\Tr \M {P} \V {d}} _2 \NR
%
\NC \leq \NC \R {S} \VNm {\M {P}^\Tr \M {P} \V {d}} _\infty \NR
%
\NC \leq \NC 4 \R {2 \log N_H} \NR
}
%
By \m {\ell_p}-norm inequality, by the definition of truncation, by \Rf {d-2-11-C1}, and by \Rf {P-2-PP-NH} just above,
%
\Disp {
\NC \VNm {\V {d} _{\SB{\MC {A}}}} _1
\leq \NC \R {S} \VNm {\V {d} _{\SB{\MC {A}}}} _2 \NR
%
\NC \leq \NC \R {S} \VNm {\V {d} _{\SB{\MC {AB}}}} _2 \NR
%
\NC \leq \NC 4 \R {2 \log N_H}
+\F {1} {\R{L \log N_H}} \VNm {\V {d} _{\SB{\MC {C}}}} _1 \NR
}
%
Next, we want to use \Rf {d-1-dA-C1} to eliminate \m {\VNm {d _{\SB{\MC {A}}}} _1}.
%
\DispNum {8,2,nn-hh;} {
\NC \leq \NC 4 \R {2 \log N_H}
+\F {1} {\R{L \log N_H}} \RB {\VNm {\V {d} _{\SB{\MC {A}}}} _1 +2\VNm {\V {g} _{\SB{\MC {C}}}} _1} \NR
}
%
And \Rf {a-2-4N-NH} gives, after arrangement,
%
\DispNum {d-A;8,2;} {
\NC \VNm {\V {d} _{\SB{\MC {A}}}} _1
\leq \NC 4 \R {2 \log N_H}
+2\VNm {\V {g} _{\SB{\MC {C}}}} _1 \NR
%
\NC \leq \NC \RB {16 \R {2} +\F {4} {\pi^2} \R {1 -\F {2} {\pi}}} \log N_H \NR
}
%
We are now in a position to bound \m {\VNm {d} _2^2}.
Using \Rf {d-2-dA-12} again, we have
%
\Disp {
\NC \VNm {\V {d}} _2^2
\leq \NC 32 S \log N_H
+8 \d_{3S} \R {2 \log N_H} 
+\F {\d_{3S}^2} {S} \VNm {\V {d} _{\SB{\MC {C}}}} _1 \NR
}
%
Finally, plug in the bound for \m {\VNm {\V {d} _{\SB{\MC {C}}}} _1}, the last unknown, and we make use of the almost-sparsity condition of \m {\VNm {\V {d} _{\SB{\MC {C}}}} _1}.
Again for simplicity, we keep only the first order small terms.
%
\Disp {
\NC \VNm {\V {d}} _2^2
\leq \NC 32 S \log N_H
+\F {8 \R {2\pi}} {3} \d_{3S} L \R {\F {\log N_H} {N_H}}
+\F {\pi} {9} \F {L} {s^2 N_H} \NR
}
%
Set \m {\d_S =1} for the worst situation, and we get the desired bound.

In addition to the bound, we shall discuss the four sources of error, namely
\DispNum {q-q-2q-ar} {
\NC q
\simeq \NC 2 \RB {q _{\Rm {isom}, B} + q _{\Rm {isom}, R}} +q _{\Rm {nois}} +q _{\Rm {spar}} \NR
}
subscript for restricted isometry, noise, sparsity, respectively.

Quoting previous results, they are
\Disp {
q
\simeq \Ss {e} ^{N_B \d}
}
%
Provided the design values
\Disp {
\NC N_R
\leftarrow \NC \RB {\log N_H} ^{3/2} \NR
%
\NC N_B
\leftarrow \NC \log \log N_H \NR
}
%
And we get the estimation
\Disp {
\NC q
=\NC \MC {O} \SB {\F {1} {\RB {\log N_H}^3}} \NR
}
%
\stopsubsection
\stopsection


