
\blank [big, force]

\Title {\TitleText}
\Subtitle {III. Numerical Experiments}
\blank [big]

\Subtitle {\AuthorText}
\blank [big]

\Subsubtitle {\DateText}

\page [yes]
% XXX % % XXX % % XXX % % XXX % % XXX % % XXX %
\Frame {Organization}
{
\I Introduction

\I Problem Setting and Proposed Method

\I Bounding the Error Norm

\I Numerical Experiments

\I Conclusion and Key References
}
% XXX % % XXX % % XXX % % XXX % % XXX % % XXX %
\Frame {Background}
{
\I \Emph {Multiple-input multiple-output} (MIMO) communication system has been proposed to be incorporated into the 5G specification.
However, its hardware overhead increases complexity and power consumption.

\I The design of new algorithms that addressed these issues requires the expression of channel to be known.

\I Estimation of MIMO channels results in high complexity, due to the large number of antennae.

\I Meanwhile, millimeter wave (mm-Wave) channels often exhibit \Emph {sparse} properties, and what if we exploit the sparsity to save for some measurements?
}
% XXX % % XXX % % XXX % % XXX % % XXX % % XXX %
\Frame {Compressive Sensing}
{
\I The situation that the number of model parameters is much larger than the number of measurements has been addressed by \Emph {compressed sensing} (CS) theory

\I A CS problem involves inverting a wide, rectangular matrix.
Indeed, with insufficient (even noisy) measurements, this does not make sense.

\I But it turns out that, when the signal is sparse, few measurements may be sufficient for the reconstruction.
And \Emph {Dantzig Selector} (DS), proposed by Cand\`es and Tao (2006) in the advent, is a possible solution.
}

