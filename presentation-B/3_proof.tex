\Frame {The Need of a Modified Proof}
{
\I \Emph {Q1}: DS is shown to work for \Emph {sparse} vectors, but is \m {\M {H}} really sparse?

\I \Emph{Q2}: Dantzig Selector requires the measurement matrix to observe restricted isometry property (more below), but is it true for \m {\M {P}}, and under what \Emph {design} of \m {\M {F}_B, \M {F}_R, \M {W}_B, \M {W}_R}?

\I \Emph{Q3}: Candès and Tao established the case for real matrices, but does \Emph {complex} case apply?

\I \Emph{Q4}: Candès and Tao showed the Big-\m {\MC {O}} of error norm depends on (in our case) \m {N_H}, but does the algorithm work successfully for \Emph {high probability}?
}
% XXX % % XXX % % XXX % % XXX % % XXX % % XXX %
\Frame {Proof Strategy}
{
\I \m {\M {H}} is not sparse, but \m {\M {K}^\Adj \M {H} \M {K}} is.
Remove the largest-magnitude \m {S} components of \m {\V{g}}, and call the remaining vector \m {\M {g} _{\SB{\MC {C}}}}, so that it remains to bound \m {\VNm {\M {g} _{\SB{\MC {C}}}} _1}.

\I Establish that all four beamformers satisfy RIP, thus \m {\M{P}} satisfies RIP too, for high probability.

\I We then establish a modified bound of error norm based on Candès and Tao.
}
% XXX % % XXX % % XXX % % XXX % % XXX % % XXX %
\Frame {Restricted Isometry Property}
{
\I \m {\V {g}} is called \Emph {\m {s}-sparse}, if only \m {s} of its components are nonzero.

\I We say that \m {\M{P}} satisfies the \Emph {restricted isometry property} (RIP) of sparsity \m {s} with respect to \m {0 < \d_s < 1}, if, for all \m {s}-sparse \m {\V {g}},
\Disp {
\NC \RB {1-\d_s} \VNm {\V {g}} _2^2
\leq \NC \VNm {\M{P} \V {g}} _2^2
\leq \RB {1+\d_s} \VNm {\V {g}} _2^2
}

\I Verbally, \m {\M{P}} is \Emph {almost unitary} up to relative error \m {\d_s}.
}
% XXX % % XXX % % XXX % % XXX % % XXX % % XXX %
\Frame {\m {\V{g}} is Almost Sparse}
{
\I Suppose for concreteness,
\Disp {
\NC s
=\NC \R {\log N_H},
\Q S
= L s^2,
\Q \d_S
\leq \F {1} {8} \NR
}

\I We may show that 
\Disp {
\NC \VNm {\M {g} _{\SB{\MC {C}}}} _1
\leq \NC \F {4} {3} \RB {\log N_H}^2 \NR
}

\I holds for probability \m {p}, with
\Disp {
\NC 1 -p
\leq \NC 2 \exp \SB {-\F {9L} {\pi}} \NR
}
}
% XXX % % XXX % % XXX % % XXX % % XXX % % XXX %
\Frame {Design of Beamformers}
{
\I Considering \Emph {Q2}: If each entry of \m {\M {P}} is i.i.d., then it's likely that, by whatever variation of \Emph {Central Limit Theorem}, for unit vector \m {\V {u}}
\Disp {
\NC \MB {E} \SB {\VNm {\M {P} \V {u}} _2 ^k}
\approx \NC 1
\Q \Ss {in\; distribution} \NR
}

\I Unfortunately, they are \Emph {not i.i.d.}, and that need closer scrutiny.

\I Try to set each entry of \m {\M{F}_B} (resp.\ \m {\M{W}_B}) to be i.i.d.\ Complex Standard Normal, up to a design constant

\I And let each entry of \m {\M{F}_R} (resp.\ \m {\M{W}_R}) be uniformly distributed argument and unity magnitude, up to a design constant
}
% XXX % % XXX % % XXX % % XXX % % XXX % % XXX %
\Frame {RIP of Effective Beamformers}
{
\I Is it true that if \m {\M {F} _R ^\Tr}, \m {\M {F} _B ^\Tr}, \m {\M {W} _R}, and \m {\M {W} _B} all have RIP, then \m {\M {P}} has?

\I Indeed, if so, \m {\M {P}} has approximately \m {2\d_s}-RIP, if there is no sparsity constraint:
\Disp {
\NC \VNm {P} _2
=\NC \VNm {\M {F}_B^\Tr \M {F}_B^\Tr} _2 \VNm {\M {W}_B \M {W}_R} _2 \NR
}

\I I argue that this is true for sparse \m {\V {g}}.
In fact, if \m {\M {K} ^\Adj \M {H} \M {K} =\M {U} \Rm {diag} \SB {\V {d}} \M {U} ^\Adj}, for some \m {S}-sparse \m {\V {d}} and unitary \m {\M {U}}, then

\Disp {
\NC \VNm {\M {W} _B \M {W} _R \M {H} \M {F} _R \M {F} _B} _2
=\NC \VNm {
   \M {W} _B \M {W} _R \M {K} \M {U}
   \R {\Rm {abs} \SB {\V {d}}} \D
   \R {\Rm {abs} \SB {\V {d}}}
   \M {U} ^\Adj \M {K} ^\Adj \M {F} _R \M {F} _B} _2 \NR
%
\NC \eqsim \NC \VNm {\M {H}} _2 \RB {1 +2\d_s} \NR
}
}
% XXX % % XXX % % XXX % % XXX % % XXX % % XXX %
\Frame {Isometry of Digital Beamformers}
{
\I For \m {0 <\d_s <1}, unit vector \m {\V {u}}, \m {\M {W}_B} (resp.\ \m {\M {F} _B ^\Adj}) satisfies
%
\Disp {
\NC \Nm {\VNm {\M {W} _B \V {u}} _2^2 - 1}
\geq \NC \d_s \NR
}

\I holds for probability \m {p}, with
\Disp {
\NC 1 -p
\leq \NC 2 \Ss {e} ^{-N_B \R {\d_s} /4} \NR
}
}
% XXX % % XXX % % XXX % % XXX % % XXX % % XXX %
\Frame {RIP of Analog Beamformers}
{
\I Suppose
%
\Disp {
\NC N_R
\geq \NC \F {s} {\d_s^2} \RB {\log s}^2 \log N_H \NR
}
%
\I Then \m {\M {W}_R} (resp.\ \m {\M {F} _R ^\Adj}) has \m {\d_s}-RIP...

\I for probability \m {p}, with
\Disp {
\NC 1 -p
\leq \NC \RB {\F {\d_s} {N_H s}} ^{1/3} \NR
}
}
% XXX % % XXX % % XXX % % XXX % % XXX % % XXX %
\Frame {Error Norm of Dantzig Selector}
{
\I In answer of \Emph {Q3}, it can be checked that the generalization to complex vectors only introduces the bound of a multiple \m {\R {2}}.

\I then the following bound holds for high probability \m {p}:
\Disp {
\NC \VNm {\V {d}} _2
%
\leq \NC 3.29 \R {L} \log N_H +4.56 \R {L} \R {\log N_H}^3 \NR
}

\I and we may bound failure probability \m {1-p}...
}
% XXX % % XXX % % XXX % % XXX % % XXX % % XXX %
\Frame {Big-\m {O} of Success Probability (1/2)}
{
\I Denote the four events of failure, subscript respectively for restricted isometry, noise, sparsity,
\Disp {
\NC q
\eqsim \NC 2 \RB {q _{\Rm {iso}, B} + q _{\Rm {iso}, R}} +q _{\Rm {no}} +q _{\Rm {sp}} \NR
}

\I Then previous discussions
\Disp {
\NC q _{\Rm {iso}, B}
\lesssim \NC 2 \Ss {e} ^{-N_B /12} \NR
%
\NC q _{\Rm {iso}, R}
\lesssim \NC 2 N_H ^{-1/3} \RB {\log N_H} ^{-1/6} \NR
%
\NC q _{\Rm {no}}
\lesssim \NC N_H ^{-2} \NR
%
\NC q _{\Rm {sp}}
\lesssim \NC 2 \Ss {e} ^{-9L/\pi} \NR
}
}
% XXX % % XXX % % XXX % % XXX % % XXX % % XXX %
\Frame {Big-\m {O} of Success Probability (2/2)}
{
\I If we provide the design values
\Disp {
\NC N_B
\gtrsim \NC 4 \log N_H \NR
%
\NC N_R
\gtrsim \NC 16 L \RB {\log N_H}^2 \NR
}

\I It follows that
\Disp {
\NC q _{\Rm {sp}}
\lesssim \NC q _{\Rm {no}}
\lesssim q _{\Rm {iso}, B}
\eqsim q _{\Rm {iso}, R}
}

\I Then we get the estimation
\Disp {
\NC q
\eqsim 2 N_H ^{-1/3} \NR
}
}


