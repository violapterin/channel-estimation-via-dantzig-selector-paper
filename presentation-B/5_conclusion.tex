\Frame {Conclusion}
{
\I We are concerned with effective and estimation of MIMO mm-wave channel by exploiting its sparsity, and studied Dantzig Selector (DS).

\I We modified DS for present setting: for the almost-sparse condition, for the angular domain, for the beamformer matrix \m {\M {P}} in our setting, and for complex vectors.

\I We bound the expected square error and , in explicit expression of system parameters such as \m {N_H}, \m {N_Y}, and \m{L}, and discussed with the big \m {\MC{O}} of the successful probability
}
% XXX % % XXX % % XXX % % XXX % % XXX % % XXX %
\Frame {Future Work}
{
\I Investigate the reason of high complexity and reduce it

\I Run for bigger values of \m {N_H}, \m {N_Y}, and \m{L} if possible

\I Maybe try the real case and use existent code for DS

\I More literature review on OMP and other methods
}
% XXX % % XXX % % XXX % % XXX % % XXX % % XXX %
\Frame {References}
{
{\tfx
\I E. Candès and J. Romberg, \It {\m {\ell_1}-magic: Recovery of Sparse Signals via Convex Programming}, retrieved from \Tt {www.acm.caltech.edu/l1magic/downloads/l1magic.pdf}, 2005.

\I E. Candès and T. Tao, ``The Dantzig Selector: Statistical estimation when \m {p} is much larger than \m {n}'', The annals of Statistics 35(6), 2313–2351, 2007.

\I M.P. Friedlander and M.A. Saunders, ``Discussion: the Dantzig Selector'', The Annals of Statistics 35(6), 2385–2391, 2007.

\I B. Klartag and E. Milman, \It {Geometric Aspects of Functional Analysis}, Springer, 2017.

\I B. Laurent and P. Massart, ``Adaptive estimation of a quadratic functional by model selection'', Annals of Statistics, 1302–1338, 2000.
}
}

